\section{$\Lambda_\mathrm{c}$ baryon measurements}
In this section, we summarize the history of \Lambdac\ measurements that have been published up to this point. We hope to convince the reader that the physics of production of charmed baryons is not yet understood and definitely compelling. Unfortunately, the facts, that the direct reconstruction is typically done from 3-body decays and that the lifetime times the speed of light is only $c\tau = 60\,\upmu$m, makes these analyses relatively challenging.

\subsection{\Lambdac\ baryon}

\begin{table}[htb]
\caption[Decay channels of \Lambdac\ that can be used in its reconstruction.]{\label{tab:lcDecayCahnnels}Decay channels of \Lambdac\ that can be used in its reconstruction~\cite{PDG}. Note that the $\Lambda_c^+ \rightarrow \mathrm{p}^+ + \mathrm{K}^- + \uppi^+$ channel can have various resonances as an intermediate step in the decay.}
\begin{center}
\begin{tabular}{llc}
\toprule
\multicolumn{2}{l}{Decay channel} & Branching ratio  \\
\midrule
$\Lambda_c^+ \rightarrow$ &  p$^+$ + K$^- + \uppi^+$ (all) & 5.0$\,\%$ \\
  & p$^+$ + K* & $1.6\,\%$ \\
  & $\Delta^{++}$ + K$^-$ & $0.86\,\%$ \\
  & $\Lambda^0(1520) + \uppi^+$ & $1.8\,\%$ \\
  & Nonresonant & $2.8\,\%$ \\
$\Lambda_c^+ \rightarrow$ &  p$^+$ + $\overline{\mathsf{K}}^0$ & $2.3\,\% $ \\
$\Lambda_c^+ \rightarrow$ & $ \Lambda^0 + \uppi^+$ & $1.07\,\%$ \\
\bottomrule
\end{tabular}
\end{center}
\end{table}


The \Lambdac\ baryon has a quark content of c, u, and d~\cite{PDG}. It is the lightest charmed baryon with the rest mass of $m = 2286.46 \pm \SI{0.14}{\mega\electronvolt}/c^2$\@. Its decay length is, however, quite short $c\tau = 60.0 \pm \SI{1.8}{\micro\metre}$ which makes the reconstruction challenging, especially in heavy-ion collisions. The measurement of \Lambdac\ is possible via the leptonic channels $\Lambdac^+ \rightarrow \mathrm{e}^+ + \upnu_\mathrm{e} + \mathrm{X}$ or $\Lambdac^+ \rightarrow \upmu^+ + \upnu_\upmu + \mathrm{X}$, however, the energy information is lost because of the outgoing neutrino, which cannot be detected, and therefore, the leptons cannot be separated from the ones from D and B mesons decay. Possible channels for the direct reconstruction of the \Lambdac\ with all the decay products detected are listed in Table~\ref{tab:lcDecayCahnnels}. In these channels all the decay particles can be reconstructed and, thus, the mass and momentum can be fixed.

\subsection{$\Lambda_\mathrm{c}/$D$^0$ ratio in e+e and e+p collisions}

The $\Lambda_\mathrm{c}/$D$^0$ ratio is sensitive to the hadronization properties of the charm quark. In Table~\ref{tab:eeep} we summarize the results of the $\Lambda_\mathrm{c}/$D$^0$ ratio measurements in e+e and e+p collisions. Other than the measurement performed by ZEUS on data recorded from e+p collision at HERA~I, the data are consistent between various experiments which suggests that \Lambdac\ and D$^0$ hadronize via the same process in e+e and e+p collisions. This process is believed to be the vacuum fragmentation of the charm quark.

\begin{table}[!htb]
\caption{\label{tab:eeep}Summary of the measurements of the \Lambdac/D$^0$ in e+e and e+p collisions at different center-of-mass energies.}
\begin{center}
\begin{tabular}{lr@{$\,\pm\,$}l@{$\,\pm\,$}lccr}
\toprule
 & $\Lambda_\mathrm{c}/$D$^0$ & stat.& syst. & System & $\sqrt{s}\,$(GeV) & Notes \\
\midrule
CLEO~\cite{CLEO} & 0.119 & 0.021 & 0.019 & e+e & 10.55 & \\[3mm]

ARGUS~\cite{ARGUS1, ARGUSee} & 0.127 & \multicolumn{2}{l}{0.031} & e+e & 10.55 & \\[3mm]

LEP average~\cite{LEPfragmentation} & 0.113 & 0.013 & 0.006 & e+e & 91.2 & \\[3mm]

\multirow{3}{*}{ZEUS DIS \cite{ZEUSep3} } &   \multicolumn{3}{c}{\multirow{3}{*}{$0.124 \pm 0.034^{+0.025}_{−0.022}$}} & \multirow{3}{*}{e+p} & \multirow{3}{*}{320} & $1 < Q^2 < 1000\,$GeV$^2$, \\
&\multicolumn{3}{c}{}&&&$0 < \pt < 10\,$GeV$/c$,\\
&\multicolumn{3}{c}{}&&&$0.02 < y < 0.7$ \\[3mm]

\multirow{3}{*}{\shortstack[l]{ZEUS $\upgamma$p \\ HERA I \cite{ZEUSep}}} & \multicolumn{3}{c}{\multirow{3}{*}{$0.220 \pm 0.035^{+0.027}_{−0.037}$}} & \multirow{3}{*}{e+p} & \multirow{3}{*}{320} & $130 < W < 300\,$GeV, \\
& \multicolumn{3}{c}{}&&& $Q^2 < 1\,$GeV$^2$, $|\eta| < 1.6$,\\
& \multicolumn{3}{c}{}&&& $\pt > 3.8\,$GeV$/c$\\[3mm]


\multirow{3}{*}{\shortstack[l]{ZEUS $\upgamma$p\\ HERA II \cite{ZEUSep2}}} & \multicolumn{3}{c}{\multirow{3}{*}{$0.107 \pm 0.018^{+0.009}_{−0.014}$}} & \multirow{3}{*}{e+p} & \multirow{3}{*}{320} & $130 < W < 300\,$GeV,\\
& \multicolumn{3}{c}{}&&& $Q^2 < 1\,$GeV$^2$, $|\eta| < 1.6$,\\
& \multicolumn{3}{c}{}&&& $\pt > 3.8\,$GeV$/c$ \\

\bottomrule
\end{tabular}
\end{center}
\end{table}  




\subsection{$\Lambda_\mathrm{c}$ in p+p and p+Pb collisions at the LHC\label{LcInPPandPPb}}
The Large Hadron Collider (LHC) is the largest proton and heavy-ion collider in the world, with a circumference of 27$\,$km. The experiments at the LHC can benefit from the large cross-section  Measurements of \Lambdac\ in both, p+p and p+Pb collisions have been reported at  the experiments ALICE (A Large-Ion-Collider Experiment\nomenclature{ALICE}{A Large-Ion-Collider Experiment}) and the LHCb. 

ALICE~\cite{ALICE} is a multipurpose detector dedicated to heavy-ion physics with excellent particle-identification (PID) capabilities. The layout of ALICE is similar to that of STAR (see chapter~\ref{STARchapter} for more detail). The key detectors that enable the $\Lambda_\mathrm{c}$ measurement are the Time-Projection Chamber (TPC) for tracking and particle identification (PID), the Time-Of-Flight (TOF) and Transition-Radiation (TRD) detectors, used for additional PID, and importantly, the Inner Tracking System (ITS) for vertexing. Until now, the ITS consisted of 6 layers of silicon detectors, but currently, it is undergoing a major overhaul upgrade that will greatly benefit the future open-heavy-flavor measurements. The ITS will consist of 7 layers of silicon detectors, all based on the MAPS technology.

The LHCb~\cite{LHCb} is dedicated to the physics of beauty and charm and has recently joined the heavy-ion physics program. It is a 20$\,$m long spectrometer arm designed for forward rapidities. The $\Lambda_\mathrm{c}$ measurement especially benefits from the new Vertex Locator (VELO) \cite{VELO} pixel detector. VELO is followed by the first RICH for PID, the main tracker, the second RICH, and calorimeters for additional PID.

\begin{figure}[!htb]
\centering
\includegraphics[height=6.8cm]{img/ALICEpppPbModels}
\includegraphics[height=6.9cm]{img/ALICEpppPbRapidity}
\caption[The $\Lambdac/$D$^0$ ratio measured in p+p and p+Pb collisions at ALICE\@.]{The $\Lambdac/$D$^0$ ratio measured in p+p and p+Pb collisions at ALICE~\cite{AliceLcPPPPb} and the LHCb~\cite{LHCbPrompt, LHCbPrivate}. In the top panel vs \pt\ and in the bottom panel vs rapidity.}
\label{fig:pppPb}
\end{figure}

The measurement of the ratio of $\Lambdac/$D$^0$ from ALICE~\cite{AliceLcPPPPb} and the preliminary ratio from the LHCb~\cite{LHCbPrompt, LHCbPrivate} are shown in Figure~\ref{fig:pppPb}. The left-hand-side plot shows the ratio of $\Lambda_\mathrm{c}/$D$^0$ versus \pt\@ from ALICE from both p+p and p+Pb collisions at mid-rapidity and the right-hand-side figure plots the $\Lambdac/$D$^0$ from ALICE and LHCb against rapidity $y$\@. The \Lambdac\ were reconstructed at ALICE from several decay channels: $\mathrm{\Lambda_c^\pm \rightarrow \uppi^\pm + K^\mp + p^\pm}$, $\mathrm{\Lambda_c^\pm \rightarrow K^0_s + p^\pm}$, and $\mathrm{\Lambda_c^\pm \rightarrow \Lambda + e^\pm + \upnu_e}$. However, only the results from the first two are shown in this figure to decrease the size of the error bars.  The ratio in p+p was compared to Monte Carlo p+p event generators PYTHIA8 with Monash tune and a tune that includes a model of string fragmentation beyond leading color~\cite{PYTHIA8strings}, DIPSY tune~\cite{DIPSY}, and Herwig7 which uses a cluster hadronization mechanism~\cite{HERWIG}\@. The PYTHIA tune with the string fragmentation beyond leading color tune is closest to the measured ratio, however all of these generators dramatically underestimate the data. Moreover, the generators do not depend on rapidity and fail to reconstruct the trend in the data. This suggest that new tunes have to be developed to take these measurements into account.

The $\Lambdac/$D$^0$ was also measured in p+Pb collisions at ALICE and is consistent with the one in p+p within error bars. The data were also compared to a calculation~\cite{LandbergShao} obtained from a parametrization of the p+p data, using the EPS09NLO nuclear modification factors. There is a slight tension between this prediction and the data although not a significant one.




% \begin{figure}
% \centering
% \includegraphics[width=.66\textwidth]{img/ALICEpp}
% \caption{Invariant mass spectra of the p+K+$\uppi$ triplets at ALICE in p--p collisions at $\sqrt{s} = 7\,$TeV divided into $p_\mathrm{T}$ bins~\cite{ALICE_QM_14}.}
% \label{fig:ppALICE}
% \end{figure}
% 
% \begin{figure}
% \centering
% \includegraphics[width=\textwidth]{img/ALICEpPb}
% \caption{Invariant mass spectra of the p+K+$\uppi$ triplets at ALICE in p--Pb collisions at $\sqrt{s_\mathrm{NN}} = 5.02\,$TeV divided into $p_\mathrm{T}$ bins~\cite{ALICE_QM_14}.}
% \label{fig:pPbALICE}
% \end{figure}

% Two collision systems were analyzed~\cite{ALICE_QM_14} at ALICE: the p-p collisions at the center-of-mass energy $\sqrt{s} = 7\,$TeV and p--Pb collisions at the center-of-mass energy per nucleon $\sqrt{s_\mathrm{NN}} = 5.02\,$TeV.
% For both systems, a Bayesian approach to the PID was used. A particle species is used if the probability that it is this particular particle species is the highest. Somewhat looser topological cuts for the secondary vertices were used compared to the D$^0$ analyses.
% 
% Invariant mass spectra of the p+K+$\uppi$ triplets is shown in Figure~\ref{fig:ppALICE} for p--p collisions and in Figure~\ref{fig:pPbALICE} for p--Pb collisions. The signal can be divided into 4 and 6 bins in $p_\mathrm{T}$ for p--p and p--Pb collisions, respectively\@. For both systems, the efficiency corrections are currently being calculated to obtain the $p_\mathrm{T}$ spectra.

\subsection{$\Lambda_\mathrm{c}$ in Pb+Pb collisions at the LHC}

$\Lambda_\mathrm{c}$ measurement in Pb+Pb collisions~\cite{AlicePbPb} from ALICE has also been recently published, using the $\sqrt{s_\mathrm{NN}} = 5.02\,$TeV data. The measurement was obtained, using the $\mathrm{\Lambda_c^\pm \rightarrow K^0_s + p^\pm}$ channel.

\begin{figure}[!htb]
\centering
\includegraphics[width=.9\textwidth]{img/LcOverD0ALICE}
\caption[The $\Lambdac/$D$^0$ ratio measured in Pb+Pb collisions by the ALICE experiment, compared to the p+Pb and p+p collisions and model calculations.]{The $\Lambdac/$D$^0$ ratio measured in Pb+Pb collisions by the ALICE experiment~\cite{AlicePbPb}, compared to the p+Pb and p+p collisions (left) and model calculations~\cite{PlumariGreco, Catania, LcPPbModelShaoSong, ShaoSongPP,  ShaoSong} (right).}
\label{fig:LcD0ALICE}
\end{figure}

The measurement of the ratio of $\Lambdac/$D$^0$ is shown in Figure~\ref{fig:LcD0ALICE}. The left-hand-side plot shows comparisons with p+p and p+Pb measurements from the previous section. The $\Lambdac/$D$^0$ ratio is significantly higher in Pb+Pb collisions. This increase is usually explained by the process of quark coalescence. The right-hand-side figure is compared to model calculations. The Catania model uses two different treatments of hadronization: One which uses only quark coalescence and one where fragmentation takes over in high \pt\@. The scenario with coalescence only is closer to the data. The Shao-Song model~\cite{LcPPbModelShaoSong, ShaoSongPP} implements coalescence in such a way that quark combination takes place with a fraction of momentum of the hadron and does not take into account the spatial and momentum distribution of the quarks in a hadron. The ratio of single charm baryons and mesons $R_\mathrm{B/m}$ is treated as a free parameter in the model. The curve obtained with  $R_\mathrm{B/m} = 0.425$, needed to obtain the p+p and p+Pb data, does not describe the Pb+Pb data and the one with $R_\mathrm{B/m} = 1.2$, which fits the Pb+Pb data, has a tension with the p+p and p+Pb data.

\begin{figure}[!htb]
\centering
\includegraphics[width=.9\textwidth]{img/LcRAA_ALICE}
\caption[Nuclear modification factor $R_\mathrm{AA}$ of the \Lambdac\ baryon, measured by the ALICE experiment.]{Nuclear modification factor $R_\mathrm{AA}$ of the \Lambdac\ baryon, measured by the ALICE experiment~\cite{AlicePbPb}\@. Left: Compared to model calculations~\cite{Catania}; Right: Compared non-strange D mesons, D$_\mathrm{s}$~\cite{AliceD}, and charged particles~\cite{AliceCharged}.}
\label{fig:LcRaaALICE}
\end{figure}

Thanks to the measurements of \Lambdac\ in p+Pb, described in previous section, ALICE was able to publish the nuclear modification factor \Raa\ which is shown in Figure~\ref{fig:LcRaaALICE}\@. The baseline was obtained by scaling the \Lambdac\ yield in p+Pb collisions at $\sqrt{s_\mathrm{NN}} = 5.02\,$TeV by $1/A$\nomenclature{$A$}{Nucleon number}\ ($A=208$)\@. Even though the $\Lambdac/$D$^0$ ratio is much higher than in p+Pb collisions, the \Raa\ is bellow unity in the measured \pt\ range, because the D$^0$ are also suppressed in Pb+Pb collisions. 
In the left-hand-side plot, a comparison to Catania model calculations~\cite{Catania}, that include three different hadronization mechanisms, is shown. The green short-dashed line illustrates a scenario that includes both vacuum fragmentation and charm-quark coalescence in Pb+Pb, but only fragmentation in p+p. The orange long-dashed line includes only coalescence in Pb+Pb collisions and coalescence plus fragmentation in p+p. Finally, the blue solid line represents fragmentation plus coalescence in both collision systems. This comparison shows that it is crucial to also describe the \Lambdac\ production mechanism in p+p collisions at the LHC energies to draw conclusions about the \Raa\@. The limited precision of this first measurement, however, does not allow to discern between the different hadronization scenarios.
The right-hand-side panel of Figure~\ref{fig:LcRaaALICE} compares the \Raa\ of \Lambdac\ in 0--80$\,\%$ most central Pb+Pb collisions to that of non-strange D mesons, D$_\mathrm{s}$, and charged particles in 0--10$\,\%$ most central collisions. The \Raa\ of charged particles is lower by more than 2$\sigma$, compared to non-strange D mesons, which are compatible within the uncertainties with the D$_\mathrm{s}$\@. ALICE observes a hint of enhancement of the \Lambdac\ baryon, compared to the D$^0$ mesons by~$\sim1.7\sigma$~\cite{AlicePbPb}\@. This observation is qualitatively consistent with the scenario where significant portion of the charm quarks hadronize via the coalescence mechanism.

\section{Future measurements of the \Lambdac\ baryon at the LHC}
At the time of writing this thesis, the LHC has concluded its second run and is closed for the Long Shutdown II\@. This period can be used for upgrades of its detectors and analyses of the data recorded during the Run 2\@. Here, we list several upgrades and analyses that concern the \Lambdac\ baryon.


\subsection{$\Lambda_\mathrm{c}$ in p+A collisions at the LHCb with SMOG}
The LHCb has recently joined the heavy-ion physics program with the unique capability to use different particle species and energy ranges thanks to its new fixed-target mode of operation. The System for Measuring Overlap with Gas (SMOG)~\cite{SMOG} is a detection system primarily dedicated to precision luminosity measurement. It injects an inert gas (He, Ne, or Ar) with the pressure of $\sim$10$^{-7}\,$mbar into the beam pipe to be able to perform beam-gas imaging, but this inert gas can also serve as a fixed target for the beam. The center of mass energy per nucleon can vary inside $69\,\text{GeV} \leq \sqrt{s_\mathrm{NN}} \leq 115\,$GeV for the beam energy from 2.5$\,$TeV to 7$\,$TeV\@. So far, the $\Lambda_\mathrm{c}$ have been analyzed in p+Ar collisions at $\sqrt{s_\mathrm{NN}} = 110\,$GeV\@.

\begin{figure}[!htb]
\centering
\includegraphics[width=.5\textwidth]{img/LHCb}
\caption[Invariant mass spectrum of the p+K+$\uppi$ triplets in p+Ar collisions with fixed target.]{Invariant mass spectrum of the p+K+$\uppi$ triplets in p+Ar collisions with fixed target at $\sqrt{s_\mathrm{NN}} = 110\,$GeV at LHCb with SMOG~\cite{LHCbQMpresentation}.}
\label{fig:LHCb}
\end{figure}

The decay channel $\mathrm{\Lambda_c^\pm \rightarrow \uppi^\pm + K^\mp + p^\pm}$ was used for this measurement~\cite{LHCbQMpresentation}. The invariant mass of the pK$\uppi$ triplets can be seen in Figure~\ref{fig:LHCb}. Detector effects are still under study for the $\Lambda_\mathrm{c}$ spectra at LHCb.

\subsection{ALICE Inner-Tracking System (ITS) upgrade}
\nomenclature{ITS}{Inner-Tracking System}

\begin{figure}[!htb]
\centering
\includegraphics[width=.5\textwidth]{img/LcITS}
\caption[Projection of statistical uncertainties of the \Lambdac/\dzero\ measurement with the improved resolution of ALICE ITS\@.]{Projection of statistical uncertainties of the \Lambdac/\dzero\ measurement with the improved resolution of ALICE ITS using $1.7\times 10^{10}$ central collisions (0--10$\,\%$), corresponding to an integrated luminosity of 10$\,$nb$^{-1}$~\cite{ITS_CDR}.}
\label{ITS_Lc}
\end{figure}

The innermost part of the ALICE detector, the Inner-Tracking System (ITS),  is undergoing a major overhaul upgrade with emphasis on open-heavy-flavor measurements~\cite{ITS_CDR}\@. In concept, the ITS sensors are similar to the STAR-HFT-Pixel~\cite{HftFinal} layers. The ITS is going to consist of 7 layers of MAPS pixel sensors out of which the innermost layer will be inside the beam pipe. This will provide the ITS with unparalleled tracking resolution while also improving the speed of the detector. Figure~\ref{ITS_Lc} shows the projection of statistical uncertainties \Lambdac/\dzero\ ratio measurement in 0--10$\,\%$ most central Pb+Pb collisions. The new ITS will allow for a differentiated measurement of the \Lambdac\ invariant yield in both \pt\ and centrality. 

