In this chapter, we explain basic concepts of physics of ultra-relativistic heavy-ion collisions, such as the ones that take place at RHIC\@. First, we briefly talk about the Standard Model of particle physics and the theory of strong interactions --- quantum chromodynamics --- then we introduce the physics of ultra-relativistic heavy-ion collisions with experimental signatures of the new state of matter -- the quark-gluon plasma. 

\section{The Standard Model}

\begin{figure}[!htb]
\centering
\includegraphics[width=\textwidth]{img/standard_model}
\caption[Elementary particle in the standard model.]{\label{standard_model}Elementary particle in the standard model. Adopted from~\cite{standardModel}\@.}
\end{figure}

The Standard Model of particle physics (SM\nomenclature{SM}{Standard Model}) is a theoretical framework for elementary particles and their interactions. This model includes several elementary particles in its framework: 12 fermions with their antifermions with spin 1/2 and 5 bosons with whole-number spins (see Fig.\ \ref{standard_model})\@.

The rapid development of detection techniques of cosmic rays and accelerator experiments after the Second World War resulted in a rapid discovery of numerous new particles. A new particle ZOO was being filled rapidly and it was the task of particle physics to find patterns among them. Two physicists --- Gell-Man and Zweig --- found the quark theory independently and thus mostly completed this task. Most of the particles -- hadrons -- were found to consist of, so called, quarks with a quantum number, so called flavor. At the time, 3 quarks were known: u (up), d (down), and s (strange) with their corresponding antiparticles. Today, these are referred to as light-flavor quarks. In nature, quarks are never found as separate particles. Rather, they are bound in hadrons. This phenomenon, related to strong interaction, is called confinement (more in Section~\ref{confinement})\@.

In 1974, the discovery of an unexpected particle at two experiments almost simultaneously changed the landscape of particle physics for good. A resonance at the invariant mass of 3.1$\,$GeV/$c^2$, initially called `J', was found at the precise e$^+$+e$^-$ pair spectrometer on the Alternating-Gradient Synchrotron (AGS\nomenclature{AGS}{Alternating-Gradient Synchrotron}) at Brookhaven National Laboratory (BNL)\@. This find was published in~\cite{JpsiBNL}. A day later, the same journal received a paper from the experiment SPEAR at Stanford Linear Accelerator Laboratory (SLAC\nomenclature{SLAC}{Stanford Linear Accelerator Laboratory})~\cite{JpsiSLAC} that discovered a sharp resonance at the same invariant mass, wishing to call it $\psi$\@. In the end, his particle was called J/$\psi$ and was found to be a long predicted new quarkonium --- a meson consisting of a quark and its antiquark --- in this case of a newly discovered charm quark (c)\@. In 1976, both directors of their respective experiments: Samuel Chao Chung Ting and Burton Richter received a shared Nobel prize for physics~\cite{nobelJpsi}\@.

The next family of heavy quarks followed as the third generation of quarks was theoretically predicted by Makoto Kobayashi and Toshihide Maskawa to explain the violation of the CP--symmetry in the weak-force theory~\cite{KobayashiMaskawa}\@. There were other explanations for this phenomenon, but finally, the discussion was settled with the discovery of the bottomonium $\Upsilon$ at the E288 experiment on the Bevalac accelerator at Fermilab~\cite{bottomDiscovery}\@. The $\Upsilon$ meson consists of the bottom quark b and its antiparticle $\overline{\mathrm{b}}$\@. The last quark, so called top (t) was discovered by the D0 experiment at the Tevatron accelerator at Fermilab by measuring the invariant mass of jets~\cite{topQuark}. 

Nowadays, 6 quark flavors are known, divided into 3 generations (or families)\@. The first generation is composed of of u and d quarks, the second consists of s and c, and the third one contains b and t. Strong theoretical and experimental evidence is mounting against the existence of a fourth or further generations~\cite{fourGensPhysLettB,fourGensPRL}.

To the 3 generations of quarks, there are 3 families of leptons. These are fermions that unlike quarks do not interact strongly and thus do not form hadrons. The electron\nomenclature{$\mathrm{e}^-$}{electron} was discovered by J.J.\ Thompson in the end of the 19th century~\cite{thomson1897cathode} in an experiment with a cathode in an electric field\@. This later led him to formulate the plum pudding model of atom. The electron's antiparticle --- the positron (e$^+$\nomenclature{$\mathrm{e}^+$}{positron}) --- was discovered much later (in 1930) from the study of cosmic rays in magnetic field by Carl D.\ Anderson~\cite{positron}\@. The same person with his colleague Seth Neddermeyer later discovered the muon ($\upmu^-$) in cosmic rays~\cite{muon}\@. At the time, the muon was theorized to be the Yukawa particle, i.e.\ the mediating particle of the strong nuclear force in the nucleus. This was later found out to be false as the muon does not interact strongly and another particle was found in its place --- the pi-meson (or pion,~$\uppi$\nomenclature{$\uppi$}{pion})~\cite{pion} a meson, consisting of the u and d quarks, that has similar mass~\cite{PDG}, but interacts strongly. The third and final generation of charged leptons was discovered at SLAC by finding the tau-lepton (tauon, $\uptau$\nomenclature{$\uptau$}{tauon})~\cite{tau} at an e$^+$+e$^-$ accelerator\@. This concluded the search for the standard-model particle charged leptons. Today we know 3 families of leptons, that contains the electron, muon, and tauon, together with their antiparticles.

In addition to the charged leptons, there are also neutral ones, called neutrinos ($\upnu$\nomenclature{$\upnu$}{neutrino})\@. %The first indirect evidence of them was theorized by Fermi by formulizing the theory of $\beta$ decay~\cite{Fermi}\@. In this decay, nuclei emit electrons and positrons with energies that are not discrete. Rather, they are emitted on a spectrum of energies that suggest that a third particle must be involved in the decay. Otherwise, this process would violate the conservation-of-energy law. The missing particle was first theorized to be the neutron that is responsible for a large portion of mass of most nuclei, but is electrically neutral. Soon, this was found not to be the case as the missing particle would have to be too massive for the $\beta$ decay. Therefore, it must be a new particle, called neutrino.
Neutrinos proved somewhat elusive to detect as they do not interact electromagnetically or strongly so even though they are relatively common in nature, they mostly traverse through matter unimpeded. The only interaction, they participate in, is the weak one. Therefore, large volumes and masses of detection material are needed which was realized by C.L.\ Cowan and F.\ Reines in 1954 when they constructed a large detector next to a nuclear reactor and detected neutrinos for the first time~\cite{CowanNeutrinos}\@.

Like their charged-lepton brethren, neutrinos come in three generations electron- , muon-, and tauon-neutrinos ($\upnu_\mathrm{e}$, $\upnu_\upmu$, and $\upnu_\uptau$, respectively) and their antiparticles ($\overline{\upnu}_\mathrm{e}$, $\overline{\upnu}_\upmu$, and $\overline{\upnu}_\uptau$, respectively). The number of neutrino families has been experimentally limited to three by the particle experiments at the Large Electron-Positron (LEP\nomenclature{LEP}{Large Electron-Positron collider}) collider at CERN\nomenclature{CERN}{European Organization for Nuclear Research (derived from Conseil Européen pour la Recherche Nucléaire)}\@.

The right-hand side of Figure~\ref{standard_model} is occupied by bosons. These are whole-number spin particles that mediate interactions between particles. The gluons (g) are responsible for the strong interaction (more on that in Section~\ref{QCD}) and therefore interact only with quarks and themselves. Photons ($\upgamma$) generate the electromagnetic interaction and interact with charged particles. The weak force is mediated by the heavy W and Z bosons: The W have an elementary charge ($+$ or $-1\,e$\nomenclature{$e$}{Elementary charge}); The Z bosons are electrically neutral and slightly heavier. The weak force interacts with all the SM fermions and is in many ways peculiar. Out of the known forces, it is the only one that does not conserve flavor. It is also the only force (other than the Higgs field or --- perhaps --- gravity) that interacts with the neutrinos. All of the SM bosons are also called gauge bosons, because their interaction follow the, so called, gauge symmetry.

While the photon was arguably discovered with the theoretical explanation of the photo-electric effect by Einstein and was at the forefront of development of the quantum theory, the other bosons took longer to be experimentally detected. The gluon was detected in 1979 in three-jet events generated from e$^+$--e$^-$ collisions at the accelerator PETRA at DESY~\cite{gluons}\@. The W~\cite{WBosonUA1,WBosonUA2} and Z~\cite{ZDiscovery} bosons were measured also in Europe at the Super Proton-antiproton Synchrotron (SPS\nomenclature{SPS}{Super Proton-antiproton Synchrotron}) at CERN\@.

When attempting to unify the electromagnetic and weak theories into a unified electroweak theory, a problem was discovered that the rest masses of the W and Z bosons were not compatible with the gauge symmetry that was needed. Moreover, divergences were found when calculating W--W boson scattering. These issues were elegantly solved by existence of a new boson with spin 0 which was proposed by Peter Higgs. This boson, therefore, bears his name and was finally discovered in 2012 at the Large Hadron Collider (LHC) by the experiments ATLAS~\cite{HiggsAtlas} and CMS~\cite{HiggsCMS}\@. 

The discovery of the Higgs boson concluded the search for Standard-Model elementary particles. Right now, the physics of the Higgs boson is in an era of precision measurements of its decays and interactions. In fact, its mass has been measured at $125.38\pm0.14\,$GeV$/c^2$~\cite{HiggsMass}, the most precisely measured of all the massive bosons. The Standard Model still, however, does not explain large fields of physics that were observed through cosmological experiments like the, so called, dark matter and dark energy. The Standard Model itself provides room for questions such as the, hierarchy problem of why the masses of particles are set to their measured values. Also, the question of the dominance of matter over anti-matter and the quantum gravity remain to be solved. Therefore, the search for  new particles at the LHC or at other experiments is not over. The hope is that at the 10- or 100-TeV scale, there are more particles to be found which would fundamentally change our understanding of physics. 





\section{\label{QCD}Quantum Chromodynamics and strong interaction}
The strong interaction derives its name from its relative strength, compared to other forces within the standard model. The strong coupling constant is approxiamtely $\approx 1$, compared to the electromagnetic coupling constant ($\approx 10^{-2}$) and weak coupling constant ($\approx 10^{-6}$)\@. It binds quarks into hadrons and has enough strength to bind protons and neutrons into nuclei, despite their electromagnetic repulsion.

The theory of the strong interaction is called Quantum ChromoDynamics (QCD\nomenclature{QCD}{Quantum ChromoDynamics}) which is a non-Abelian gauge field theory that describes the strong interaction between quarks and gluons. The quarks posses a quantum number called color charge which can have values of 1, 2, or 3 or red, green, and blue. The requirement for 3 colors is found e.g.\ due to the spin $J = 3/2$ of the baryon $\Delta^{++}$ (u$^\uparrow$u$^\uparrow$u$^\uparrow$) which otherwise could not exist due to the Pauli exclusion principle. These 3 colors can be represented by the SU(3), therefore QCD is non-Abelian.

The QCD Lagrangean $\mathcal{L}$ can be written as
\begin{equation}
 \mathcal{L} = \sum_{q}\overline{\psi}_{q,a}(i\gamma^\mu \partial_\mu \delta_{ab} - g_s \gamma^\mu t_{ab}^CA_\mu^C - m_q\delta_{ab})\psi_{q,b} - \frac{1}{4}F_{\mu\nu}^AF^{\mu\nu}_A
\end{equation}
where $q$ is the quark flavor, $\psi_{qa}$ is the quark-field spinor with $a$ as a color index which runs between 1 and $N_q = 3$\@. The $\gamma^\mu$ stands for the Dirac $\gamma$ matrices and $\delta_{ab}$ is the Kronecker Delta. The $g_s$ is the strong coupling constant, the $t_{ab}^C$ are $3\times3$ matrices that correspond to the 8 generators of the SU(3) group. The $A_\mu^C$ is the color field where $C$ stands for the gluon colors, running from 1 to $N_q=8$\@. The field tensor $F^A_{\mu\nu}$ is defined as
\begin{equation}
 F_{\mu\nu}^A = \partial_\mu A^A_\nu - \partial_\nu A^A_\mu - g_s f_{ABC} A^B_\mu A_\nu^C
\end{equation}
where $f_{ABC}$ is the structure constant of the SU(3) group.


\subsection{Running coupling constant}
Similarly to QED, the QCD is a renormalized theory~\cite{renormalizationTHOOFT,renormalizationBOLLINI}\@. As such, the QCD coupling constant is floating in order to avoid ultraviolet divergencies.

In QED, the coupling constant $\alpha$ is not a constant number either. Rather, it is a function of the energy scale $Q^2$ of the process. The reason for this phenomenon are virtual pairs of electrons and positrons that get created with larger energy and screen the electric charge. The scale dependency of the coupling constant can be formulated in terms of the $\beta$ function 
\begin{equation}
\begin{gathered}
 Q\frac{\dd\alpha}{\dd Q} \equiv  \alpha\beta(\alpha) \\
 \beta(\alpha) =  \beta_0\frac{\alpha}{\pi} + \beta_1 \left(\frac{\alpha}{\pi}\right)^2 + \dots \,,
 \end{gathered}
\end{equation}
where the terms $\beta_0$, $\beta_1$, \dots correspond to 1 loop, 2 loops, etc. The first term is equal to
\begin{equation}
 \beta_0 = 2/3\,.
\end{equation}
This restricts the case of one loop diagrams ($\beta(\alpha)\simeq \beta_0\frac{\alpha}{\pi}$). Then $\alpha(Q^2)$ becomes
\begin{equation}
 \alpha(Q^2) =  \frac{\alpha (Q_0^2)}{1 - \frac{\beta_0 \alpha (Q_0^2)}{2\pi} \ln(\frac{Q^2}{Q_0^2})}\,,
\end{equation}
where $Q_0$ is an arbitrary scale at which the $\alpha(Q^2)$ is known from empirical measurements.

The situation changes, however, in the QCD case. New terms arise from the gluon self-interaction and the first term becomes
\begin{equation}
 \beta_0^\mathrm{QCD} =  \frac{2N_f - 11N_C}{6}\,,
\end{equation}
where $N_C = 3$ is the number of colors and $N_f$ is the number of flavors. From this, the new form of scale dependency can be derived as 
\begin{equation} \label{alphaQCD}
 \alpha_s(Q^2) = \frac{12\pi}{(33 - 2 N_f) \ln (\frac{Q^2}{\LambdaQCD^2})}\,,
\end{equation}
where the arbitrary scale, typical for QCD, is in this case $Q_0 \equiv \LambdaQCD \approx 220\,$MeV\nomenclature{\LambdaQCD}{Typical QCD scale, infrared cutoff for perturbative QCD}\@.

Since most of the QCD problems are impossible to solve analytically, the perturbation theory is often used. For it to be valid, the energy scale must be $\alpha_s \ll 1$ and $Q \gg \LambdaQCD$\@. The $\LambdaQCD$ scale is, therefore, used as an infrared cutoff for the perturbative QCD\@.

Similarly to QED, the quark color charge is also effectively weakened by the QCD renormalization by the creation of virtual quark-antiquark pairs. Unlike in QED however, virtual gluons around the quarks can strengthen their color charge. This effect is called anti-screening and is directly caused by the negative term in Eq.\ \eqref{alphaQCD}\@.






\subsection{Confinement\label{confinement}}

Quarks are never observed as free particles, rather they are always bound into objects that are overall colorless, so called hadrons. This is caused by the behavior of the strong force with increasing distance $r$\@. The strength of the coupling rises beyond all limits, however at these scales, perturbation theory is no longer valid.

The efforts to derive the correct form of the quark--(anti)quark potential from first principles are currently still ongoing. Therefore, within the framework of the string model~\cite{LundString}, the quark--quark potential is described by the empirical formula
\begin{equation} \label{strongV}
 V_\mathrm{strong} (r) = - \frac{4}{3} \cdot \frac{\alpha_s}{r} + kr\,.
\end{equation}

The first term in Eq.\ \eqref{strongV} is a Coulomb-like potential that dominates at small distances. However, as the distance grows ($r \gtrsim \alpha_s$) the potential becomes governed by the second term.  At this point a string is created between the two quarks and, as the distance increases, it will reach a critical distance $r_c$, at which the energy density is enough to create a quark--antiquark pair. These new quarks start immediately interacting with the original particles which reduces the overall potential and creates colorless objects once again.

\subsection{Asymptotic freedom}

The behavior of the strong coupling constant leads to peculiar behavior of the strong interaction. At high energy scales and very small distances the $\alpha_\mathrm{s}$ becomes arbitrarily small
\begin{equation}
 Q^2 \rightarrow \infty \Rightarrow \alpha_\mathrm{s} \rightarrow \infty\,.
\end{equation}
This phenomenon is called asymptotic freedom as, at very large energy scales, quarks and gluons can effectively move freely, i.e.\ they behave in an opposite manner to the electromagnetic interaction where the coupling constant gets larger with higher energy.

% \section{Heavy quarks}

\section{Ultra-relativistic heavy-ion collisions}
One of the scientific frontiers of the current physical research is the study of properties of matter under extreme conditions. The highest man-made temperatures and pressures yet can be achieved in heavy ion-ion and proton-proton collisions in particle accelerators.
In such facilities, a new state of matter --- the strongly-coupled quark-gluon plasma (sQGP)~\cite{QGPdiscovered} is created\@. One of such accelerators is the Relativistic Heavy Ion Collider (RHIC -- described in Section~\ref{RHIC}) located in Brookhaven National Laboratory in the USA\@. It is capable of reaching the collision energy per nucleon of \snnFull\@. The collisions are called ultra-relativistic because the energy of the nucleons is much higher than their rest masses \nomenclature{$E$}{Energy}\nomenclature{$m$}{Mass (or rest mass)}($E \gg m$). Note that in this chapter, we use natural units where the speed of light in vacuum is equal to $c = 1$, the Boltzmann constant $k = 1$, and the reduced Planck constant $\hslash = 1$\@. 

\section{Phase diagram of the quark-gluon plasma (QGP)}

% Vymenit obrazek
\begin{figure}[!htb]
\centering
\includegraphics[width=0.5\textwidth]{img/phaseDiagram}
\caption[The QCD phase diagram.]{\label{fig:phaseDiagram}The QCD phase diagram. Taken from~\cite{bnlSite}.}
\end{figure}
 

Experimental evidence and theoretical predictions suggest that with high enough temperature (around \SI{170}{\mega\electronvolt}) the usual nuclear matter can enter a new state, the so called strongly-coupled quark gluon plasma~\cite{QGPdiscovered}\@. Figure~\ref{fig:phaseDiagram} shows an illustration of the QCD phase diagram. 
\textcolor{red}{
 The y-axis shows the temperature and the x-axis shows the baryon density, which is the density of baryons minus anti-baryons. This density corresponds directly to the baryon-chemical potential $\eta$\nomenclature{$\eta$}{Baryon-chemical potential} 
}
which represents the potential of the difference in numbers between baryons and anti-baryons. In the bottom left corner, we can see the hadronic matter, where the quarks are always trapped -- confined inside hadrons (e.g.\ the protons at rest have a baryon chemical potential equal to $\eta = \SI{938}{\mega\electronvolt}$ and the temperature of $T \approx 0$\nomenclature{$T$}{Temperature}). 
As we go higher in temperature in heavy-ion collisions, the baryon density gets lower (the number of anti-baryons, compared to baryons, is increasing). At one point, a transition into a new state of matter occurs --- the quark-gluon plasma is created. The temperature is so high that the quarks and gluons are no longer trapped inside hadrons --- they become deconfined. 

\begin{figure}[!htb]
\centering
\includegraphics[width=0.8\textwidth]{img/lightCone}
\caption[The space-time evolution of a heavy-ion collision.]{The space-time evolution of a heavy-ion collision. Taken from~\cite{helen}.}
\label{fig:lightCone}
\end{figure}

At the top RHIC energies ($\snn = 200\,$GeV) and at the LHC energies for the Pb+Pb collisions (e.g.\ $\snn > \SI{2.76}{\tera\electronvolt}$ in the collider mode), the experimental evidence and lattice--QCD (lQCD\nomenclature{lQCD}{lattice--QCD}) calculations point to a crossover (second-order) transition between the hadron matter and the sQGP\@. lQCD calculations suggest, however, that at lower temperatures and higher baryon chemical potential, the transition changes to a first-order one. Therefore, in between, there has to be a critical point. RHIC beam energy scan program, in which \snn\ is scanned by lowering the energy of the accelerated nuclei, is an ongoing endeavor to find this critical point~\cite{BESII}. 


The space-time evolution of a heavy-ion collision is shown in Figure~\ref{fig:lightCone}. When we look at the proper time around $\tau \simeq 1\,$fm$/c$, the local thermodynamic equilibrium is established and from this point on, the system can be described hydrodynamically. This is when the system is going throgh the QGP phase, in which the quarks are deconfined, until it cools down to the temperature of the chemical freeze-out \nomenclature{$T_\mathrm{ch}$}{Temperature of chemical freeze-out}$T_\mathrm{ch} \simeq 165\,$MeV~\cite{ALICE_lightFlavor,STAR_LightFlavor}\@. From now on, the single quarks become confined again in hadrons -- baryons and mesons. The hadrons can, however still interact with each other because the density is high enough; this state of the system is called the hadron gas. After some time, the distances between the hadrons become long enough that the mean free path of the hadrons becomes higher than the volume of the medium. This is called the kinetic freeze-out with the corresponding temperature measured as \nomenclature{$T_\mathrm{fo}$}{Temperature of kinetic freeze-out}$T_\mathrm{fo} \simeq 100\,$MeV. At this point, the average bulk velocity is still about $\sim$1/2$c$ when the hadrons become free-streaming and can be picked by the detectors.





\section{Experimental signatures of the QGP}
Even though the QGP was theoretically predicted by the lattice QCD, it took decades, before its formation was experimentally proven at ion--ion colliders such as the SPS at CERN and RHIC at BNL\@.  
This was not done by one single measurement, but rather several analyses of the collision products that draw the overall picture of the medium properties. These measurements include spectra of the outgoing particles, their azimuthal distribution, or the properties of jets. Today, the formation of the QGP very-high-energy heavy-ion collisions is seldom disputed, but these measurements still serve as probes into the properties of this novel strongly-interacting state of matter and --- with ever increasing precision --- provide the needed restrictions to theoretical calculations that model the QGP's behavior.

\subsection{The nuclear modification factor \Raa}
One of the ways of looking into the properties of the QGP is to compare the spectra of particles produced in p+p collisions $\dd N_\mathrm{AA}/\dd \pt$ and in heavy-ion collisions $\dd N_\mathrm{AA}/\dd \pt$, divided by the average number of binary nucleon-nucleon collisions $\langle N_\mathrm{coll} \rangle$. The ratio of these two is called the nuclear modification factor \nomenclature{\Raa}{Nuclear modification factor}\Raa
\begin{equation}
\Raa = \frac{\dd N_\mathrm{AA}/\dd \pt}
{\langle N_\mathrm{coll} \rangle\, \dd N_\mathrm{pp}/\dd \pt}
\end{equation}
and describes how much the spectrum is modified in nucleus--nucleus collisions.

\begin{figure}[!htb]
\centering
\includegraphics[width=0.8\textwidth,trim={0 9cm 0 0},clip]{img/phenix_Raa}
\caption[\Raa\ vs \pt\ for $\uppi^0$, electrons from open-heavy-flavor decays, and direct photons.]{\label{PhenixRaa}\Raa\ vs \pt\, measured by the Phenix experiment in central Au+Au collisions for several identified hadrons, electrons from open-heavy-flavor decays, and direct photons. Taken from~\cite{PhenixDecadal}\@.}

\end{figure}


\begin{figure}[!htb]
\centering
\includegraphics[width=0.8\textwidth]{img/RaaHydro}
\caption[Nuclear-modification factors \Raa\ and \Rcp\ vs \pt\ for identified $\uppi$ and p, measured by STAR at \snnFull\ and at $\snn = 62.4\,$GeV\@.]{\label{RaaHydro}Nuclear-modification factors \Raa\ and \Rcp\ vs \pt\ for identified $\uppi$ and p, measured by STAR at \snnFull\ and at $\snn = 62.4\,$GeV\@. Taken from~\cite{RaaPiP}\@.}

\end{figure}

A beautiful illustration of the modification of particle spectra in central heavy-ion collisions is shown in Figure~\ref{PhenixRaa}\@. It summarizes \Raa\ measurements of identified particles, performed by the Phenix experiment at RHIC in central Au+Au collisions at \snnFull\@. Its data-points clearly demonstrate that different particles interact differently with the QCD medium, e.g.:


\begin{itemize}
 \item $\uppi^0$: pions are greatly suppressed in central Au+Au collisions, compared to p+p. This is attributed to momentum loss due to interactions of quarks with the medium and also of formed mesons in the hadron-gas phase.
 \item heavy-flavor electrons: one of the big surprises at RHIC, even after the observation of jet
quenching, was that the suppression for electrons resulting from heavy-flavor electrons was almost as strong as that observed
for light mesons, despite the fact that heavy-flavor quarks were not expected to couple
strongly to the medium.
 \item direct photons: the prompt photons, coming from the initial stages of the collision, are not expected to interact with the medium in a significant way. This can be seen at the intermediate and high \pt, although there is a hint of suppression at the two data-points with highest \pt\@. At low \pt, an enhancement is observed in Au+Au collisions, because of thermal photons generated in later stages of the expansion of the medium.
\end{itemize}

Similarly to \Raa, we can study the nuclear modification by comparing the particle yield in central heavy-ion collisions to peripheral ones, divided by the ratio of the number of binary collisions in these centrality brackets. This ratio is called modiication factor \Rcp\ and is defined as
\begin{equation}
 \Rcp = \frac{\dd N_\mathrm{central}/\dd \pt}
{\dd N_\mathrm{peripheral}/\dd \pt} \cdot
\frac{\langle N_\mathrm{coll} (\text{peripheral}) \rangle}
{\langle N_\mathrm{coll} (\text{central}) \rangle}\,.
\end{equation}
Figure~\ref{RaaHydro} shows the \Raa\ and \Rcp\ of identified $\uppi$ and p in Au+Au collisions at \snnFull\ and at $\snn = 62.4\,$GeV\@. \Raa\ of $\uppi$ is compared to several theoretical predictions that include ideal (non-viscous) hydrodynamic calculations and energy loss of quarks with high momenta in the QGP~\cite{Alam, Wang, Vitev}\@. 



\subsection{Azimuthal distribution of the collision, the $v_n$ coefficients, and NCQ scaling}
\begin{figure}[!htb]
\centering
\includegraphics[width=0.5\textwidth]{img/vnATLAS}
\caption[$v_n$ coefficients vs centrality in Pb+Pb collisions.]{\label{vnATLAS}$v_n$ coefficients vs centrality in Pb+Pb collisions. Taken from~\cite{vnATLAS}\@.}

\end{figure}

Another way of probing the sQGP medium is measuring the distribution of outgoing particles in the azimuthal angle $\phi$\@. To evaluate the properties of this distribution, it is usually recalculated in terms of the Fourier expansion
\begin{equation}
E\frac{\dd^3 N}{ \dd p^3} = \frac{1}{2\pi}\frac{\dd^2 N}{\pt \dd \pt \dd y} \left(1 + 2\sum_{n=1}^\infty v_n \cos[i(\phi - \psi_n)] \right)
\end{equation}
where $\psi_n$ is the spatial plane of symmetry for the $n$-th harmonic. The first three $v_n$ coefficients ($v_1$\nomenclature{$v_1$}{Direct flow}, $v_2$\nomenclature{$v_2$}{Elliptic flow}, $v_3$\nomenclature{$v_3$}{Triangular flow}, \dots) are called the direct flow, the elliptic flow, and triangular flow coefficients for $v_1$, $v_2$, and $v_3$, respectively. They can be measured using the formula
\begin{equation}
v_n = \langle \cos[n(\phi - \psi_n)] \rangle\,,
\end{equation}
where the angular brackets denote the average over all particles in an event.



\begin{figure}[!htb]
\centering
\includegraphics[width=\textwidth]{img/massOrderingv2.pdf}
\caption[$v_2$ as a function of \pt\ of several hadron species in four centrality regions of Au+Au collisions.]{\label{massOrdering}$v_2$ as a function of \pt\ of several hadron species in four centrality regions of Au+Au collisions at \snnFull\@. Taken from~\cite{StrangeAndChargedv2paper}.}

\end{figure}
Figure~\ref{vnATLAS} shows the second and further $v_n$ coefficients of charged particles~\cite{vnATLAS}, all the way to $v_6$, as functions of the collision centrality\@. The shape of the $v_n$ is mainly driven by two competing phenomena: Looking at the elliptic flow $v_2$, we can notice that, as we move from central (from 0$\,\%$ on the right-hand side) to more peripheral collisions, the $v_2$ rises up to $\approx50\,\%$ centrality. This can be attributed to increasing eccentricity in the initial collision as in this centrality region, the $v_2$ value is mostly driven by the pressure gradient. When the impact parameter of the two spherical nuclei increases, the shape of the overlapping region becomes more oblong, therefore the pressure gradient increases in the direction of the shorter axis. In more peripheral collisions (left-hand side of Figure~\ref{vnATLAS}), however, this trend is stopped as the size of the system decreases. $v_2$ falls in peripheral collisions as the outgoing particles do not have a chance to interact as much.

When looking at the triangular flow $v_3$, we can observe that the overall trend is similar to $v_2$, although it is overall smaller. The initial rise in central collisions is caused by the increase in fluctuations in the initial shape of the fireball as the system gets smaller. The downward trend in peripheral collisions is attributed to less interactions in smaller systems like in the case of $v_2$\@. Further flow coefficients $v_4$ -- $v_6$ show that the $v_n$ decrease with rising $n$ in intermediate centralities as well.


 
\begin{figure}[!htb]
\centering
\includegraphics[width=0.5\textwidth]{img/NCQ_scaling.pdf}
\caption[$v_2/n_\mathrm{q}$ plotted vs $K E_\mathrm{T}/n_\mathrm{q}$ for several hadron species in Au+Au collisions.]{\label{NCQscaling}$v_2/n_\mathrm{q}$ plotted vs $K E_\mathrm{T}/n_\mathrm{q}$ for several hadron species in Au+Au collisions at \snnFull\@. Taken from~\cite{NCQscalingPhenix}.}

\end{figure}


Another way to differentiate the azimuthal anizotropy of ion-ion collisions is to look at the $v_n$ coefficients of different particle species. Figure~\ref{massOrdering} shows the $v_2$ for several identified hadrons in Au+Au collisions measured by the STAR experiment. The solid yellow band in the plot~\ref{massOrdering}(c) represents a non-flow estimate, where particles would fly from the initial collision without interacting with one another. The solid and dashed lines correspond to ideal (non-viscous) hydrodynamic calculations~\cite{idealHydroReview} for (from top to bottom) $\uppi$, K, p, $\Lambda$, $\Xi$, and $\Omega$\@. These calculations overpredict the data at $\pt > 2\,$GeV$/c$, however at low \pt, they predict the $v_2$ relatively well. From this, we can conclude that the viscosity in the QGP is very low, and thus the QGP is close to an ideal liquid. 


\begin{figure}[!htb]
\centering
\includegraphics[width=\textwidth]{img/brokenNCQ.pdf}
\caption[$v_2/n_\mathrm{q}$ plotted vs $K E_\mathrm{T}/n_\mathrm{q}$ for several hadron species in Pb+Pb collisions measured by ALICE.]{\label{brokenNCQ}$v_2/n_\mathrm{q}$ plotted vs $K E_\mathrm{T}/n_\mathrm{q}$ for several hadron species in Pb+Pb collisions measured by ALICE\@. Taken from~\cite{ALICEncq}.}

\end{figure}

Another observation, that we can draw from Figure~\ref{massOrdering}, is that the $v_2$ is higher for particles with lower mass. This approximate phenomenon is called mass ordering. 
If we plot the $v_2$, however, divided by the number of valence quarks $n_\mathrm{q}$\nomenclature{$n_\mathrm{q}$}{number of valence quarks in a hadron} as a function of the transverse mass (or transverse energy)
\begin{equation}
 K E_\mathrm{T}  = \sqrt{m^2 + \pt^2}\,,
\end{equation}
divided by $n_\mathrm{q}$ as well, just like in Figure~\ref{NCQscaling}, we observe that this function aligns remarkably for all the measured hadron species. This phenomenon is called the number-of-constituent-quarks (NCQ\nomenclature{NCQ scaling}{Number-of-Constituent-Quarks scaling}) scaling and suggests that the $v_2$ is driven predominantly by the motion in the quark stage. When performing a higher-precision measurement of the $v_2$ at higher collision energies, however, such as in Figure~\ref{brokenNCQ} performed by ALICE at the LHC, the scaling is observed to work only approximately. This can be attributed to the collective behavior in the hadron stage where hadrons interact as single particles.



\subsection{Jet quenching and parton energy loss}

One of the most important signatures that confirmed the creation of the QGP in heavy-ion collisions is the, so called, jet quenching. High-\pt\ partons must be created in the initial stages of the collision as later, the energy density of the medium is not sufficient to boil off such high-energy partons. These partons loose energy in the medium and later form jets that can further interact with the medium and loose energy even further. This phenomenon is called jet quenching.

\begin{figure}[!htb]
\centering
\includegraphics[width=.7\textwidth]{img/Jet_quenching}
\caption[Correlations of high-\pt\ charged particles measured in Au+Au, d+Au, and p+p collisions.]{\label{jetQuenching}Correlations of high-\pt\ charged particles measured in Au+Au, d+Au, and p+p collisions. The trigger particle is measured in the \pt\ range of $4\,$GeV$/c < \pt < 6\,$GeV$/c$\@. Taken from~\cite{jetQuenching }\@.}

\end{figure}


One of the observable signatures of jet-quenching is the correlation of high-energy hadrons in azimuth $\phi$ in central heavy ion-ion collisions. Most high-energy partons are created in binary processes where two partons have opposite direction of movement ($\Delta\phi = \pi$) due to momentum conservation. This is clearly shown in Figure~\ref{jetQuenching} where the trigger particles have a transverse momentum of $4\,$GeV$/c < \pt < 6\,$GeV$/c$, but the measured particles are required to have only $\pt > 2\,$GeV$/c$\@. In p+p and d+Au collisions, we observe a peak of correlated particles on the near-side ($\Delta \phi = 0$) and on the away-side ($\Delta \phi = \pi$) of the trigger particle, whereas in Au+Au collisions, this peak is clearly missing. This is due to the energy loss of the away-side going particles in the medium.


\begin{figure}[!htb]
\centering
\includegraphics[width=\textwidth]{img/JetRaa}
\caption[Jet nuclear modification factor $\Raa^\mathrm{Pythia}$ measured in Au+Au collisions at \snnFull.]{\label{jetRaa}Jet nuclear modification factor $\Raa^\mathrm{Pythia}$ measured in Au+Au collisions at \snnFull, compared to theoretical calculations~\cite{JetRaaNLOinitialState, JetRaaSCET1,JetRaaSCET2, JetRaaHybrid,JetRaaLBT1, JetRaaLBT2}. Taken from~\cite{RusyPaper}\@.}

\end{figure}

Another way of measuring the properties of jets in heavy-ion collisions is to measure the nuclear modification factor \Raa\@. The \pt\ spectrum of jets that loose energy will be shifted towards lower \pt\ which will result in lower \Raa\@. A recently published paper from STAR~\cite{RusyPaper} shows the jet \Raa, measured in Au+Au collisions with PYTHIA~\cite{PYTHIA} used as a p+p reference. This result is shown in Figure~\ref{jetRaa}, correspond to relatively high modification of jets in the heavy-ion collisions. It is, however, consistent with models that include interaction of jets with QGP~\cite{JetRaaNLOinitialState, JetRaaSCET1,JetRaaSCET2, JetRaaHybrid,JetRaaLBT1, JetRaaLBT2}\@.
