The method of choice for studying of Quark-Gluon Plasma, that was discussed in length in the previous chapter, are collisions of relativistic nuclei. These are achieved in laboratory on particle accelerators. The work in this thesis was performed at Relativistic Heavy-Ion Collider (RHIC) that can collide several species of nuclei, as well as polarized protons, at various energies. 


Four experiments have operated at RHIC so far: Phobos, Brahms\footnote{Broad RAnge Hadron Magnetic Spectrometers}, PHENIX\footnote{Pioneering High Energy Nuclear Interaction eXperiment}, and finally, recording data at the time of writing this thesis: STAR. Phobos and Brahms were decommissioned after concluding their physics programs in 2005 and 2006, respectively, and PHENIX is currently undergoing a major overhaul upgrade to be transformed into a new experiment: the sPHENIX~\cite{sphenix}. This state-of-the-art experiment will be able to measure charged particles with high rate in the entire azimuth. It will feature a time-projection chamber with a continuous readout, as well electromagnetic and hadronic calorimeters which will make it a perfect tool for measuring jets and heavy flavor.

%-------------------------------------------------------------------------------------------------------------------

\section{Relativistic Heavy Ion Collider (RHIC)\label{RHIC}}

\begin{table}[!htb]
\caption{\label{runTableYears}Summary of RHIC runs between 2013 and 2019 with integrated luminosity at the STAR experiment~\cite{RHICrunsTable}. As of writing this thesis, the 2019 run is currently on-going, therefore the final luminosity data are not available.}
\begin{center}
\begin{tabular}{llD{.}{.}{3.1}c}
\toprule
 Year & Species & \multicolumn{1}{c}{\snn\ (GeV)} & Luminosity (nb$^{-1}$) \\
\midrule
  2013 & polarized p+p & 499.8 & $5.00\times10^5$ \\
  2014 & Au+Au & 14.6 & $2.12\times 10^{-2}$\\
       & Au+Au & 200.0 & 20.7\\
       & ${}^3$He+Au & 203.5 & $61.4$\\
  2015 & polarized p+p & 200.4 & $1.85\times10^5$ \\
       & polarized p+Au & 202.4 & $640$ \\
       & polarized p+Al & 202.5 & $1.54\times10^3$ \\
  2016 & Au+Au & 200.0 & 21.9\\
       & d+Au  & 200.7 & 134\\
       & d+Au  & 62.4 & 21.1\\
       & d+Au  & 19.7 & 3.44\\
       & d+Au  & 39.0 & 10.0\\
  2017 & polarized p+p& 499.8 & $5.44\times10^5$ \\
       & Au+Au & 54.4 & 0.477 \\
  2018 & ${}^{96}$Zr+${}^{96}$Zr & 200.0 & 3.91\\
       & ${}^{96}$Ru+${}^{96}$Ru & 200.0 & 4.00\\
       & Au+Au & 27.0 & 0.282 \\
       & Au+Au fixed target & 3.0 & 0.054 \\
       & Au+Au fixed target & 7.2 & 0.035 \\
  2019 & Au+Au & 14.6 & --- \\
       & Au+Au & 19.6 & --- \\
       & Au+Au fixed target & 3.9 & --- \\
       & Au+Au & 7.7 & --- \\
       & Au+Au fixed target & 4.5 & --- \\
       & Au+Au fixed target & 7.8 & --- \\
\bottomrule
\end{tabular}
\end{center}
\end{table} 

RHIC~\cite{RHICproject, RHICdesign} is a circular accelerator located in the Brookhaven National Laboratory 
(BNL)\nomenclature{BNL}{Brookhaven National Laboratory} in the USA\@. It is a versatile accelerator capable of
colliding several species of nuclei (Au+Au, d+Au, Au+$^3$He, Au+Cu, U+U,~\dots) at
center-of-mass energies per nucleon
ranging $\snn = (\text{7.7--200})\,\si{\giga\electronvolt}$ in a collider mode or down to
$\snn = 3.9\,\si{\giga\electronvolt}$
using a fixed target \cite{fixedTarget}\@. This versatility allows for studying various properties of the QGP since, by changing the collision system, we are able to vary the initial geometry and energy density, pressure, and electromagnetic-field profiles. Moreover, RHIC has a unique capability to collide polarized protons with
the center-of-mass energy up to $\sqrt{s} = \SI{500}{\giga\electronvolt}$~\cite{polarizedProtons}\@. The overview
of the RHIC runs between 2013 and 2019 is summarized in Table~\ref{runTableYears}\@. The energies per nucleon $E_1$ and $E_2$ were taken from~\cite{RHICrunsTable} and \snn\ was calculated as
\begin{equation}
 \snn = \sqrt{m_\mathrm{N1}^2c^4 + m_\mathrm{N2}^2c^4 + 2E_1E_2 + 2\sqrt{E_1^2-m_\mathrm{N1}^2c^4}\sqrt{E_2^2-m_\mathrm{N2}^2c^4}}
\end{equation}
where $m_\mathrm{N1}$ and $m_\mathrm{N2}$ are the masses of an average nucleon in the collided ion.


 

\subsection{RHIC Accelerator Complex}
Ions and protons are accelerated in several stages before colliding in one of the RHIC experiments at the desired energy. Ions are generated in the Electron Beam Ion Source (EBIS\nomenclature{EBIS}{Electron Beam Ion Source}~\cite{EBIS}) which provides all stable ion beam source ranging from deuterons to uranium. A part of the EBIS is the Linac which can reach energies up to\footnote{200 GeV can be reached in the case of protons} 200$\,$GeV\@. The EBIS can switch different ion species in the order of seconds which was important for the 2018 isobar running with zirconium and xenon.

\begin{figure}[htb]
\begin{center}
 \includegraphics[width=0.7\textwidth]{img/RHIC}\\
\end{center}
\caption{\label{RHICcomplex}Diagram of the RHIC accelerator complex~\cite{RHICpages}.}
\end{figure}

The ions then enter the circular Booster Synchrotron which accelerates them enough to enter the Alternating Gradient Synchrotron (AGS\nomenclature{AGS}{Alternating Gradient Synchrotron}~\cite{AGS}). This is an accelerator that fulfilled a rich physics program, including three Nobel-prize discoveries, before the RHIC facility was constructed. From AGS, the ions enter the AGS-to-RHIC (AtR\nomenclature{AtR}{AGS-to-RHIC transfer line}~\cite{AtR}) transfer line. Finally, the ions are injected into RHIC where they are accelerated to their final energies. 

Protons experience a similar journey to the ions. However, they are produced by ionizing hydrogen gas and then injecting them directly into the Linac, and then they continue to be accelerated in the same way as the ions. One of the unique capabilities of RHIC is the ability to produce polarized protons up to 500$\,$GeV, which is facilitated by the so-called Siberian-Snake magnets in the RHIC accelerator ring. In the end, they are collided in one of the six crossing points. Currently, the only experiment, measuring collisions is STAR, as PHENIX is undergoing an upgrade into a new experiment called sPHENIX\@.




% -----------------------------------------------------------------------------------------------------------

\section{Solenoidal Tracker At RHIC (STAR)}

STAR is a multipurpose detector that is capable of measuring charged particles and 
% When RHIC started its operation in the year 2000, it was a dream come true for many people studying the QGP and spin of the proton. 

\begin{figure}[htb]
\begin{center}
 \includegraphics[width=0.9\textwidth]{img/STAR}\\
\end{center}
\caption{\label{STAR}Overview of the STAR detector.}
\end{figure}

The Solenoidal Tracker At RHIC (STAR) experiment \cite{STARoverview} can be seen in Figure~\ref{STAR}. It is a multipurpose detector with full-azimuth
coverage dedicated to
studying ultrarelativistic heavy-ion collisions and polarized proton--proton collisions. Currently in 2016, STAR is the
only detector system running at RHIC\@.

The main barrel of STAR is enclosed in a water-cooled magnet with a magnetic field of \SI{0.5}{\tesla}. The magnet contains the
following detectors: Time -Projection Chambre (TPC\nomenclature{TPC}{Time-Projection Chamber} --- Section~\ref{TPCsection}), the GEM
Chambers to Monitor the TPC Tracking Calibrations (GMT\nomenclature{GMT}{GEM Chambers to Monitor the TPC Tracking Calibrations}), 
which is a set of 8 GEM detectors, used for the calibration of the TPC callibration, and Time-Of-Flight
(TOF\nomenclature{TOF}{Time-Of-Flight Detector}) Detector (Section~\ref{TOFsection}). In 2014, STAR received two major upgrades
concerning the heavy flavor measurements: The Heavy Flavor Tracker (HFT\nomenclature{HFT}{Heavy Flavor Tracker} --- section~\ref{HFTsection}), and the Muon Telescope Detector (MTD) which consists of fast Multi-wire Resistive Plate Chambers (MRPC) palced
outside the magnet. The main purpose of the MTD is to detect muons, which can easily traverse the volume of the magnet. The important
detectors for the open-charm hadrons reconstruction are listed here:

% -----------------------------------------------------------------------------------------------------------
\section{Time-Projection Chamber (TPC)\label{TPCsection}} 
It is not an exaggeration to say that the Time-Projection Chamber~\cite{TpcNim} is the beating heart of STAR. It combines the role of the main tracking detector with the ionizing-energy-loss
(\dedx) information for particle identification (PID\nomenclature{PID}{Particle IDentification}). The TPC is able to provide excellent tracking of particles down to low \pt\ with the large multiplicities at top RHIC energies.

\begin{figure}[htb]
\begin{center}
 \includegraphics[width=0.9\textwidth]{img/TPC}\\
\end{center}
\caption{\label{TPC}Overview of the STAR Time-Projection Chamber~\cite{TpcNim}.}
\end{figure}


 

When a charged particle traverses through the TPC, it ionizes the gas in the TPC volume. Then, the created negative 
charge (electrons) drifts to the anodes where it encounters \SI{20}{\micro\metre} wide wires which amplify the signal that is,
consequently, measured. The position in the transverse plane to the beam line (the xy-direction) is measured thanks to the
granularity of the TPC anodes and the position along the beam line (in the z-direction) is fixed by measuring the time taken by
he electrons to drift to the anodes. Hence, the spatial position of the track is fixed. The drift velocity of the electrons
in the P10 gas is calibrated every several hours via a laser calibration system~\cite{laser}. 


The layout of the TPC is shown in Figure~\ref{TPC}. It is barrel-shaped with the outer radius of \SI{2}{\metre}, inner radius of \SI{0.5}{\metre} and length of \SI{4.2}{\metre}
with the beam pipe going through the center. This barrel is filled with the P10 gas: a mixture of 90$\,\%$ argon and 10$\,\%$ methane. It covers the full azimuth ($0 < \phi < 2\pi$) and the outer edge covers the
pseudorapidity range of $|\eta| < 1$. 
In the center ($z = 0$), it is divided into two halves by the, so called, Central Membrane which is connected to an electric 
potential of $-\SI{28}{\kilo\volt}$\@.  The sides of the cylinder consist of the, so called, field cage that ensures that the electric field in the TPC stays uniform.

\begin{figure}[htb]
\begin{center}
 \includegraphics[width=0.6\textwidth]{img/TPC_side}\\
\end{center}
\caption{\label{TPCside}A side view of the TPC-outer-sector pad plane. The bubble shows additional information on the wire planes of the anodes~\cite{TpcNim}.}
\end{figure}

The bases of the TPC cylinder are grounded and equipped with multi-wire proportional chambers
(MWPC\nomenclature{MWPC}{Multi-Wire Proportional Chamber}) as sensitive detectors. Each side consists of 12 sectors, numbered like on a clock and split into 
the inner and outer parts. A side (the $xy$-plane) view of the sectors is shown in Figure~\ref{TPCside}\@. Each MWPC consists of 3 planes of wires mounted over a plane of sensitive pads. Going from inside of the TPC volume out, the innermost plane of wires makes the, so called, gating grid. This part acts as a shutter that does not allow any electrons to enter the MWPC and it stops the ions from escaping the MWPC into the drift area where they would distort the electric field. It also lowers the amount of noise in the MWPC as ions are eaten out by the wires. The second row of wires is called the Shield Grid and it terminates the amplification region created by the outermost plane of wires: the Anode Grid. Here, the drifting electrons are multiplied in avalanches around the wires where the voltage differentials are high. These wires are set on potential of $\sim$1350$\,$V which sets the signal multiplication to $\sim$20$\times$\@: A value selected as a sweet spot between spatial resolution and the signal amplification. The signal is subsequently read out from the rectangular pads behind the anode wires. The pads are placed so that they always have an anode wire in the middle and their pitch is 6.7$\,$mm along the wires and 20$\,$mm --- same as the distance between the wires --- across. 

Until the run 2018, the inner part utilized 13 rows and outer part with 32 rows of pads, the inner sectors, however, underwent a substantial 
upgrade in the year 2019~\cite{iTPC}. This upgrade increased the number of sensitive pad rows in the inner sector to the same number as in the outer sectors. This greatly
increased the efficiency, resolution, and pseudorapidity coverage of the TPC which, in turn, enabled the fixed-event measurements at STAR with the endcap-Time-Of-Flight (eTOF) detector in 2019 and, in the future, it will be able to facilitate the measurement of electron+ion and electron+polarized-proton collisions after the upgrade to eRHIC\@. 

\begin{figure}[!htb]
\begin{center}
 \includegraphics[width=0.9\textwidth]{img/TPC_dedx}\\
\end{center}
\caption{\label{TpcPid}PID capabilities of the TPC via \dedx\ separation in the 2014 Au+Au run at $\snn = 200\,$GeV\@.}
\end{figure}

The TPC provides excellent momentum-measuring and PID capabilities in a high particle-multiplicity environment of the heavy-ion collisions.
A plot of particle-species separation can be found in Figure~\ref{TpcPid}.
% -----------------------------------------------------------------------------------------------------------
\section{Time-Of-Flight Detector (TOF)\label{TOFsection}} 



The main purpose of the Time-Of-Flight (TOF\nomenclature{TOF}{Time-Of-Flight detector}) detector~\cite{TOFproposal} is to extend the PID capabilities of STAR into higher 
momenta via the measurements of the time of flight from its point of origin of the charged particles (or vertex) to the TOF 
detector, thus measuring its velocity. It was partially installed at STAR in 2009 and the remainder was added in 2010 
to make the TOF fully operational.

\begin{figure}[!htb]
\begin{center}
 \includegraphics[width=\textwidth]{img/TOF_MRPC}\\
\end{center}
\caption{\label{TOF_mrpc}Layout of the TOF MRPC~\cite{TOFproposal}\@. The upper and lower images show views of the longer and shorted edge, respectively. The two views do not show the same scale.}
\end{figure}

The TOF consists of 120 trays of Multi-gap Resistive Plate Chambers (MRPC)\nomenclature{MRPC}{Multi-gap Resistive Plate
Chamber}~\cite{MRPC} that cover $|\eta| \lesssim 0.96$ and $0 < \phi < 2\pi$. The time of the original collision is measured 
by the Vertex Position Detector (VPD\nomenclature{VPD}{Vertex Position Detector} --- see section~\ref{VPD}). The leading edge of 
the signal is sampled with \SI{25}{\pico\second} binning. Corrections can be made for the decay products of particles that 
decayed outside of the primary vertex, such as K$^0$ or $\Lambda$\@. 



Figure~\ref{TOF_mrpc} shows two side views of the TOF tray. A MRPC basically consists of a stack of resistive plates with uniform gas gaps in between. A high voltage is applied to the graphite electrodes that have electrically floating glass plates in between Typical resistivity of the glass plates is in the order of $10^13\,\Omega/$cm. A charged particle going through the tray generates electrical avalanches in the gas gaps. The signal is read out via copper pads on the outside of the graphite electrodes, separated by $0.35\,$mm of mylar. Because of the high resistivity of the glass plates, they are transparent to the charge induction in the gaps. The signal is the sum of all avalanches in the gas gaps in between the opposing copper pads. The gas consists of 95$\,\%$ Freon R134a and 5$\,\%$ iso-butane. This gas mixture was chosen for its dielectric strength and electronegativity. The trays are mechanically supported poly-carbonate (PC\nomenclature{PC}{Poly-Carbonate}) boards on each side, glued to honeycomb structures that are rigid while having a low material budget.

\begin{figure}[!htb]
\begin{center}
 \includegraphics[width=0.9\textwidth]{img/TOF_PID}\\
\end{center}
\caption{\label{TOF_pid}PID capabilities of the TOF via measurement of the inverse velocity $1/\beta$ versus \pt in the 2014 Au+Au run at $\snn = 200\,$GeV\@. }
\end{figure}

The PID capabilities of TOF are demonstrated in Figure~\ref{TOF_pid}\@. The fraction of the speed of light in vacuum $\beta$ is calculated as
\begin{equation}
 \beta = \frac{L}{ct}
\end{equation}
where $L$ is the path-length from the primary vertex (PV\nomenclature{PV}{Primary Vertex}) and $t$ is the flight time, measured by TOF\@.
A clear separation of the particle species is visible, moreover it reaches further in \pt, compared to \dedx\ PID of the TPC e.g.\ shown in Figure~\ref{TpcPid}\@.

\begin{figure}[!htb]
\begin{center}
 \includegraphics[width=0.8\textwidth]{img/eTOF}\\
\end{center}
\caption{\label{eTOF}Illustration of the endcap-Time-Of-Flight detector layout~\cite{eTOF}. }
\end{figure}

% \subsection{Endcap-Time-of-Flight Detector (eTOF)}
In 2019, the time-of-flight coverage of STAR was greatly expanded by a new upgrade to the STAR detector: the endcap-Time-Of-Flight detector (eTOF\nomenclature{eTOF}{endcap-Time-Of-Flight detector}~\cite{eTOF})\@. This detector was developed in collaboration with the Compressed Baryonic Matter (CBM\nomenclature{CBM}{Compressed Baryonic Matter}~\cite{CBM}) as eTOF doubles as a test of the CBM Time-Of-Flight detector. eTOF consists of trays of similar design to the TOF mounted to the endcap of the TPC\@. The layout is shown in Figure\ref{eTOF}\@. eTOF enhances the pseudorapidity reach of PID via time-of-flight measurements down to at least $\eta > 1.5$~\cite{eTOF_LOI}\@.


% -----------------------------------------------------------------------------------------------------------
\section{Heavy Flavor Tracker (HFT)\label{HFTsection}} 

The Heavy Flavor Tracker (HFT) was installed at STAR for the years 2014--2016 data 
taking \cite{HftTdr, HFTLeo, HftFinal}. It resides between the beam pipe and the TPC which makes it the innermost 
sub-detector of STAR\@. The 
HFT
subdetectors are summarized in Table~\ref{HFTtab}\@. The HFT consists of 4
layers of silicon detectors --- from the outermost layer in: A double-sided strip detector, called Silicon Strip
Tracker/Detector
(SST/SSD\nomenclature{SST/SSD}{Silicon Strip Tracker/Detector}) which waa previously installed at STAR and has been
refurbished and upgraded with new electronics; Next, the second outermost layer is formed by conventional silicon 
pad detectors
of the Intermediate Silicon Tracker (IST\nomenclature{IST}{Intermediate Silicon Tracker}) 
with rectangular pads; Last, but not least, the two innermost layers consist of the Pixel (PXL\nomenclature{PXL}{Pixel Detector}) 
Detector
that employs the novel MAPS technology that was used for the first time in a collider experiment. Figure~\ref{HFT_whole} shows the layout of the HFT\@.


\begin{table}[!htb]
\caption[HFT subdetectors.]{\label{HFTtab}HFT subdetectors, their average radii from the center of the beam pipe, and pitches of the sensitive pads. For the SSD, the hit resolutions $\sigma_{r\phi}$ and $\sigma_z$ are listed, instead of the pitch, because it is a double sided strip detector where the strips are not perpendicular~\cite{HftTdr}.}
\begin{center}
\begin{tabular}{lcccc}
\toprule
System & Type & Radius [cm] & Pitch--$r\phi$ [$\upmu$m] & Pitch--$z$ [$\upmu$m]\\
\midrule
SSD & Double-sided strip detector & 22 & $\sigma$ = 20 & $\sigma$ = 740 \\
IST & Silicon pad detector & 14 & 600 & 6000 \\
PXL & Silicon pixel detector & 2.7, 8 & 20.7 & 20.7 \\
\bottomrule
\end{tabular}
\end{center}
\end{table}

\begin{figure}[!htb]
\begin{center}
 \includegraphics[width=0.7\textwidth]{img/HFT_whole}\\
\end{center}
\caption[Render of the HFT installed inside STAR.]{\label{HFT_whole}Render of the HFT installed inside STAR~\cite{HftTdr}\@.}
\end{figure}




The purpose of the IST and the SSD is to guide the track from the TPC to the innermost layers of HFT--PXL in the high-track-multiplicity environment of STAR. Figure~\ref{HFT_layers} illustrates the improvement of the tracking resolution with each layer of the HFT\@. The PXL especially provides an unparalleled pointing precision thanks to its high granularity as well as the proximity to the primary vertex. For the runs with the HFT, the beam pipe had to be replaced with a narrower one, in order to accommodate for the low radius of the innermost layer of PXL\@. The resolution of the distance of closest approach (DCA resolution) of identified particles can be seen in Figure~\ref{DCA}. For high-momentum tracks, the resolution is as low as \SI{20}{\micro\metre}.

\begin{figure}[!htb]
\begin{center}
 \includegraphics[width=0.8\textwidth]{img/HFT_layers}\\
\end{center}
\caption[Average resolution improvement with each layer of the HFT.]{\label{HFT_layers}Average resolution improvement with each layer of the HFT, going from the TPC down to the innermost layer of the PXL detector. Taken from~\cite{KubaVyzkumak}.}
\end{figure}

\begin{figure}[!htb]
\begin{center}
 \includegraphics[width=0.6\textwidth]{img/DCAXy}\\
\end{center}
\caption[DCA from the PV in the $xy$-plane for identified particles.]{\label{DCA}Distance of Closest Approach (DCA\nomenclature{DCA}{Distance of Closest Approach}) from the PV in the $xy$-plane for identified particles. Taken from~\cite{D0v2paper}.}
\end{figure}


Great care was used when engineering the mechanical support so that it is lightweight while providing enough support that allows for cooling of the system. The entire HFT is mounted on a novel structure on rails so that it can be removed and replaced within 24 hours. This structure also provides air flow for cooling. For PLX and SSD  air cooling is sufficient, but the IST is also liquid cooled.

\subsection{Silicon-Strip Detector (SSD)}

The Silicon-Strip Detector (SSD) is a refurbished detector with upgraded electronics that can facilitate the high collision rates in runs 2014 and 2016\@. The SSD consists of 20 ladders 67$\,$cm in length with double-sided silicon strip wafers, mounted 22$\,$cm from the center of the beam pipe. The strips are 95$\,\upmu$m wide and 4.2$\,$cm long and are crossed at an angle of 35 mrad, oriented to prioritize the resolution in the $r\phi$ direction. The material of the SSD amounts to $\sim$1$\,\%$ of the radiation length.

\subsection{Intermediate Silicon Tracker (IST)\label{IstSection}}

The Intermediate Silicon Tracker (IST) is a single-sided silicon-pad detector with rectangular pads of $600\,\upmu\mathrm{m}\times6\,$mm. It is located between the PXL detector and the SSD at a radius of 14$\,$cm. The IST consists of 24 ladders with 6 sensors each, supported on kapton mechanical structure with cooling tubes. The IST is cooled by the Novec 7200 liquid which is a dielectric, it evaporates quickly in the event of a spill, and it is not harmful to the environment. The material budget of the IST is calculated as $\sim$1.5$\,\%$ of the radiation length.

\subsection{Pixel (PXL) detector}
The two innermost layers of the HFT consist of the Pixel (PXL) detector which employs a unique design as it is the
first detector using the Monolithic Active-Pixel Sensor (MAPS)
technology. This enables the pixel sensors to have a miniscule pixel pitch of \SI{20.7}{\micro\metre} while maintaining
a small material budget.

\begin{figure}[!htb]
\begin{center}
 \includegraphics[width=.7\textwidth]{img/MAPS_illustration.jpg}\\
\end{center}
\caption[Illustration of the MAPS technology principle.]{\label{MAPS}Illustration of the MAPS technology principle~\cite{MAPS_illustration}.}
\end{figure}

Compared to conventional bump-bonded pixel sensors, the advantage of the MAPS technology is that it incorporates the front-end electronics (FEE\nomenclature{FEE}{Front-End Electronics}), such as amplifiers and discriminators, inside the pixels themselves. The MAPS uses one layer of silicon instead of two layers in traditional pixels which reduces the material budget and cost of the detector. Moreover, the pixels can be made smaller as they benefit from the advancement of the CMOS\nomenclature{CMOS}{Complementary Metal-Oxide-Semiconductor}-printing technology in the industry. An illustration of a cut through a MAPS wafer is shown in Figure~\ref{MAPS}\@. The FEEs can be incorporated on the same silicon wafer, because they are shielded inside a deep P-well while the signal electrons are read out via an N-well. When a charged particle traverses the PXL detector, a cloud of electron-hole pairs is created predominantly in the P+-type epitaxial layer. Then some of the electrons drift into the depleted zone around the N-well collection electrode via diffusion, and are subsequently picked up as signal. A single-particle hit is typically read out by multiple N-wells and it creates a cluster of several wounded pixels. This improves the resolution of the pixel sensor as the position of multiple pixels can be averaged as a probable position of the particle impact.

\begin{figure}[!htb]
\begin{center}
 \includegraphics[width=\textwidth]{img/PXL_mounted}\\
\end{center}
\caption[Layout of the PXL detector.]{\label{PXL}Layout of the PXL detector mounted on the supporting structure~\cite{HftTdr}.}
\end{figure}

The layout of the PXL detector, mounted on the HFT-support structure, is illustrated in Figure~\ref{PXL}\@. The sensors themselves are on the right-hand side of the image, marked by a blue color. The PXL consists of two concentric barrels, each 20$\,$cm long, first one at a radius of 2.7$\,$cm and the second one at 8$\,$cm.  The sensors are organized into ladders, 10 in the inner barrel and 30 in the outer one. The support structure for the PXL consists of 10 roughly trapeziodal kapton sections, each supporting three ladders from the outer layer and one ladder from the inner one. The innermost layer is placed right on the beam pipe which had to be made anew to accommodate for the small radius of the PXL\@. The sensors are milled so that the material budget of each layer was lowered to only 0.5$\,\%$ of the radiation length.

The combination of the small pixel pitch, the small radius of the first layer, and the low material budget of the PXL make for a before-unforseen pointing resolution of the HFT\@. This is crucial for the precise measurements of open-heavy-flavor hadrons that decay relatively close to the PV such as the \Lambdac\ with $c\tau\approx 60 \,\upmu$m. 

\subsubsection{HFT Slow Simulator Evaluation and comparison to measured data}
The process of charge deposition in MAPS is illustrated in Figure~\ref{DigmapsIllustration}\@.
When an ionizing particle passes through a MAPS wafer, it generates a cloud of electrons and holes that can drift to several pixels where they are collected inside the CMOS N-wells and, subsequently, measured as signal.  

\begin{figure}[!htb]
\begin{center}
 \includegraphics[width=0.65\textwidth]{img/DIGMAPS_particle}\\
\end{center}
\caption[An illustration of particle energy deposition and charge transport in a MAPS chip.]{\label{DigmapsIllustration}An illustration of particle energy deposition and charge transport in a MAPS chip~\cite{DIGMAPS}.}
\end{figure}

Every analysis of measured data depends on reliable Monte-Carlo simulation for corrections of the efficiency of the detectors. At STAR, GEANT3~\cite{GEANT} is typically used for simulating the detector response to an ionizing particle. However, GEANT does not take into account processes in thin silicon wafers, so the detector response has to be simulated outside of the GEANT environment and then embedded into the GEANT simulation. 

Two types of detector simulators are employed for PXL, the so called, fast simulator and slow simulator. The fast simulator takes a position where a track crossed the detector, smears it according to the detector resolution, and calculates the efficiency of the detector, according to the particle species and its \pt\@. The slow simulator, on the other hand, simulates deposited energy of the particle inside the detector and generates detector response to the pixel layer. The slow simulator can create a more realistic picture of the detector response, including noise in the detector. This is crucial when using the clustering and tracking algorithms on the simulated detectors. Slow simulator is mainly used in embedding of the simulated tracks in measured data as it requires information on the level single pixels. The fast simulation is adequate when simulated data are not mixed with the measured ones, because, typically, only the position of the hits is required.

A tool for simulation of the MAPS-detector response --- MAPS digitizer or ``DIGMAPS\nomenclature{DIGMAPS}{MAPS Digitizer}'' --- has been developed at Strasbourg University~\cite{DIGMAPS}\@. This model simulates the relatively complicated process from a passing particle to the output of the MAPS sensor in the following steps: energy deposition, charge transport of the electron--hole pairs, digitization of the ADC with added noise.
\begin{enumerate}
 \item The \textbf{energy deposition} is calculated via a Landau distribution with the PDF~\cite{Landau}
 \begin{equation}
  P(x,\mathrm{MPV},\mathrm{width}) = \frac{1}{\pi}\int^{\infty}_{0}\, \eee^{-s\cdot y - s\ln s} \cos(\pi s)\ddd s \,,\qquad y=\frac{x-\mathrm{MPV}}{\mathrm{width}}
 \end{equation}
 where MPV = 80$\,\eee^-/\upmu$m and width = 18$\,\eee^-/\upmu$m.
The effective thickness of the sensor is calculated as $L_\mathrm{epi}/\cos \theta$ where $L_\mathrm{epi}$ is the width of the epitaxial layer and $\theta$ is the angle between the passing-particle trajectory and the normal to the sensor plane.
 \item The \textbf{charge transport} is simulated as a sum of a gaussian and a Lorentzian with the center in the middle of the track's intersection with the MAPS wafer. This ditribution has described the test beam data the best. The charge is then deposited on 25 N-wells around the center of the distribution.
 \item In the \textbf{digitization} step, noise is introduced into each pixel with a Gaussian distribution, and then the pixels with the number of electrons higher than a set ADC threshold are counted as wounded.
\end{enumerate}

\begin{figure}[!htb]
\begin{center}
 \includegraphics[width=0.7\textwidth]{img/DIGMAPS_cosmic}\\
\end{center}
\caption[Pixel slow-simulator-cluster size compared to cosmic data at the angle of 0--10°.]{\label{cosmicDigmaps}Pixel slow-simulator-cluster size compared to cosmic data at the angle of 0--10°.}
\end{figure}

The DIGMAPS package can be run in the simulation mode that returns a map of wounded pixels or it can be run in a stand-alone mode, without any GEANT, in which it clusterizes the pixels into hits.


DIGMAPS has several free parameters that have to be tuned through data. In this test, we use parameters tuned on minimum-ionizing particles (MIP\nomenclature{MIP}{Minimum-Ionizing Particle}) from beam tests with the exception of the ADC threshold which can be set for each pixel row individually, and is tuned several times during the run. The value 6.2$\,$mV has proven to reproduce the data the best.
 
The DIGMAPS model had to be verified with measured data from run 2014 to ensure that the PXL-slow simulator is accurate. Cluster sizes were compared between the measured data and the Digmaps simulation. At first, cosmic data with zero-magnetic field were used, because data from collisions were not available yet. This data also proved useful when comparing them to the beam tests, because MIP\ are used in both cases.


The cosmic tracks are selected as follows: Only tracks, that traverse through the center barrel of the HFT and have 4 hits in the PXL and 2 hits in the IST, are selected. At least one of the hits has to have a cluster size of two or more. These precautions are applied to eliminate fake tracks that merely connect noise hits. The hits are chosen within a square window $6\,\mathrm{mm}\times6\,\mathrm{mm}$ around the center of the track. Only 1 hit may be present inside the square.

\begin{figure}[!tb]
\begin{center}
 \includegraphics[width=\textwidth]{img/DIGMAPS}\\
\end{center}
\caption[Comparison of cluster size of the Slow simulator and global identified tracks and primary track only.]{\label{AuAuDigmaps}Comparison of cluster size of the Slow simulator and global identified tracks (left) and primary (tracks that originate in the primary vertex) track only
(right)~\cite{KubaVyzkumak}.}
\end{figure}


Measured-cluster data were compared to the DIGMAPS simulations for angles up to $\theta < 60$°. Figure~\ref{cosmicDigmaps} shows a comparison of PXL cluster sizes between the cosmic data at impact angles $0\text{°} < \theta < 10$° and DIGMAPS generated at 5°\@. Both, the distributions from simulation and data, are normalized to unity for easier comparison.  ADC threshold of 6.2$\,$mV was used for the DIGMAPS simulation. The number of hits rises with more pixels until the cluster size of 4, then it drops steeply. The width and the overall shape of the cluster-size distribution is reproduced well in DIGMAPS, however the dip at 3-pixel clusters does not show in the simulation which is, however, not considered a significant problem as it likely has little effect on the pointing resolution.



A similar evaluation was later performed with Au+Au collisions at the center-of-mass energy per nucleon $\snn = \SI{200}{\giga\electronvolt}$ from run 2014~\cite{KubaVyzkumak}\@. This time, the cluster sizes generated by DIGMAPS are compared to the clusters in tracks from identified particles and all charged tracks (see Figure~\ref{AuAuDigmaps})\@.  In general, Protons have slightly higher cluster sizes, compared to pions and kaons, however the overall shape stays the same and is reproduced by DIGMAPS.

Overall, DIGMAPS have proven to be a useful tool that is incorporated in the PXL slow simulator. The cluster-size distributions copy the data well enough so that the resolution of the HFT and the efficiency are reproduced with sufficient accuracy.



% -----------------------------------------------------------------------------------------------------------
\section{Zero Degree Calorimeter (ZDC)\label{ZDCsection}} \nomenclature{ZDC}{Zero Degree Calorimeter}





%-------------------------------------------------------------------------------------------------------------------

\section{Vertex-Position Detector (VPD)\label{VPD}}

\begin{figure}[!htb]
\begin{center}
 \includegraphics[width=0.7\textwidth]{img/VPD}\\
\end{center}
\caption{\label{VPDassembly}VPD assembly~\cite{VPD}.}
\end{figure}

The VPD~\cite{VPD} consists of two identical assemblies, positioned \SI{5.7}{\metre} from the center of STAR\@. A schematic 
picture of an assembly can be seen in Figure~\ref{VPDassembly}. Each of them is made of 19 detectors, consisting of a lead 
converter, followed by a fast plastic scintillator read by a photo-multiplier tube. The main purpose of the VPD is to measure 
photons from $\uppi^0$ decays in forward direction and thus accurately measure the time of the collision, as well as 
the position of the primary vertex with the resolution of $\sim \SI{1}{\centi\metre}$ and  $\sim \SI{2.5}{\centi\metre}$ 
for Au+Au and p+p collisions, respectively.

% \section{Data Acquisition}
% \subsection{Trigger}

