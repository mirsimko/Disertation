The method of choice for studying of Quark-Gluon Plasma, that was discussed in length in the previous chapter, are collisions of relativistic nuclei. These are achieved in laboratory on particle accelerators. The work in this thesis was performed at Relativistic Heavy-Ion Collider (RHIC) that can collide several species of nuclei, as well as polarized protons, at various energies. 


Four experiments have operated at RHIC so far: Phobos, Brahms\footnote{Broad RAnge Hadron Magnetic Spectrometers}, PHENIX\footnote{Pioneering High Energy Nuclear Interaction eXperiment}, and finally, recording data at the time of writing this thesis: STAR. Phobos and Brahms were decommissioned after concluding their physics programs in 2005 and 2006, respectively, and PHENIX is currently undergoing a major overhaul upgrade to be transformed into a new experiment: the sPHENIX~\cite{sphenix}. This future experiment will be able to measure charged particles with high rate in the entire azimuth and it will feature electromagnetic and hadronic calorimeters which will make it a perfect tool for measuring jets and heavy flavor.

%-------------------------------------------------------------------------------------------------------------------

\section{Relativistic Heavy Ion Collider (RHIC)\label{RHIC}}

RHIC~\cite{RHICproject, RHICdesign} is a circular accelerator located in the Brookhaven National Laboratory 
(BNL)\nomenclature{BNL}{Brookhaven National Laboratory} in the USA\@. It is a versatile accelerator capable of
colliding several species of nuclei (Au+Au, d+Au, Au+$^3$He, Au+Cu, U+U,~\dots) at
center-of-mass energies per nucleon
ranging $\snn = (\text{7.7--200})\,\si{\giga\electronvolt}$ in a collider mode or down to
$\snn = 3.9\,\si{\giga\electronvolt}$
using a fixed target \cite{fixedTarget}\@. This versatility allows for studying various properties of the QGP since, by changing the collision system, we are able to vary the initial geometry and energy density, pressure, and electromagnetic-field profiles. Moreover, RHIC has a unique capability to collide polarized protons with
the center-of-mass energy up to $\sqrt{s} = \SI{500}{\giga\electronvolt}$~\cite{polarizedProtons}\@. The overview
of the RHIC runs between 2013 and 2019 is summarized in Table~\ref{runTableYears}\@. The energies per nucleon $E_1$ and $E_2$ were taken from~\cite{RHICrunsTable} and \snn\ was calculated as
\begin{equation}
 \snn = \sqrt{m_\mathrm{N1}^2c^4 + m_\mathrm{N2}^2c^4 + 2E_1E_2 + 2\sqrt{E_1^2-m_\mathrm{N1}^2c^4}\sqrt{E_2^2-m_\mathrm{N2}^2c^4}}
\end{equation}
where $m_\mathrm{N1}$ and $m_\mathrm{N2}$ are the masses of an average nucleon in the collided ion.


\begin{table}[!htb]
\caption{\label{runTableYears}Summary of RHIC runs between 2013 and 2019 with integrated luminosity at the STAR experiment~\cite{RHICrunsTable}. As of writing this thesis, the 2019 run is currently on-going, therefore the final luminosity data are not available.}
\begin{center}
\begin{tabular}{llD{.}{.}{3.1}c}
\toprule
 Year & Species & \multicolumn{1}{c}{\snn\ (GeV)} & Luminosity (nb$^{-1}$) \\
\midrule
  2013 & polarized p+p & 499.8 & $5.00\times10^5$ \\
  2014 & Au+Au & 14.6 & $2.12\times 10^{-2}$\\
       & Au+Au & 200.0 & 20.7\\
       & ${}^3$He+Au & 203.5 & $61.4$\\
  2015 & polarized p+p & 200.4 & $1.85\times10^5$ \\
       & polarized p+Au & 202.4 & $640$ \\
       & polarized p+Al & 202.5 & $1.54\times10^3$ \\
  2016 & Au+Au & 200.0 & 21.9\\
       & d+Au  & 200.7 & 134\\
       & d+Au  & 62.4 & 21.1\\
       & d+Au  & 19.7 & 3.44\\
       & d+Au  & 39.0 & 10.0\\
  2017 & polarized p+p& 499.8 & $5.44\times10^5$ \\
       & Au+Au & 54.4 & 0.477 \\
  2018 & ${}^{96}$Zr+${}^{96}$Zr & 200.0 & 3.91\\
       & ${}^{96}$Ru+${}^{96}$Ru & 200.0 & 4.00\\
       & Au+Au & 27.0 & 0.282 \\
       & Au+Au fixed target & 3.0 & 0.054 \\
       & Au+Au fixed target & 7.2 & 0.035 \\
  2019 & Au+Au & 14.6 & --- \\
       & Au+Au & 19.6 & --- \\
       & Au+Au fixed target & 3.9 & --- \\
       & Au+Au & 7.7 & --- \\
       & Au+Au fixed target & 4.5 & --- \\
       & Au+Au fixed target & 7.8 & --- \\
\bottomrule
\end{tabular}
\end{center}
\end{table}  

\subsection{RHIC Accelerator Complex}
Ions and protons are accelerated in several stages before colliding in one of the RHIC experiments at the desired energy. Ions are generated in the Electron Beam Ion Source (EBIS\nomenclature{EBIS}{Electron Beam Ion Source}~\cite{EBIS}) which provides all stable ion beam source ranging from deuterons to uranium. A part of the EBIS is the Linac which can reach energies up to\footnote{200 GeV can be reached in the case of protons} 200$\,$GeV\@. The EBIS can switch different ion species in the order of seconds which was important for the 2018 isobar running with zirconium and xenon.

\begin{figure}[htb]
\begin{center}
 \includegraphics[width=0.7\textwidth]{img/RHIC}\\
\end{center}
\caption{\label{RHICcomplex}Diagram of the RHIC accelerator complex~\cite{RHICpages}.}
\end{figure}

The ions then enter the circular Booster Synchrotron which accelerates them enough to enter the Alternating Gradient Synchrotron (AGS\nomenclature{AGS}{Alternating Gradient Synchrotron}~\cite{AGS}). This is an accelerator that fulfilled a rich physics program, including three Nobel-prize discoveries, before the RHIC facility was constructed. From AGS, the ions enter the AGS-to-RHIC (AtR\nomenclature{AtR}{AGS-to-RHIC transfer line}~\cite{AtR}) transfer line. Finally, the ions are injected into RHIC where they are accelerated to their final energies. 

Protons experience a similar journey to the ions. However, they are produced by ionizing hydrogen gas and then injecting them directly into the Linac, and then they continue to be accelerated in the same way as the ions. One of the unique capabilities of RHIC is the ability to produce polarized protons up to 500$\,$GeV, which is facilitated by the so-called Siberian-Snake magnets in the RHIC accelerator ring. In the end, they are collided in one of the six crossing points. Currently, the only experiment, measuring collisions is STAR, as PHENIX is undergoing an upgrade into a new experiment called sPHENIX\@.




% -----------------------------------------------------------------------------------------------------------

\section{Solenoidal Tracker At RHIC (STAR)}

STAR is a multipurpose detector that is capable of measuring charged particles and 
% When RHIC started its operation in the year 2000, it was a dream come true for many people studying the QGP and spin of the proton. 

\begin{figure}[htb]
\begin{center}
 \includegraphics[width=0.9\textwidth]{img/STAR}\\
\end{center}
\caption{\label{STAR}Overview of the STAR detector.}
\end{figure}

The Solenoidal Tracker At RHIC (STAR) experiment \cite{STARoverview} can be seen in Figure~\ref{STAR}. It is a multipurpose detector with full-azimuth
coverage dedicated to
studying ultrarelativistic heavy-ion collisions and polarized proton--proton collisions. Currently in 2016, STAR is the
only detector system running at RHIC\@.

The main barrel of STAR is enclosed in a water-cooled magnet with a magnetic field of \SI{0.5}{\tesla}. The magnet contains the
following detectors: Time -Projection Chambre (TPC\nomenclature{TPC}{Time-Projection Chamber} --- Section~\ref{TPCsection}), the GEM
Chambers to Monitor the TPC Tracking Calibrations (GMT\nomenclature{GMT}{GEM Chambers to Monitor the TPC Tracking Calibrations}), 
which is a set of 8 GEM detectors, used for the calibration of the TPC callibration, and Time-Of-Flight
(TOF\nomenclature{TOF}{Time-Of-Flight Detector}) Detector (Section~\ref{TOFsection}). In 2014, STAR received two major upgrades
concerning the heavy flavor measurements: The Heavy Flavor Tracker (HFT\nomenclature{HFT}{Heavy Flavor Tracker} --- section~\ref{HFTsection}), and the Muon Telescope Detector (MTD) which consists of fast Multi-wire Resistive Plate Chambers (MRPC) palced
outside the magnet. The main purpose of the MTD is to detect muons, which can easily traverse the volume of the magnet. The important
detectors for the open-charm hadrons reconstruction are listed here:

% -----------------------------------------------------------------------------------------------------------
\section{Time-Projection Chamber (TPC)\label{TPCsection}} 
It is not an exaggeration to say that the Time-Projection Chamber~\cite{TpcNim} is the beating heart of STAR. It combines the role of the main tracking detector with the ionizing-energy-loss
(\dedx) information for particle identification (PID\nomenclature{PID}{Particle IDentification}). The TPC is able to provide excellent tracking of particles down to low \pt\ with the large multiplicities at top RHIC energies.

\begin{figure}[htb]
\begin{center}
 \includegraphics[width=0.9\textwidth]{img/TPC}\\
\end{center}
\caption{\label{TPC}Overview of the STAR Time-Projection Chamber~\cite{TpcNim}.}
\end{figure}


 

When a charged particle traverses through the TPC, it ionizes the gas in the TPC volume. Then, the created negative 
charge (electrons) drifts to the anodes where it encounters \SI{20}{\micro\metre} wide wires which amplify the signal that is,
consequently, measured. The position in the transverse plane to the beam line (the xy-direction) is measured thanks to the
granularity of the TPC anodes and the position along the beam line (in the z-direction) is fixed by measuring the time taken by
he electrons to drift to the anodes. Hence, the spatial position of the track is fixed. The drift velocity of the electrons
in the P10 gas is calibrated every several hours via a laser calibration system~\cite{laser}. 


The layout of the TPC is shown in Figure~\ref{TPC}. It is barrel-shaped with the outer radius of \SI{2}{\metre}, inner radius of \SI{0.5}{\metre} and length of \SI{4.2}{\metre}
with the beam pipe going through the center. This barrel is filled with the P10 gas: a mixture of 90$\,\%$ argon and 10$\,\%$ methane. It covers the full azimuth ($0 < \phi < 2\pi$) and the outer edge covers the
pseudorapidity range of $|\eta| < 1$. 
In the center ($z = 0$), it is divided into two halves by the, so called, Central Membrane which is connected to an electric 
potential of $-\SI{28}{\kilo\volt}$\@.  The sides of the cylinder consist of the, so called, field cage that ensures that the electric field in the TPC stays uniform.

\begin{figure}[htb]
\begin{center}
 \includegraphics[width=0.6\textwidth]{img/TPC_side}\\
\end{center}
\caption{\label{TPCside}A side view of the TPC-outer-sector pad plane. The bubble shows additional information on the wire planes of the anodes~\cite{TpcNim}.}
\end{figure}

The bases of the TPC cylinder are grounded and equipped with multi-wire proportional chambers
(MWPC\nomenclature{MWPC}{Multi-Wire Proportional Chamber}) as sensitive detectors. Each side consists of 12 sectors, numbered like on a clock and split into 
the inner and outer parts. A side (the $xy$-plane) view of the sectors is shown in Figure~\ref{TPCside}\@. Each MWPC consists of 3 planes of wires mounted over a plane of sensitive pads. Going from inside of the TPC volume out, the innermost plane of wires makes the, so called, gating grid. This part acts as a shutter that does not allow any electrons to enter the MWPC and it stops the ions from escaping the MWPC into the drift area where they would distort the electric field. It also lowers the amount of noise in the MWPC as ions are eaten out by the wires. The second row of wires is called the Shield Grid and it terminates the amplification region created by the outermost plane of wires: the Anode Grid. Here, the drifting electrons are multiplied in avalanches around the wires where the voltage differentials are high. These wires are set on potential of $\sim$1350$\,$V which sets the signal multiplication to $\sim$20$\times$\@: A value selected as a sweet spot between spatial resolution and the signal amplification. The signal is subsequently read out from the rectangular pads behind the anode wires. The pads are placed so that they always have an anode wire in the middle and their pitch is 6.7$\,$mm along the wires and 20$\,$mm --- same as the distance between the wires --- across. The inner part utilizes 13 rows and outer part with 32 rows of pads, the inner sectors, however, underwent a substantial 
upgrade in the year 2019~\cite{iTPC}, which increased the number of sensitive pad rows in the inner sector to the same number as in the outer sectors. This greatly
increased the efficiency, resolution, and pseudorapidity coverage of the TPC\@.

\begin{figure}[!htb]
\begin{center}
 \includegraphics[width=0.6\textwidth]{img/TPC_dedx}\\
\end{center}
\caption{\label{TpcPid}PID capabilities of the TPC via \dedx\ separation~\cite{TpcNim}.}
\end{figure}

The TPC provides excellent momentum-measuring and PID capabilities in a high particle-multiplicity environment of the heavy-ion collisions.
A plot of particle-species separation can be found in Figure~\ref{TpcPid}.
% -----------------------------------------------------------------------------------------------------------
\section{Time-Of-Flight Detector (TOF)\label{TOFsection}} 

The main purpose of the Time-Of-Flight (TOF) detector~\cite{TOFproposal} is to extend the PID capabilities of STAR into higher 
momenta via the measurements of the time of flight from its point of origin of the charged particles (or vertex) to the TOF 
detector, thus measuring its velocity. It was partially installed at STAR in 2009 and the remainder was added in 2010 
to make the TOF fully operational.

\begin{figure}[!htb]
\begin{center}
 \includegraphics[width=0.6\textwidth]{img/TOF_pid}\\
\end{center}
\caption{\label{TOF_pid}PID capabilities of the TOF via measurement of the inverse velocity $1/\beta$ versus \pt~\cite{TOFpid}. A clear separation of the particle species is visible, moreover it reaches further in \pt, compared to \dedx\ measurements of the TPC.}
\end{figure}

The TOF consists of 120 trays of Multi-gap Resistive Plate Chambers (MRPC)\nomenclature{MRPC}{Multi-gap Resistive Plate
Chamber}~\cite{MRPC} th	at cover $|\eta| \lesssim 0.96$ and $0 < \phi < 2\pi$. The time of the original collision is measured 
by the Vertex Position Detector (VPD\nomenclature{VPD}{Vertex Position Detector} --- see section~\ref{VPD}). The leading edge of 
the signal is sampled with \SI{25}{\pico\second} binning. Corrections can be made for the decay products of particles that 
decayed outside of the primary vertex, such as K$^0$ or $\Lambda$. The PID capabilities of TOF are demonstrated in Figure~\ref{TOF_pid}.

% -----------------------------------------------------------------------------------------------------------
\section{Heavy Flavor Tracker (HFT)\label{HFTsection}} 

The Heavy Flavor Tracker (HFT) was installed at STAR for the years 2014 -- 2016 data 
taking \cite{HftTdr, HFTLeo, HftFinal}. It is the innermost 
sub-detector of STAR and consists of 4
layers of silicon detectors. From the outermost layer in: A double-sided strip detector, called Silicon Strip
Tracker/Detector
(SST/SSD\nomenclature{SST/SSD}{Silicon Strip Tracker/Detector}), which is a previously installed detector in STAR that has been
refurbished and upgraded with new electronics; Next, the second outermost layer is formed by conventional silicon 
pad detectors
of Intermediate Silicon Tracker (IST\nomenclature{IST}{Intermediate Silicon Tracker}) 
with rectangular pads; Finally, the two innermost layers consist of the Pixel (PXL\nomenclature{PXL}{Pixel Detector}) 
Detector
that employs the novel MAPS technology that was used for the first time in a collider experiment. The overview of the 
HFT
subdetectors is summarized in Table~\ref{HFTtab}.

\begin{table}[htb]
\caption{\label{HFTtab}HFT subdetectors and their properties. For the SSD, the hit resolution $\sigma_{xy}$ and $\sigma_z$ are listed, instead of the pitch, because it is a double sided strip detector where the strips are not perpendicular.}
\begin{center}
\begin{tabular}{lcccc}
\toprule
System & Type & Radius [cm] & Pitch--$r\phi$ [$\upmu$m] & Pitch--$z$ [$\upmu$m]\\
\midrule
SSD & Double-sided strip detector & 22 & $\sigma$ = 20 & $\sigma$ = 740 \\
IST & Silicon pad detector & 14 & 600 & 6000 \\
PXL & Silicon pixel detector & 2.7, 8 & 20.7 & 20.7 \\
\bottomrule
\end{tabular}
\end{center}
\end{table}  



\begin{figure}[!htb]
\begin{center}
 \includegraphics[width=0.6\textwidth]{img/DCAXy}\\
\end{center}
\caption{\label{DCA}Distance of Closest Approach (DCA\nomenclature{DCA}{Distance of Closest Approach}) from the PV in the $xy$-plane for identified particles~\cite{D0v2paper}.}
\end{figure}

The purpose of the IST and the SSD is to guide the track from the TPC to the innermost layers of HFT--PXL in the high-track-multiplicity environment of STAR. The PXL provides an unparalleled pointing precision thanks to its high granularity as well as the proximity to the primary vertex. The beam pipe had to be replaced with a narrower one, in order to accommodate for the low radius of the innermost layer of PXL\@. The resolution of the distance of closest approach (DCA resolution) can be seen in Figure~\ref{DCA}. For high-momentum tracks the resolution is as low as \SI{20}{\micro\metre}.

\begin{figure}[!htb]
\begin{center}
 \includegraphics[width=0.7\textwidth]{img/HFT_layers}\\
\end{center}
\caption{\label{HFT_layers}Average resolution from the TPC down to the innermost layer of the PXL detector~\cite{KubaVyzkumak}.}
\end{figure}

\subsection{Pixel (PXL) detector}
The two innermost layers of the HFT consist of the Pixel (PXL) detector which employs a unique design as it is the
first detector using the Monolythic Active Pixel Sensor (MAPS\nomenclature{MAPS}{Monolythic Active Pixel Sensor})
technology. This enables the pixel sensor to have a miniscule pixel pitch of \SI{20.7}{\micro\metre} is extremely close to the center of the beam pipe (the innermost layer has a radius of 2.7$\,$cm) while maintaining
a small radiation length. This is important for the recognition of secondary and tertiary vertices of charm- and beauty-mesons
decays. The resolution of the distance of closest approach (DCA) of the reconstructed 
tracks to the primary vertex, depending on the particle species, is shown in Figure~\ref{DCA}.



% -----------------------------------------------------------------------------------------------------------
\section{Zero Degree Calorimeter (ZDC)\label{ZDCsection}} \nomenclature{ZDC}{Zero Degree Calorimeter}


The Zero Degree Calorimeter (ZDC)~\cite{ZDC, ZDCSMD} is placed on both sides of the RHIC tunnel behind the first dipole magnet. This placement gives the ZDC the unique capability of measuring the energy of non-charged particles, such as spectator neutrons, without any contribution of the charged particles, because the neutral particles continue in a straight line and the charged ones are deflected by the RHIC magnetic field. The ZDC serves as an important trigger detector, it is used
to determine the frequency of collisions, and can be utilized for the measurement of the collision event plane, and the polarization of the protons.

\begin{figure}[!htb]
\begin{center}
  \includegraphics[width=0.65\textwidth]{img/ZDC_in_STAR}\\
  \includegraphics[width=0.8\textwidth]{img/ZDC_scheme}
\end{center}
\caption{\label{ZDC_scheme}ZDC positions in the RHIC tunnel (top image) and the ZDC layout (bottom image)~\cite{ZDCSMD}. }
\end{figure}


The ZDC consists of two identical sides, both placed in the RHIC tunnel, behind 
the first deflecting magnets. Each side consists of 3 towers made of a tungsten absorber, sensitive volume, consisting of plastic optical fibers, and a photomultiplier tube (PMT)\@. A layout of the ZDC detector and the positions of the two ZDC sides are shown in Figure~\ref{ZDC_scheme}\@. A photo of the ZDC assembly is shown in Figure~\ref{ZDC_photo}\@. As a neutral particle traverses the absorber, it creates a shower which emits Cherenkov radiation in the plastic fibers which are bundled together at a 45° angle to increase their sensitivity. The ZDC towers do not use scintillators, because the deposited energy (tens to thousands of GeV) is high enough that the Cherenkov radiation in the fibers is sufficient. The light is then carried into the PMT which is placed freely on the optical fibers bundle without any optical grease or glue.

\begin{figure}[!htb]
\begin{center}
  \includegraphics[width=0.55\textwidth]{img/ZDC_modules_photo}
\end{center}
\caption{\label{ZDC_photo}Photo of the ZDC assembly installed between RHIC accelerator tubes~\cite{ZDCSMD}. }
\end{figure}

Behind the first tower, there is the Shower-Maximum Detector, consisting of two overlaid layers of scintillator strips placed perpendicularly to the $z$ direction. The SMD adds spatial information to the ZDC which is crucial to the event-plane determination, using spectator particles, and for the determination of the proton polarization\@. The scintillator strips are glued to optical fibers which lead the emitted light into a 16-channel PMT (enclosed in the black box on top of the assembly in Figure~\ref{ZDC_photo})\@.

The entire detector is placed on rails between the RHIC accelerator tubes and can be manipulated using a chain crank.

\subsection{Calibration of the ZDC towers\label{ZDCcalibration}}
At the start of each ion run, the ZDC towers are calibrated as the PMT . When an ultra-peripheral
collision\nomenclature{UPC}{Ultra-Peripheral Collision} occurs, with a high probability, the nucleus is excited and loses energy by emiting neutrons
(typically one or two). These neutrons have low momenta (units to tens of MeV/$c$) in the frame of the emitting nucleus. Therefore, in the laboratory frame, the neutron momenta do not differ significantly from the per nucleon momentum of the emitting nucleus (i.e.\ $\sim$100$\,$GeV$/c$ from a Au ion at top RHIC energy). These neutrons manifest themselves in the ZDC in the form of Single- and Double-Neutron Peaks
(SNP\nomenclature{SNP}{Single-Neutron Peak} and DNP\nomenclature{DNP}{Double-Neutron Peak}, respectively),
where the SNP has the energy per nucleon of the beam and the DNP has double the energy of the SNP, i.e.\ if
the ion beam has $\snn = 200\,$GeV, the SNP will sit at 100$\,$GeV and the DNP at 200$\,$GeV\@.

The ADC readout value is proportional to the energy loss in the sensitive volume and the high voltage applied to the PMT\@. Therefore, we can calibrate the output of the ZDC towers by changing the high voltage (HV\nomenclature{HV}{High Voltage}) applied to the PMTs. 

When a single neutron, emitted from a nucleus at 100$\,$GeV, hits the three towers of the ZDC, simulations show that the ratio of energy loss
in the three towers should be 
approximately 6:3:1~\cite{ZDCphysics}\@. We analyze the SNP created in heavy-ion collisions and adjust the high voltage so that the ADC value 
distribution of each tower matches the ideal ratio of 6:3:1 as closely as possible. Equally, we expect the ZDC 
single-neutron peak (SNP) to
be at the same ADC value in both, the East and the West side. In reality, these two conditions are not achieved precisely, but the voltages are adjusted in an iterative process to be close to the ideal case.



% \begin{figure}[htb]
% \begin{center}
% \includegraphics[width=.49\textwidth]{img/17038001_east_1}
% \includegraphics[width=.49\textwidth]{img/17038001_east_2}
% \includegraphics[width=.49\textwidth]{img/17038001_east_3}
% \end{center}
% \caption{\label{gain}Illustration of the gains of the individual towers in the ZDC gain analysis from Run16.}
% 
% \end{figure}



For the calibration, in each step, we take a dedicated ZDC run with
$\sim$10$\,$M events (at full luminosity, this takes less than $5\,$min). Then, we
look at the resulting Single-Neutron Peak (SNP\nomenclature{SNP}{Single-Neutron Peak}).
For an illustration of the SNP see Figure~\ref{SNPillustration} and the gain of the single
towers can be seen in Figure~\ref{gain}.
During the calibration, we try to shift the SNP so that it is at a safe ADC value so that the whole width of the SNP is in the ZDC measured
range. Also, the ratio between the gain of the ZDC towers should be corrected to match the ideal ratio 6:3:1\@. If the
SNP is too low and/or the ratio between the towers' gain is wrong,
the voltage in the photomultiplier tubes has to be adjusted and then the whole procedure
has to be repeated until the desired ratio of the gains is close enough.

The gain on the PMTs follows a power law
\begin{equation}
G = aU^b
\end{equation}
where $G$ is a gain and $U$ stands for voltage. The coefficients $a$ and $b$ differ for each PMT, but approximately they are close to the values of~\cite{ZDCvoltsDependence}
\begin{equation}
b=4.2\,, \qquad a=4.0\,.
\end{equation}
These values were used before the run 2018, however, in 2018, the PMTs were taken out for refurbishment and the values $a$ and $b$ were measured for all the PMTs. This measurement is described in the next section.

The desired position of the SNP is usually set as 60 ADC values. The gain at the SNP is obtained via a fit of the gain, using two gaussians for the SNP and DNP on an exponential background (which empirically fits the background the best)\$. The mean of the DNP gaussian is set as 2$\times$ the mean of the SNP\@.
% \begin{equation}
%  N = A \exp(-D x) + B \exp\left[-\frac{(x-G_\mathrm{SNP})^2}{\sigma_\mathrm{SNP}^2}\right]  + C \exp\left[-\frac{(x-G_\mathrm{DNP})^2}{\sigma_\mathrm{DNP}^2}\right]
% \end{equation}
To calculate

the desired voltages we use the formula
\begin{equation}
U_{\text{result}} = U_\text{current}\left(\frac{G_\text{desired}}{G} 
\frac{R_\text{desired}}{R}\right)^{1/b}
\end{equation}
where $G$ is the current position of the neutron peak, $G_\text{desired}$ is the desired position
of the neutron peak (currently 60), $R_\text{desired}$ is the desired ratio between the gain of the ADC SUM
tower and the current tower, and $R$ is the current ratio.

This calibration has to be performed at the beginning of each ion run at the top energy down to the energy per nucleon of about $E_\mathrm{N}\gtrsim40\,$GeV\@ as bellow this energy, the SNP falls bellow the sensitive range of the ZDC\@.

\subsection{ZDC Tower PMT replacement and calibration}
The calibrations, described in the previous section, have shown that the yield from SNP in the ZDC towers is lowering in between the RHIC runs each year. The possible causes include gas leaks into the PMTs, slow decremental loss of the PMT dinodes properties, or darkening of the ZDC optical fibers. 

In 2018, before the isobar (${}^{96}$Zr and ${}^{96}$Ru) running at $\snn = 200\,$GeV, the ZDC towers were thoroughly tested and several PMTs were replaced. The original goal was to replace all of the ZDC PMTs with spares from the Brahms experiment, however a check had to be made, whether the spares performance exceeded one of the original PMTs.

\begin{figure}[!htb]
\begin{center}
 \includegraphics[width=0.47\textwidth]{img/ZDC_PMT_test_setup.jpg}$\quad$\includegraphics[width=0.47\textwidth]{img/ZDC_PMT_test_people.jpg}
\end{center}
\caption{\label{photo_ZDC_PTM_setup}Photos of the ZDC-PMT and possible spare-PMT test setup. Left: Inside of the black box. Right: The entire setup deployed inside the RHIC tunel.}
\end{figure}

A photo of the experimental setup is shown in Figure~\ref{photo_ZDC_PTM_setup}\@. It consists of a flashing weak LED light, a control PMT, and the tested PMT, all enclosed in a wooden black box. The LED is controlled by a Beagle Board, powered via a USB charger which supplies the control PMT as well. The tested PMT uses an external high-voltage (HV) power supply. Both, the tested PMT and the control are read out in an oscilloscope that uses the control PMT as a trigger. The main purpose of the control PMT is, however, to measure the light output of the LED and ensure that it is stable throughout the measurement.

The first measurement consisted of measuring the gain performance of the Brahms-spare PMTs. All data from the test are available in~\cite{PMT_spreadsheet}\@. Gain at a set HV of 2500$\,$V was measured as well as the a test whether a second pulse --- a, so called, afterpulse --- appears behind main pulse caused by the LED flash. The afterpulses are caused by ionized gas inside of the PMT and may indicate that the sealing had been compromised. The gain was measured as height of the pulse in the oscilloscope. The requirements for the PMTs were high gain, stability throughout our measurement, low afterpulsing, and similar time performance to the PMTs, already used in the ZDC towers. Approximately 100 PMTs were cleaned and tested, however --- except for 2 PMTs --- the Brahms spares did not exceed the performance of the PMTs they were supposed to replace. The 2 well-performing PMTs were selected as replacements for the PMTs in the towers with the lowest gains. Moreover, all of the spares had relatively large afterpulses, compared to the old ZDC towers which, however, afterpulse as well.

Next, all the PMT from the ZDC towers and the chosen spares were tested for linearity of gain vs the light input. The voltage on the LED light was risen and the increase in gain was compared to the control PMT which had been previously proven to be linear. The results were fitted by a linear function. All of the ZDC towers and the tested spares followed a linear response to the change of light input.

\begin{table}[htb] 
\caption{\label{HVtable}Swaps and final ZDC voltages after the calibration performed for the 2018 isobar run.}
\label{corected}
\begin{center}
\begin{tabular}{lccc}
\toprule
 &Tower&Swapped with&Operating voltage [V]\\
\midrule
East  &1 & same & 2444 \\
      &2 & swapped from East3 & 2633 \\
      &3 & new AA1783 & 2329 \\
\midrule
West  &1 & same & 2431  \\
      &2 & same & 3000  \\
      &3 & new H2431-50 & 2101 \\
\bottomrule
\end{tabular}
\end{center}
\end{table}

\begin{figure}[!htb]
% \begin{center}
\includegraphics[width=0.33\textwidth]{img/ZDC_E_1}
\includegraphics[width=0.33\textwidth]{img/ZDC_E_2}
\includegraphics[width=0.33\textwidth]{img/ZDC_E_3}\\
\includegraphics[width=0.33\textwidth]{img/ZDC_W_1}
\includegraphics[width=0.33\textwidth]{img/ZDC_W_2}
\includegraphics[width=0.33\textwidth]{img/ZDC_W_3}\\
\includegraphics[width=0.33\textwidth]{img/AA1783}
\includegraphics[width=0.33\textwidth]{img/H2431-50}
% \end{center}
\caption{\label{ZDCgains}Measurement of the dependence of the PMT gain on the applied high voltage. All the ZDC towers were measured as well as two spares that were used instead of ZDC East3 and ZDC West3\@. AA1783 and H2431-50 are serial numbers of the PMTs. The data are fitted by the power law~\eqref{powerLaw} where $a=1/U_0$ and $b$ is the Exponent.}
\end{figure}

An important measurement for the calibration of the ZDC, described in the previous section, is the response of the gain to the increase of HV on the PMT\@. The gain follows a power law~\eqref{powerLaw} in which the exponent $b$ has to be measured to make the calibration more precise and less time consuming in the beginning of the run. The results of the tests are plotted in Figure~\ref{ZDCgains}\@. The HV was varied between 1500$\,$V and 3500$\,$V where the highest HV value of 3500$\,$V is outside of the operating range of the ZDC power supply, but serves as a stability check of the PMTs. All the PMTs follow the power law~\eqref{powerLaw}, however two of the original ZDC towers -- ZDC East2 and ZDC West3 had low gains overall and were replaced in the end. According to this measurement and the calibration, several swaps of the PMTs were decided and 2 PMTs were replaced by the Brahms spares. The swaps and operating voltages, that are used since the calibration in Run 2018, are summarized in Table~\ref{HVtable}\@. The new PMTs were installed at the furthermost places from the IP, because the light inputs in the last towers are the lowest the afterpulses fall linearly with the initial pulses. 





%-------------------------------------------------------------------------------------------------------------------

\section{Vertex-Position Detector (VPD)\label{VPD}}

\begin{figure}[!htb]
\begin{center}
 \includegraphics[width=0.7\textwidth]{img/VPD}\\
\end{center}
\caption{\label{VPDassembly}VPD assembly~\cite{VPD}.}
\end{figure}

The VPD~\cite{VPD} consists of two identical assemblies, positioned \SI{5.7}{\metre} from the center of STAR\@. A schematic 
picture of an assembly can be seen in Figure~\ref{VPDassembly}. Each of them is made of 19 detectors, consisting of a lead 
converter, followed by a fast plastic scintillator read by a photo-multiplier tube. The main purpose of the VPD is to measure 
photons from $\uppi^0$ decays in forward direction and thus accurately measure the time of the collision, as well as 
the position of the primary vertex with the resolution of $\sim \SI{1}{\centi\metre}$ and  $\sim \SI{2.5}{\centi\metre}$ 
for Au+Au and p+p collisions, respectively.

\section{Data Acquisition}
\subsection{Trigger}

