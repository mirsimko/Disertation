The Zero Degree Calorimeter (ZDC)~\cite{ZDC, ZDCSMD} is placed on both sides of the RHIC tunnel behind the first dipole magnet. This placement gives the ZDC the unique capability of measuring the energy of non-charged particles, such as spectator neutrons, without any contribution of the charged particles, because the neutral particles continue in a straight line and the charged ones are deflected by the RHIC magnetic field. The ZDC serves as an important trigger detector, it is used
to determine the frequency of collisions, and can be utilized for the measurement of the collision event plane, and the polarization of the protons.

\begin{figure}[!htb]
\begin{center}
  \includegraphics[width=0.65\textwidth]{img/ZDC_in_STAR}\\
  \includegraphics[width=0.8\textwidth]{img/ZDC_scheme}
\end{center}
\caption{\label{ZDC_scheme}ZDC positions in the RHIC tunnel (top image) and the ZDC layout (bottom image)~\cite{ZDCSMD}. }
\end{figure}


The ZDC consists of two identical sides, both placed in the RHIC tunnel, behind 
the first deflecting magnets. Each side consists of 3 towers made of a tungsten absorber, sensitive volume, consisting of plastic optical fibers, and a photomultiplier tube (PMT)\@. A layout of the ZDC detector and the positions of the two ZDC sides are shown in Figure~\ref{ZDC_scheme}\@. A photo of the ZDC assembly is shown in Figure~\ref{ZDC_photo}\@. As a neutral particle traverses the absorber, it creates a shower which emits Cherenkov radiation in the plastic fibers which are bundled together at a 45° angle to increase their sensitivity. The ZDC towers do not use scintillators, because the deposited energy (tens to thousands of GeV) is high enough that the Cherenkov radiation in the fibers is sufficient. The light is then carried into the PMT which is placed freely on the optical fibers bundle without any optical grease or glue.

\begin{figure}[!htb]
\begin{center}
  \includegraphics[width=0.75\textwidth]{img/ZDC_modules_photo}
\end{center}
\caption{\label{ZDC_photo}Photo of the ZDC assembly installed between RHIC accelerator tubes~\cite{ZDCSMD}. }
\end{figure}

Behind the first tower, there is the Shower-Maximum Detector, consisting of two overlaid layers of scintillator strips placed perpendicularly to the $z$ direction. The SMD adds spatial information to the ZDC which is crucial to the event-plane determination, using spectator particles, and for the determination of the proton polarization\@. The scintillator strips are glued to optical fibers which lead the emitted light into a 16-channel PMT (enclosed in the black box on top of the assembly in Figure~\ref{ZDC_photo})\@.

The entire detector is placed on rails between the RHIC accelerator tubes and can be manipulated using a chain crank.

\subsection{Calibration of the ZDC towers\label{ZDCcalibration}}
At the start of each ion run, the ZDC towers are calibrated as the PMT . When an ultra-peripheral
collision\nomenclature{UPC}{Ultra-Peripheral Collision} occurs, with a high probability, the nucleus is excited and loses energy by emiting neutrons
(typically one or two). These neutrons have low momenta (units to tens of MeV/$c$) in the frame of the emitting nucleus. Therefore, in the laboratory frame, the neutron momenta do not differ significantly from the per nucleon momentum of the emitting nucleus (i.e.\ $\sim$100$\,$GeV$/c$ from a Au ion at top RHIC energy). These neutrons manifest themselves in the ZDC in the form of Single- and Double-Neutron Peaks
(SNP\nomenclature{SNP}{Single-Neutron Peak} and DNP\nomenclature{DNP}{Double-Neutron Peak}, respectively),
where the SNP has the energy per nucleon of the beam and the DNP has double the energy of the SNP, i.e.\ if
the ion beam has $\snn = 200\,$GeV, the SNP will sit at 100$\,$GeV and the DNP at 200$\,$GeV\@.

The ADC readout value is proportional to the energy loss in the sensitive volume and the high voltage applied to the PMT\@. Therefore, we can calibrate the output of the ZDC towers by changing the high voltage (HV\nomenclature{HV}{High Voltage}) applied to the PMTs. 

When a single neutron, emitted from a nucleus at 100$\,$GeV, hits the three towers of the ZDC, simulations show that the ratio of energy loss
in the three towers should be 
approximately 6:3:1~\cite{ZDCphysics}\@. We analyze the SNP created in heavy-ion collisions and adjust the high voltage so that the ADC value 
distribution of each tower matches the ideal ratio of 6:3:1 as closely as possible. Equally, we expect the ZDC 
single-neutron peak (SNP) to
be at the same ADC value in both, the East and the West side. In reality, these two conditions are not achieved precisely, but the voltages are adjusted in an iterative process to be close to the ideal case.



% \begin{figure}[htb]
% \begin{center}
% \includegraphics[width=.49\textwidth]{img/17038001_east_1}
% \includegraphics[width=.49\textwidth]{img/17038001_east_2}
% \includegraphics[width=.49\textwidth]{img/17038001_east_3}
% \end{center}
% \caption{\label{gain}Illustration of the gains of the individual towers in the ZDC gain analysis from Run16.}
% 
% \end{figure}



For the calibration, in each step, we take a dedicated ZDC run with
$\sim$10$\,$M events (at full luminosity, this takes less than $5\,$min). Then, we
look at the resulting Single-Neutron Peak (SNP\nomenclature{SNP}{Single-Neutron Peak}).
For an illustration of the SNP see Figure~\ref{SNPillustration}\@.
During the calibration, we change the voltages until at least 1 standard deviation of the SNP is in the ZDC measured
range. Also, the ratio between the gain of the ZDC towers is corrected to match the ideal ratio 6:3:1\@. If the position of the SNP is too low and/or the ratio between the towers' gain is wrong,
the voltage in the photomultiplier tubes has to be adjusted and then the whole procedure
has to be repeated until the ratio of the gains approximates the desired one well enough.


\begin{figure}[!htb]
\begin{center}
\includegraphics[width=.6\textwidth]{img/SNPandDNPexample.pdf}
\end{center}
\caption{\label{SNPillustration} Single-neutron and double-neutron peaks for the West towers for in a 2018 test run.}

\end{figure}

The gain on the PMTs follows a power law
\begin{equation} \label{powerLaw}
G = aU^b
\end{equation}
where $G$ is a gain and $U$ stands for voltage. The coefficients $a$ and $b$ differ for each PMT, but approximately they are close to the values of~\cite{ZDCvoltsDependence}
\begin{equation}
b=4.2\,, \qquad a=4.0\,.
\end{equation}
These values were used before the run 2018, however, in 2018, the PMTs were taken out for refurbishment and the values $a$ and $b$ were measured for all the PMTs. This measurement is described in the next section.

The desired position of the SNP is usually set as 60 ADC values. The gain at the SNP is obtained via a fit of the gain, using two gaussians for the SNP and DNP on an exponential background (which empirically fits the background the best)\$. The mean of the DNP gaussian is set as 2$\times$ the mean of the SNP\@.
% \begin{equation}
%  N = A \exp(-D x) + B \exp\left[-\frac{(x-G_\mathrm{SNP})^2}{\sigma_\mathrm{SNP}^2}\right]  + C \exp\left[-\frac{(x-G_\mathrm{DNP})^2}{\sigma_\mathrm{DNP}^2}\right]
% \end{equation}
To calculate
the desired voltages we use the formula
\begin{equation}
U_{\text{result}} = U_\text{current}\left(\frac{G_\text{desired}}{G} 
\frac{R_\text{desired}}{R}\right)^{1/b}
\end{equation}
where $G$ is the current position of the neutron peak, $G_\text{desired}$ is the desired position
of the neutron peak (currently 60), $R_\text{desired}$ is the desired ratio between the gain of the ADC SUM
tower and the current tower, and $R$ is the current ratio.

This calibration has to be performed at the beginning of each ion run at the top energy down to the energy per nucleon of about $E_\mathrm{N}\gtrsim40\,$GeV\@ as bellow this energy, the SNP falls bellow the sensitive range of the ZDC\@.

\subsection{ZDC Tower PMT replacement and calibration}
The calibrations, described in the previous section, have shown that the yield from SNP in the ZDC towers is lowering in between the RHIC runs each year. The possible causes include gas leaks into the PMTs, slow decremental loss of the PMT dinodes properties, or darkening of the ZDC optical fibers. 

In 2018, before the isobar (${}^{96}$Zr and ${}^{96}$Ru) running at $\snn = 200\,$GeV, the ZDC towers were thoroughly tested and several PMTs were replaced. The original goal was to replace all of the ZDC PMTs with spares from the Brahms experiment, however a check had to be made, whether the spares performance exceeded one of the original PMTs.

\begin{figure}[!htb]
\begin{center}
 \includegraphics[width=0.47\textwidth]{img/ZDC_PMT_test_setup.jpg}$\quad$\includegraphics[width=0.47\textwidth]{img/ZDC_PMT_test_people.jpg}
\end{center}
\caption{\label{photo_ZDC_PTM_setup}Photos of the ZDC-PMT and possible spare-PMT test setup. Left: Inside of the black box. Right: The entire setup deployed inside the RHIC tunel.}
\end{figure}

A photo of the experimental setup is shown in Figure~\ref{photo_ZDC_PTM_setup}\@. It consists of a flashing weak LED light, a control PMT, and the tested PMT, all enclosed in a wooden black box. The LED is controlled by a Beagle Board, powered via a USB charger which supplies the control PMT as well. The tested PMT uses an external high-voltage (HV) power supply. Both, the tested PMT and the control are read out in an oscilloscope that uses the control PMT as a trigger. The main purpose of the control PMT is, however, to measure the light output of the LED and ensure that it is stable throughout the measurement.

The first measurement consisted of measuring the gain performance of the Brahms-spare PMTs. All data from the test are available in~\cite{PMT_spreadsheet}\@. Gain at a set HV of 2500$\,$V was measured as well as the a test whether a second pulse --- a, so called, afterpulse --- appears behind main pulse caused by the LED flash. The afterpulses are caused by ionized gas inside of the PMT and may indicate that the sealing had been compromised. The gain was measured as height of the pulse in the oscilloscope. The requirements for the PMTs were high gain, stability throughout our measurement, low afterpulsing, and similar time performance to the PMTs, already used in the ZDC towers. Approximately 100 PMTs were cleaned and tested, however --- except for 2 PMTs --- the Brahms spares did not exceed the performance of the PMTs they were supposed to replace. The 2 well-performing PMTs were selected as replacements for the PMTs in the towers with the lowest gains. Moreover, all of the spares had relatively large afterpulses, compared to the old ZDC towers which, however, afterpulse as well.


\begin{figure}[!htb]
% \begin{center}
\includegraphics[width=0.32\textwidth]{img/ZDC_E_1}
\includegraphics[width=0.32\textwidth]{img/ZDC_E_2}
\includegraphics[width=0.32\textwidth]{img/ZDC_E_3}\\
\includegraphics[width=0.32\textwidth]{img/ZDC_W_1}
\includegraphics[width=0.32\textwidth]{img/ZDC_W_2}
\includegraphics[width=0.32\textwidth]{img/ZDC_W_3}\\
\includegraphics[width=0.32\textwidth]{img/AA1783}
\includegraphics[width=0.32\textwidth]{img/H2431-50}
% \end{center}
\caption{\label{ZDCgains}Measurement of the dependence of the PMT gain on the applied high voltage. All the ZDC towers were measured as well as two spares that were used instead of ZDC East3 and ZDC West3\@. AA1783 and H2431-50 are serial numbers of the PMTs. The data are fitted by the power law~\eqref{powerLaw} where $a=1/U_0$ and $b$ is the Exponent.}
\end{figure}

Next, all the PMT from the ZDC towers and the chosen spares were tested for linearity of gain vs the light input. The voltage on the LED light was risen and the increase in gain was compared to the control PMT which had been previously proven to be linear. The results were fitted by a linear function. All of the ZDC towers and the tested spares followed a linear response to the change of light input.


\begin{table}[!htb] 
\caption{\label{HVtable}Swaps and final ZDC voltages after the calibration performed for the 2018 isobar run.}
\label{corected}
\begin{center}
\begin{tabular}{lccc}
\toprule
 &Tower&Swapped with&Operating voltage [V]\\
\midrule
East  &1 & same & 2444 \\
      &2 & swapped from East3 & 2633 \\
      &3 & new AA1783 & 2329 \\
\midrule
West  &1 & same & 2431  \\
      &2 & same & 3000  \\
      &3 & new H2431-50 & 2101 \\
\bottomrule
\end{tabular}
\end{center}
\end{table}


An important measurement for the calibration of the ZDC, described in the previous section, is the response of the gain to the increase of HV on the PMT\@. The gain follows a power law~\eqref{powerLaw} in which the exponent $b$ has to be measured to make the calibration more precise and less time consuming in the beginning of the run. The results of the tests are plotted in Figure~\ref{ZDCgains}\@. The HV was varied between 1500$\,$V and 3500$\,$V where the highest HV value of 3500$\,$V is outside of the operating range of the ZDC power supply, but serves as a stability check of the PMTs. All the PMTs follow the power law~\eqref{powerLaw}, however two of the original ZDC towers -- ZDC East2 and ZDC West3 had low gains overall and were replaced in the end. According to this measurement and the calibration, several swaps of the PMTs were decided and 2 PMTs were replaced by the Brahms spares. The swaps and operating voltages, that are used since the calibration in Run 2018, are summarized in Table~\ref{HVtable}\@. The new PMTs were installed at the furthermost places from the IP, because the light inputs in the last towers are the lowest the afterpulses fall linearly with the initial pulses. 


