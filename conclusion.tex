In this thesis, the first direct reconstruction of $\Lambda_\mathrm{c}$ in Au+Au collisions at \snnFull\ is reported. Baryon-to-meson ratio in the charm sector has been measured in the form of \Lambdac/\dzero\ yield ratio. 

The measurement of the \Lambdac\ in heavy-ion collisions represents a challenging task, due to the short life time ($c\tau \approx 60\,\upmu$m) and the large combinatorial background, because the direct reconstruction requires three body decays that involve pions, protons, and kaons which are abundant products of Au+Au collisions. The measurement was enabled thanks to the utilization of the new silicon detector HFT whose two innermost layers were placed very close to the interaction point and utilize the MAPS technology for the first time in heavy-ion experiments\@. Because of this, the innermost layers of the HFT have excellent granularity. Machine-learning techniques were utilized to separate the signal from the combinatorial background, which come mostly from the primary vertex. The relatively large number of the \Lambdac, that was obtained thanks to these detection techniques as well as the large statistical sample, recorded by STAR in the years 2014 and 2016, allowed for the \Lambdac/\dzero\ measurement to be differentiated in the transverse momentum and centrality for the first time in heavy-ion collisions.

The \Lambdac/\dzero\ yield ratio in Au+Au collisions at midrapidity ($|y| < 1$) is significantly larger than that in p+p collisions simulated by PYTHIA and is comparable with the baryon-to-meson ratios in the light- and strange-flavor hadrons in the same kinematic region. The measured ratio is also compatible with models that include charm-quark coalescence in the hadronization process.
The \Lambdac/\dzero\ ratio increases with more central collisions which is also consistent with models that include charm-quark coalescence.

The \pt-integrated \Lambdac/\dzero\ ratio was obtained by extrapolating the measurement to low \pt\ and is calculated as $0.80\pm0.12\text{(stat.)}\pm0.22\text{(sys., data)}\pm0.41\text{(sys., models extrapolation)}$\@. This shows that the \Lambdac\ make a sizable contribution to the total charm yield in heavy-ion collisions at RHIC. The \pt-integrated ratio is also consistent with the THERMUS model, in our comparison calculated with the thermal freeze-out temperature of $T_\mathrm{ch} = 160\,$MeV\@.


Two service tasks were performed as a part of this thesis: Work on the slow simulator for the HFT-Pixel detector and, secondly, maintenance, calibration, and documentation of the STAR Zero Degree Calorimeter (ZDC). 

The HFT recorded data at STAR the years 2014--2016 and was a critical upgrade for the precise measurements of open-charm hadrons, including enabling the measurement of the \Lambdac\@. The HFT-Pixel slow-simulator implementation and evaluation helped with the clustering component of the embedding procedure, in which simulated tracks are embedded into measured collisions at STAR\@. 

The ZDC plays a vital role for triggering and the assessment of instant luminosity at STAR, and the determination of centrality in ion--ion collisions. The task was to ensure smooth operation of this critical component during the data taking at STAR\@. The ZDC was calibrated, using the single-neutron peak, at the start of each run. The performance of the ZDC photo-multiplier tubes has been evaluated and several have been upgraded to ones with higher gain. Moreover, the entire setup of the ZDC, together with the control, trigger, and readout electronics have been documented~\cite{ZDCmanual}\@.

Thanks to the HFT and the large data samples of Au+Au collisions, recorded in 2014 and 2016, STAR has performed measurements that can draw an overall picture of the behavior of the charm quark in heavy-ion collisions. From the understanding of energy loss and collective movement of the charm quark in the QCD medium, provided by the \dzero\ and D$^\pm$ \Raa, $v_2$, and $v_3$ measurements, to the understanding of the hadronization process, given by the \Lambdac\ and \Ds\ measurements.

\begin{figure}[!htb]
\centering
\includegraphics[width=.5\textwidth]{img/LcITS}
\caption[Projection of statistical uncertainties of the \Lambdac/\dzero\ measurement with the improved resolution of ALICE ITS\@.]{Projection of statistical uncertainties of the \Lambdac/\dzero\ measurement with the improved resolution of ALICE ITS using $1.7\times 10^{10}$ central collisions (0--10$\,\%$), corresponding to an integrated luminosity of 10$\,$nb$^{-1}$~\cite{ITS_CDR}.}
\label{ITS_Lc_conclusion}
\end{figure}

ALICE detector at the LHC at CERN is currently undergoing an overhaul upgrade of its Inner Tracking System which will consist of 7 layers of MAPS pixel sensors. When finished, this will greatly improve the tracking accuracy, and as a result, the separation of secondary vertices of open-heavy-flavor hadrons from the primary vertex. At the LHC energies of $\snn = 5.5\,$TeV, the cross-section of charm-quark production is larger than at RHIC, and ALICE plans to collect a large sample of Pb+Pb collisions. All of this is planed, in part, to create a next-level-of-precision measurement of the \Lambdac, which will provide, together with other precise measurements of other open-heavy-flavor hadrons, an accurate quantitative understanding of the behavior of the charm quarks on the QGP, as well as heavy-quark hadronization in heavy-ion collisions. A projection of the error bars of the \Lambdac/\dzero\ yield ratio measurement from LHC Run-3 Pb+Pb sample, that will be recorded by ALICE with the ITS upgrade, is shown in Figure~\ref{ITS_Lc_conclusion}\@.

% an be improved by a better cut optimization and by using the 
% The statistical uncertainty cmixed-event method for the extraction of the combinatorial background. The systematic uncertainty needs a more thorough study of all the factors, contributing to it. Therefore, there is room for improvement of this result before publication. More observables are also being extracted from the data, including \Lambdac\ azimuthal anisotropy, and the ratio between the baryon $\Lambdac^+$ and the anti-baryon $\overline{\Lambdac}^-$ which has a potential to shed more light on the process of quark coalescence.
% 
% Moreover, STAR has recorded approximately twice more data in 2016, compared to the year 2014 with better performance of the HFT\@. This will allow to measure the ratio of $\Lambda_\mathrm{c}$ to D$^0$ in more centrality and $p_\mathrm{T}$ bins to put more constrain on theoretical predictions. 
