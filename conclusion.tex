% ALICE has measurements of the $\Lambda_\mathrm{c}$ in p--p and p--Pb collisions with good significance in multiple $p_\mathrm{T}$ bins. The efficiency corrections and systematic uncertainties are currently being inferred.
% 
% For the LHC Run 3, ALICE is preparing a major upgrade of the TPC and, importantly, the ITS  which will enable the measurement of $\Lambda_\mathrm{c}$ in Pb--Pb with high statistics.
% 
% With the SMOG instrument, LHCb was able to take novel p+Ar data at $\sqrt{s_\mathrm{NN}} = 110\,$GeV, in which the $\Lambda_\mathrm{c}$ were already observed with high significance.
% 
% LHCb has recently joined the heavy-ion program at the LHC and has taken high-statistics p--Pb data, in which the $\Lambda_\mathrm{c}$ is being analyzed and, in 2017, LHCb will take Pb--Pb data. As a dedicated experiment for charm and beauty analyses, LHCb can achieve high precision in the $\Lambda_\mathrm{c}$ measurements.

STAR has measured $\Lambda_\mathrm{c}$ for the first time in heavy ion--ion collisions thanks to the addition of the HFT in the years 2014--2016. The ratio of the yields of $N(\Lambda_\mathrm{c}^+ + \overline{\Lambda_\mathrm{c}}^-)/N(\mathrm{D^0 + \overline{D^0}}) = 1.31 \pm 0.26\text{(stat.)} \pm 0.42$(sys.)  was calculated from the Au+Au data taken in 2014 for centralities of 10--60$\,\%$ and $\SI{3}{\giga\electronvolt}/c < \pt < \SI{6}{\giga\electronvolt}/c$\@. The results point to an enhancement of the $\Lambda_\mathrm{c}$ within the measured range and are consistent within 2$\sigma$ with theoretical calculations that contain quark coalescence.

The statistical uncertainty can be improved by a better cut optimization and by using the mixed-event method for the extraction of the combinatorial background. The systematic uncertainty needs a more thorough study of all the factors, contributing to it. Therefore, there is room for improvement of this result before publication. More observables are also being extracted from the data, including \Lambdac\ azimuthal anisotropy, and the ratio between the baryon $\Lambdac^+$ and the anti-baryon $\overline{\Lambdac}^-$ which has a potential to shed more light on the process of quark coalescence.

Moreover, STAR has recorded approximately twice more data in 2016, compared to the year 2014 with better performance of the HFT\@. This will allow to measure the ratio of $\Lambda_\mathrm{c}$ to D$^0$ in more centrality and $p_\mathrm{T}$ bins to put more constrain on theoretical predictions. 