\chapter*{Author's Contribution}
All of the work, described in this thesis, was done in the context of the Solenoidal Tracker at RHIC (STAR) collaboration. As an active member of this collaboration, I was able to perform two service tasks on two of the STAR subdetectors and analysis of the \Lambdac\ baryon on STAR data, recorded in 2014 and 2016 from Relativistic Heavy-Ion Collider (RHIC) Au+Au collisions at full energy (\snnFull)\@. In this chapter, I describe the tasks that I have performed for the STAR collaboration that led to the results, presented in this thesis.


\section*{Service tasks}
During the course of my Ph.D.\ studies, I had two major service tasks for the Solenoidal Tracker at RHIC (STAR) experiment: One was on the slow simulator on the Heavy Flavor Tracker (HFT --- for more detail see Section~\ref{HFTsection}) which was installed at STAR in the years 2014--2016.  The other one is the maintenance of the STAR Zero Degree Calorimeters (ZDC --- for more detail see Section~\ref{ZDCsection}).

\subsection*{HFT--Pixel Slow Simulator Evaluation}
Simulations are an integral part of every analysis on high-energy experiments. My task was to evaluate the slow simulator for the Pixel (PXL) detector of the HFT\@. Part of this work was performed during my stay at Lawrence Berkeley National Laboratory (LBNL\nomenclature{LBNL}{Lawrence Berkeley National Laboratory}) for 6 months in 2013--2014\@. The HFT~\cite{HFTLeo, HftFinal, HftTdr} is a state-of-the-art silicon vertex detector out of which the two innermost layers consist of the novel MAPS-technology based Pixel detector which provides a yet unparalleled tracking resolution that can facilitate the tracking of the relatively short-lived heavy-flavor hadrons via their decay products, e.g.\ the \Lambdac\ which was never measured in heavy-ion collisions before.

The point of a Slow simulator is to generate the output of the detector on the level of a single pixel, whereas the fast simulator operates on the level of a hit, i.e.\ an already reconstructed cluster of pixels. The slow simulator is mainly used in the, so called, embedding, in which the signal of a particle in the detector is simulated, using GEANT~\cite{GEANT} and then embedded into a real measured event and then reconstructed in the same manner as a real collision.
This is used at STAR to evaluate the detector efficiencies for particle tracks.


My task was to simulate hits of particles in the detector and then compare them to real measured particles. This comparison was done for cosmic particles and, later, for Au+Au collisions at the center-of-mass energy per nucleon $\snn = \SI{200}{\giga\electronvolt}$ from 2014\@. The particles were divided into bins by angle and then the sizes of the clusters and compared to the simulation output.

This work was later continued by Jakub Kvapil, to whom I served as an advisor.
His results, presented in~\cite{KubaVyzkumak}, were not only differentiated in angle, but also by particle species, identified by the Time Projection Chamber (TPC). More on the HFT slow simulator can be found in Section~\ref{HFTsection}\@.

\subsection*{Zero Degree Calorimeter (ZDC)}


\begin{table}[!htb]
\caption[Table of STAR on-call runs.]{\label{ZDConCall}Table of runs where the author was present as a ZDC on-call expert.}
\begin{center}
\begin{tabular}{llc}
\toprule
Year & Species & \snn\ (GeV)  \\
\midrule
  2015 & polarized p+p & 200.4  \\
       & polarized p+Au & 202.4  \\
       & polarized p+Al & 202.5  \\
  2016 & Au+Au & 200.0 \\
       & d+Au  & 200.7 \\
       & d+Au  & 62.4 \\
       & d+Au  & 19.7 \\
       & d+Au  & 39.0 \\
  2017 & polarized p+p& 499.8  \\
       & Au+Au & 54.4  \\
  2018 & ${}^{96}$Zr+${}^{96}$Zr & 200.0 \\
       & ${}^{96}$Ru+${}^{96}$Ru & 200.0 \\
       & Au+Au & 27.0  \\
       & Au+Au fixed target & 3.0  \\
       & Au+Au fixed target & 7.2  \\
  2019 & Au+Au & 14.6  \\
       & Au+Au & 19.6 \\
       & Au+Au fixed target & 3.9 \\
       & Au+Au & 7.7  \\
       & Au+Au fixed target & 4.5 \\
       & Au+Au fixed target & 7.8  \\
\bottomrule
\end{tabular}
\end{center}
\end{table}


The ZDC is an integral part of the STAR--trigger system, but also serves as an important tool for monitoring the instant luminosity at RHIC\@. It consists~\cite{ZDC, ZDCSMD} of two identical sides, both placed in the RHIC tunnel, behind 
the first deflecting magnets. Each side consists of 3 towers made of a tungsten absorber, scintillator volume and a photomultiplier tube.
Behind the first tower, there is the Shower-Maximum Detector (SMD) which adds spatial information to the ZDC\@.

In the years 2015--2019, I was personally responsible for the smooth operation of the ZDC system, partially on-site at BNL and remotely as an on-call expert. The RHIC runs from this period are listed in Table~\ref{ZDConCall}\@. Part of the duty is also the preparation of the detector before each run and its calibration (see Section~\ref{ZDCcalibration}). In 2018, as a part of  maintenance, several photomultilier tubes (PMT) were exchanged for ones with better performance. As an upcycling project, we refurbished and tested old PMTs from the currently phased-out Broad RAnge Hadron Magnetic Spectrometers (Brahms) experiment. This effort is described in more detail in Section~\ref{ZdcPmtTest}\@. This year, as a part of a plan to increase the reliability and stability of the detector, a new high-voltage power supply is going to be installed at the ZDC and deployed with a new slow-controls system for remote control of the source. Moreover, as there was no prior documentation on how to operate the detector, we created a manual~\cite{ZDCmanual} for maintenance of the ZDC together with Lukáš Kramárik and Jan Vaněk who continue as ZDC on-call experts. More on this effort is described in Section~\ref{ZDCsection}\@.





\section*{\Lambdac\ analysis}
The main topic of my research and this thesis is the analysis of the \Lambdac\ baryon at the STAR experiment at RHIC\@. Other than my supervisors, this analysis has been done in a close collaboration with a number of people from LBNL; most notably Sooraj Radhakrishnan, Guannan Xie, Xin Dong, Jochen Thaeder, Mustafa Mustafa, and Michael Lomnitz. Much of this work was done during my stays as an affiliate at LBNL in 2015 (2 months), 2016 (1 month), and 2018 (2 months)\@. 
As a result of the collaboration with LBNL, the \Lambdac\ analysis was published in the article Phys.\ Rev.\ Lett.\ 124, 172301 (2020)~\cite{LambdacPaper} where I figured as one of the primary authors responsible with providing feedback within the Physics-Working-Group and STAR-collaboration review processes and feedback to the referees.

Within the \Lambdac\ analysis, my tasks were to perform an independent check of the \Lambdac\ reconstruction and the difference between the reconstruction of $\Lambdac^+$ baryon and its antibaryon $\overline{\Lambdac}^-$\@. I also notably developed the mixed-event combinatorial-background subtraction method for the \Lambdac\ analysis and provided it in the general STAR open-heavy-flavor-analysis code~\cite{mixedEventCode}\@.



% \section*{Student advisory}
% 
% One of my other duties is advisory of students. One of them is Jakub Kvapil who is, on top of his work on the HFT, successfully working on the analysis of the D$^\pm$ meson which was presented at the Quark Matter 2017 conference, among others. He is currently finishing his M.Sc.\ degree. Another one is Zuzana Moravcová who is finishing her B.Sc.\ degree. She is working on machine learning methods of signal extraction of the D$^\pm$ meson and \Lambdac.

% \section*{Presentations}
% 
% This work has been presented at a number of international conferences and workshops:
% \begin{enumerate}
% \item Cracow School of Theoretical Physics, LIV Course, 2014, Cracow, Poland, Oral presentation: ``Simulations for the HFT--Pixel detector at the STAR experiment''.
% \item VERTEX2014, Macha Lake, Czech Republic, Poster: ``Simulations for the HFT--Pixel detector at the STAR experiment'' (available in the back of this document).
% \item 18th Conference of Czech and Slovak Physicists 2014, Olomouc, Czech Republic, Talk: ``Simulations for the HFT--Pixel detector at the STAR experiment''.
% \item ICPAQGP2015, Kolkata, India, Talk: ``Heavy Flavor Tracker at the STAR Experiment''.
% \item Quark Matter 2015, Kobe, Japan, Poster: ``$\Lambda_\mathrm{c}^+$ baryon production in Au+Au collisions at
% $\sqrt{s_{\mathrm{NN}}}$ = 200 GeV'' (available in the back of this document).
% \item Zimanyi School 2015, Budapest, Hungary, Oral presentation: ``Lambda\_c baryon production in Au+Au collisions at 200 GeV''.
% \item 2016 European School of High-Energy Physics, Skeikampen, Norway, Poster: $\Lambda_\mathrm{c}^+$ baryon production in Au+Au collisions at
% $\sqrt{s_{\mathrm{NN}}}$ = 200 GeV'' (available in the back of this document).
% \item Hot Quarks 2016, Padre Island, Texas, USA, Talk: ``Measurements of open charm hadrons at STAR experiment'' (Proceedings is available in the back of this document).
% \item 2017 Winter Workshop of Experimental Nuclear and Particle Physics at FNSPE, Bily Potok, Czech Republic, Talk: ``\Lambdac\ BARYON RECONSTRUCTION IN HEAVY-ION COLLISIONS'' (Proceedings is available in the back of this document).
% 
% \end{enumerate}

