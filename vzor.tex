%%%%%%%%%%%%%%%%%%%%%%%%%%%%%%%%%%%%%%%%%%%%%%%%%%%%%%%%%%%%
%% Zacatek vzorove strany %%%%%%%%%%%%%%%%%%%%%%%%%%%%%%%%%%
\pagestyle{fancy}
\mbox{}
\newpage

\noindent {\it Title:}\\
{\bf \engtitle}\\

\noindent
{\it Author:}        Ing.\ Miroslav Šimko \\

\noindent
{\it Specialization:}     Nuclear Engineering\\


\noindent {\it Supervisor:} \skolitel \\

\noindent {\it Co-supervisor:} \konzultant\\  
--------------------------------------------------------------------------------------- \\

\noindent {\it Abstract:} \\
\noindent  This works focuses on the analysis of $\Lambda_\mathrm{c}$ which is the lightest baryon containing a charm quark and, as such, presents a unique probe to study
the behavior of charm quarks in the hot and dense QCD medium created in ultra-relativistic heavy-ion collisions. Together with the measurement of the D$^0$ meson,
we can study the various modes of charm quark hadronization in heavy-ion collisions and bring additional insights into the  quark coalescense process in the strongly coupled quark-gluon plasma.
$\Lambda_\mathrm{c}$ baryons have an extremely short lifetime ($c \tau \sim 60\,\upmu$m) which makes the reconstruction experimentally challenging. The novel detector Heavy Flavor Tracker, installed at the STAR experiment between the years 2014--2016, has
shown high efficiency and an unparalleled pointing resolution that can facilitate the $\Lambda_\mathrm{c}$ reconstruction  in heavy-ion collisions. In 2014 and 2016, STAR collected 2.5 billion minimum-bias
Au+Au collisions at $\sqrt{s_\mathrm{NN}} = 200\,$GeV\@.
We describe the first reconstruction of the $\Lambda_\mathrm{c}$ baryons via hadronic decays in Au+Au collisions and, moreover, the first such measurement differentiated in transverse momentum and the number of participants in heavy-ion collisions. The measured yield shows that the \Lambdac\ make a sizeable contribution to the total charm yield in Au+Au collisions. Moreover, the $\Lambda_\mathrm{c}/$D$^0$ yield ratio is significantly larger in Au+Au collisions, compared to simulated p+p collisions, and is consistent with theoretical calculations that include charm-quark coalescence.
\\

\noindent {\it Key words:} \\
\noindent  Lambda\_c, coalescence, STAR, RHIC, HFT

%\null \vfill \noindent{\bf\Large
Acknowledgement}
\bigskip

\noindent I would like to thank everyone in the whole world!

\newpage

 \newpage
%\thispagestyle{empty}
 \noindent
{\it N\' azev pr\' ace:}\\
{\bf \czechtitle}\\

\noindent
{\it Autor:} \myself\\
--------------------------------------------------------------------------------------- \\


\noindent {\it Abstrakt:} \\
% Tato práce se soustředí na analýzu \Lambdac, což je nejlehčí baryon, který obsahuje půvabný (c) kvark, a tak představuje unikátní sondu do silně interagující hmoty, jež se produkuje v ultrarelativistických srážkách těžkých jader.\ Spolu s měřením mezonu D$^0$, může analýza \Lambdac\ přinést nové poznatky o t.zv.\ procesu kvarkové koalescence v silně interagujícím kvark gluonovém plazmatu.\ \Lambdac\ mají nicméně velmi krátkou dobu života ($c \tau \sim 60\,\upmu$m), a tak je jejich rekonstrukce experimentálně poměrně náročná.\ Mezi lety 2014--2016 byl ale nainstalován dráhový detektor Heavy Flavor Tracker (HFT), který má zatím ve světě ojedinělou rozlišovací schopnost, takže umožňuje měření baryonů \Lambdac\ ve srážkách těžkých iontů, a to při energii srážky na jeden nukleon $\sqrt{s_\mathrm{NN}} = 200\,$GeV.\ V této práci je popsáno zrekonstruování \Lambdac\ v hadronovém rozpadovém kanálu za použití těchto dat.\ Dále ukážeme podíl výtěžků mezi \Lambdac\ a D$^0$, a jeho srovnání s teoretickými předpověďmi.\\

\noindent {\it Kl\' i\v cov\' a slova:} \\
Lambda\_c, koalescence, STAR, RHIC, HFT

%% Konec vzorove strany %%%%%%%%%%%%%%%%%%%%%%%%%%%%%%%%%%%%
%%%%%%%%%%%%%%%%%%%%%%%%%%%%%%%%%%%%%%%%%%%%%%%%%%%%%%%%%%%%

%\end{document}
