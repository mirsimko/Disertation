%%%%%%%%%%%%%%%%%%%%%%%%%%%%%%%%%%%%%%%%%%%%%%%%%%%%%%%%%%%%
%% Zacatek vzorove strany %%%%%%%%%%%%%%%%%%%%%%%%%%%%%%%%%%
\pagestyle{fancy}
% \mbox{}
% \newpage

\thispagestyle{empty}
\begin{tabular}{lc}
\toprule
\textbf{Bibliographic Entry} & \\
\midrule
{\it Author:}     &  \textbf{ Ing.\ Miroslav Šimko} \\
 & Czech Technical University in Prague \\
 &  Faculty of Nuclear Sciences and Physical Engineering \\
 & Physics Department \\
 {\it Title of the doctoral thesis:} & {\ \engtitle}\\
{\it Specialization:}   &  Nuclear Engineering\\
{\it Supervisor:} & \textbf{\skolitel} \\
 &  Faculty of Nuclear Sciences and Physical Engineering \\
 & of the Czech Technical University in Prague\\
 {\it Supervisor Specialist:} & \textbf{\konzultant} \\
 & Nuclear Physics Institute of the Czech Academy of Sciences \\
  {\it Academic year:} & 2020/2021 \\
 {\it Number of pages:} & 170 \\
{\it Key words:} & Lambda\_c, coalescence, STAR, RHIC, HFT\\
\bottomrule
\end{tabular}

\newpage
\thispagestyle{empty}
\begin{tabular}{lc}
\toprule
\textbf{Bibliografický záznam} & \\
\midrule
{\it Author:}     &  \textbf{ Ing.\ Miroslav Šimko} \\
 & České vysoké učení technicé v Praze \\
 &  Fakulta jaderná a fyzikálně inženýrská \\
 & Katedra fyziky \\
 {\it Název práce:} & {\ \czechtitle}\\
{\it Obor:}   &  Jaderné inženýrství\\
{\it Školitel:} & \textbf{\skolitel} \\
 &  Fakulta jaderná a fyzikálně inženýrská \\
 & českého vysokého učení technického v Praze\\
 {\it Školitel specialista:} & \textbf{\konzultant} \\
 & Ústav jaderné fyziky akademie věd České republiky, v.v.i. \\
  {\it Akademický rok:} & 2020/2021 \\
 {\it Počet stran:} & 170 \\
{\it Klíčová slova:} & Lambda\_c, koalescence, STAR, RHIC, HFT\\
\bottomrule
\end{tabular}

%--------------------------------------------------------------------------------------- \\

\newpage

\null \vfill \noindent{\bf\Large
English abstract}
\bigskip

\noindent  This work focuses on the analysis of $\Lambda_\mathrm{c}$ which is the lightest baryon containing a charm quark. As such, the \Lambdac\ presents a unique probe to study
the behavior of charm quarks in the hot and dense QCD medium created in ultra-relativistic heavy-ion collisions. Together with the measurement of the D$^0$ meson,
we can study modes of charm quark hadronization and bring additional insights into the  quark coalescense process in the strongly coupled quark-gluon plasma.
$\Lambda_\mathrm{c}$ baryons have an extremely short lifetime ($c \tau \sim 60\,\upmu$m) which makes the reconstruction experimentally challenging. The novel detector Heavy Flavor Tracker, installed at the STAR experiment between the years 2014--2016, has
shown high efficiency and an unparalleled pointing resolution that can facilitate the $\Lambda_\mathrm{c}$ reconstruction  in heavy-ion collisions.
In this thesis, we describe the first reconstruction of the $\Lambda_\mathrm{c}$ baryons via hadronic decays in Au+Au collisions and, moreover, the first such measurement differentiated in transverse momentum and the number of participants in heavy-ion collisions. The measured yield shows that the \Lambdac\ make a sizeable contribution to the total charm yield in Au+Au collisions. Moreover, the $\Lambda_\mathrm{c}/$D$^0$ yield ratio is significantly larger in Au+Au collisions, compared to simulated p+p collisions, and is consistent with theoretical calculations that include charm-quark coalescence.
\\


 \newpage
%\thispagestyle{empty}

%--------------------------------------------------------------------------------------- \\

\newpage

\null \vfill \noindent{\bf\Large
Český abstrakt}
\bigskip

Cílem této práce je analýza baryonu \Lambdac, což je nejlehčí baryon, který obsahuje půvabný (c) kvark. Jako takový představují \Lambdac\ unikátní nástroj pro studování vlastností silně interagující hmoty, jež se produkuje ve srážkách ultrarelativistických těžkých iontů. Spolu s měřením mezonu \dzero\ může analýza \Lambdac\ přinést nové poznatky o procesu hadronizace c kvarku a objasnit, zda je hlavním způsobem pro hadronizaci c kvarku v kvark gluonovém plazmatu t.zv.\ koalescence. \Lambdac\ mají nicméně velice krátkou střední dobu rozpadu ($c \tau \sim 60\,\upmu$m), kvůli čemuž je jejich analýza velice náročná. Mezi lety 2014--2016 byl ale nainstalován přesný dráhový detektor Heavy Flavor Tracker (HFT), který má zatím ve světě ojedinělé rozlišení, takže umožňuje měření baryonů \Lambdac\ i ve srážkách těžkých iontů při energii srážky na jeden nukleon--nukleonový pár $\sqrt{s_\mathrm{NN}} = 200\,$GeV. V této práci popisujeme první přímou rekonstrukci \Lambdac\ v hadronovém rozpadovém kanálu ve srážkách Au+Au. Změřený výtěžek ukazuje, že \Lambdac\ se výrazně podílí na celkovém výtěžku půvabného kvarku ve srážkách Au+Au. Dále je poměr výtěžků $\Lambda_\mathrm{c}/$D$^0$ výrazně vyšší ve srážkách Au+Au než v ekvivalentních simulacích srážek p+p, což je konzistentní výsledek s teoretickými předpověďmi, které obsahují koalescenci půvabného kvarku. 
% Tato práce se soustředí na analýzu \Lambdac, což je nejlehčí baryon, který obsahuje půvabný (c) kvark, a tak představuje unikátní sondu do silně interagující hmoty, jež se produkuje v ultrarelativistických srážkách těžkých jader.\ Spolu s měřením mezonu D$^0$, může analýza \Lambdac\ přinést nové poznatky o t.zv.\ procesu kvarkové koalescence v silně interagujícím kvark gluonovém plazmatu.\ \Lambdac\ mají nicméně velmi krátkou dobu života ($c \tau \sim 60\,\upmu$m), a tak je jejich rekonstrukce experimentálně poměrně náročná.\ Mezi lety 2014--2016 byl ale nainstalován dráhový detektor Heavy Flavor Tracker (HFT), který má zatím ve světě ojedinělou rozlišovací schopnost, takže umožňuje měření baryonů \Lambdac\ ve srážkách těžkých iontů, a to při energii srážky na jeden nukleon $\sqrt{s_\mathrm{NN}} = 200\,$GeV.\ V této práci je popsáno zrekonstruování \Lambdac\ v hadronovém rozpadovém kanálu za použití těchto dat.\ Dále ukážeme podíl výtěžků mezi \Lambdac\ a D$^0$, a jeho srovnání s teoretickými předpověďmi.\\\Lambdac\


%% Konec vzorove strany %%%%%%%%%%%%%%%%%%%%%%%%%%%%%%%%%%%%
%%%%%%%%%%%%%%%%%%%%%%%%%%%%%%%%%%%%%%%%%%%%%%%%%%%%%%%%%%%%





