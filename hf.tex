Open-heavy-flavor hadrons are ones that have a non-zero charm content, meaning that the number of valence-anti-charm quarks is different from the number of valence-charm quarks. In this chapter, we deal with the production of open charm in heavy-ion collisions at RHIC and, in the end, we summarize the current knowledge about the production of the charmed \Lambdac\ baryon in e+e, e+p, p+p, p+A, and A+A collisions.

\section{Heavy-flavor production in heavy-ion collisions}

The heavy quarks such as c or b have much higher masses ($m_\mathrm{c} \simeq 1.25\,$GeV$/c^2$ and $m_\mathrm{b} \simeq 4.5\,$GeV$/c^2$~\cite{PDG}) as compared to the light quarks (u, d, and s) which, together with gluons, make most of the QGP bulk.  During most of the expansion (after $\sim 0.1\,$fm$/c$ for the c-quarks at top-RHIC energies) the thermal energy of the system is too low to create the heavy quarks~\cite{summaryHF,Prino_Rapp_HF}\@. As a result, they can be only produced in hard processes during the very early stages of the collision. Due to their relatively long life times, as compared to the thermally interacting medium, they can experience the whole evolution of the system and can act as excellent probes into the processes of energy loss in the sQGP\@. Moreover, since their masses are much higher than $T_\mathrm{ch}$, the heavy quarks retain their identity and can serve as ideal probes into the hadronization process of the medium, e.g.\ determine whether lighter quarks are picked up by the heavy quarks or they fragment independently.

The typical momentum exchange of the heavy particles (both quarks and hadrons) is relatively small, compared to the thermal momentum $p^2_\mathrm{Q,th} \simeq 2m_\mathrm{Q}T$~\cite{Prino_Rapp_HF} where $m_\mathrm{Q}$ is the mass of the particle and $T$ is the temperature of the system\@. The thermal relaxation 
time of heavy particles  $\tau_\mathrm{Q} \simeq \tau_\mathrm{th}m_\mathrm{Q}/T$ is also much longer than in the bulk medium $\tau_\mathrm{th}$\@. The heavy particles, therefore, move akin to Brownian particles in the expanding medium with many small kicks from the bulk particles.


\begin{figure}[!htb]
\begin{center}
 \includegraphics[width=0.7\linewidth]{img/Diffusion_coefficient}
\caption[Charm-quark spatial diffusion coefficient $D_s$.]{\label{diffusion}Charm-quark spatial diffusion coefficient $D_s$ multiplied by $2\pi T$ at the c-quark momentum limit of $p = 0$, calculated in lQCD (black circles and squares~\cite{BanerjeeLattice,DingLattice}) and compared to models with different elastic interactions of the charm quark. The green and red bands denote a T-matrix approach with heavy-quark+gluon and heavy-quark+light-quark interactions. The bands denote limit cases with the free-quark potential (F-potential) and internal-quark potential (U-potential), calculated in lQCD~\cite{Tmatrix}\@. The dashed-dotted line denotes a perturbative-QCD (pQCD) approach with strong-interaction constant set as $\alpha_\mathrm{s} = 0.4$\@. The dashed line denotes the diffusion of D-mesons in a hadron-resonance gas~\cite{DmesonHRG}\@. Figure taken from~\cite{summaryHF}.}
\end{center}
\end{figure}


\nomenclature{pQCD}{Perturbative ChromoDynamics}\nomenclature{HRG}{Hadron-Resonance Gas}Phenomenology can greatly benefit from the measurements of the heavy-flavor particles also because several variables that describe the behavior of the heavy quarks in QGP can be calculated in lQCD\@.  One of these variables is the spatial diffusion coefficient \nomenclature{$D_s$}{Spatial diffusion coefficient}$D_s$~\cite{BanerjeeLattice,DingLattice}, defined by the average displacement squared
\begin{equation}
 \langle r^2 \rangle = (2d) D_s t
\end{equation}
where $t$ denotes time and the factor $d$ is the number of spatial dimensions. A small value of the $D_s$ characterizes strong coupling (therefore frequent rescattering of the particle) with the medium. The thermal relaxation $\tau_\mathrm{Q}$ is directly related to the $D_s$~\cite{Prino_Rapp_HF}
\begin{equation}
 \tau_\mathrm{Q} = \frac{m_\mathrm{Q}}{T} D_s\,.
\end{equation}
The delay in thermal relaxation of the heavy quarks is, therefore, proportional to $m_\mathrm{Q}/T$\@. This relation suggests that $D_s$ is a general medium property and when scaled by the thermal wavelength of the medium $\lambda_\mathrm{th} = 1/(2 \pi T)$, one receives a dimensionless property of the medium that has been suggested to be related to the ratio of sheer viscosity to the entropy density~$\eta/s$~\cite{heavyThermalizationAndEtaOverS,QGP4}
\begin{equation}
D_s(2\pi T) \propto \frac{\eta}{s} (4\pi)\,.
\end{equation}

Figure~\ref{diffusion} shows theoretical calculations of $D_s(2\pi T)$ from the first principles (i.e.\ lQCD~\cite{BanerjeeLattice,DingLattice}) and estimates, employing pQCD and the T-matrix in the quantum field theory~\cite{Tmatrix}\@. The temperature scale is prolonged bellow the critical temperature $T_\mathrm{c}$ and the diffusion of the D-mesons in the hadron-resonance gas is considered from the calculation~\cite{DmesonHRG}, because the c-quarks are likely hadronized in a medium with such temperature.

$D_s$ cannot be  measured experimentally directly. However with the increasing precision of the data, phenomenological models can make relatively accurate estimates, when comparing the calculations to the experimental data, e.g.\ the nuclear-modification factor $R_\mathrm{AA}$ or the elliptic flow coefficient $v_2$ of the D-mesons with one c-quark\@.


\nocite{Langevin,LangevinTranslation}








\section{Baryon enhancement, baryon to anti-baryon ratio, and coalescence of the \Lambdac\ baryon}

An enhancement of strange baryons compared to mesons has been observed in the intermediate transverse momentum (\pt) range in central heavy-ion collisions at RHIC~\cite{StrangenessEnhancementSTAR} and the LHC~\cite{LambdaALICE}\@. This phenomenon is known as strange baryon enhancement and is believed to be one of the key pieces of evidence of the existence of the sQGP\@. This behavior can be explained via hadronization models which include quark coalescence~\cite{coalescenceKrakow,coalescenceKFKI}, which is a process in which the quarks are combined to form hadrons, as compared to the quark fragmentation, in which new quarks are created from the vacuum which is the process that governs the hadronization in p+p and e+e collisions.  

\begin{figure}[htb]
\centering
\includegraphics[width=0.8\linewidth]{img/coalescencePic}
\caption[Baryon to meson ratio in RHIC Au+Au collisions.]{\label{fig:LambdaKzero}Baryon to meson ratio in RHIC Au+Au collisions with the center of mass energy per nucleon $\sqrt{s_\mathrm{NN}} = 200\,$GeV vs transverse momentum ($p_\mathrm{T}$)~\cite{GuannanLc}\@. Left: Ratio of the invariant yields of p and $\overline{\mathrm{p}}$ over $\uppi^+$ and $\uppi^-$ at STAR for the centralities 0--12$\,\%$ and 60--80$\,\%$~\cite{StrangenessEnhancementSTAR}. Middle: ratio of the yields of $\Lambda$ over K$^0_\mathrm{s}$ at STAR for central (0--5$\,\%$) and peripheral (60--80$\,\%$) collisions. Right: Models of ratios of $\Lambda_\mathrm{c}$ over D$^0$~\cite{ShaoSong, Ghosh_Lc_rescattering, SHM}.}

\end{figure}

The baryon enhancement observed at RHIC is demonstrated in Figure~\ref{fig:LambdaKzero}, in which the left-hand-side panel shows the ratio of the yield of p and $\overline{\mathrm{p}}$ to $\uppi^+$ and $\uppi^-$, and the middle panel is a plot of the ratio of $\Lambda^+$ and $\overline{\Lambda}^-$ to 2-times the yield of K$^0_\mathrm{s}$, which both contain a strange quark. An enhancement in the $p_\mathrm{T}$ region of $\sim$2--4$\,$GeV$/c$ is clearly observed in the case of the $\Lambda$ baryons. 

An interesting question (and one of the main topics of this document) is whether the charm baryons follow the same pattern as the strange ones. As the quark compositions of the \Lambdac\ and D$^0$ are cud and c$\overline{\mathrm{u}}$, respectively, we can draw a direct comparison to the measurement of $\Lambda$ and~K$^0$.

The plot in the right-hand-side panel of Figure~\ref{fig:LambdaKzero} shows theoretical estimates of the ratio of the yields of $\Lambda_\mathrm{c}$ to D$^0$. The scenario with no coalescence is demonstrated by the green line which was produced using the PYTHIA simulator~\cite{PYTHIA}. The dashed lines (Ko) show two coalescence models~\cite{ShaoSong}: One where the  quarks coalesce as the charm quark with a light di-quark structure and one where all three quarks coalesce. No rescattering in the hadron gas is considered in these two models. The darker gray band (Greco) indicates a model with three-quark coalescence calculated in the framework described in~\cite{Greco_framework} with the results from~\cite{Greco_results}, then the $\Lambda_\mathrm{c}$ and D meson diffusion is calculated, using an effective T--matrix approach~\cite{Ghosh_Lc_rescattering}. Note that the denominator for this band is the sum of the yields of all D mesons (D$^\pm$, D$^0$, and~$\mathrm{\overline{D^0}}$). The light gray rectangle (SHM --- Scattering with Hadronic Matter) is a model~\cite{SHM} with coalescence of di-quark and the c-quark. This model uses $\Lambda_\mathrm{c}$ diffusion in the hadronic matter, in which the $\Lambda_\mathrm{c}$ is allowed to change into other hadron species when scattering on other hadrons.

Another key signiture of coalescence process is the ratio between the yields of anti-baryons. Figure~\ref{fig:BtoAntiB} shows such ratios for different experiments. At STAR, we can observe that with increasing strangeness content, the ratio gets closer to unity. This is because the u and d quarks are more abundant, compared to $\overline{\mathrm{u}}$ and $\overline{\mathrm{d}}$, due to the non-zero baryon chemical potential. 

\begin{figure}
\centering
\includegraphics[width=0.6\textwidth]{img/baryonToAntibaryon}
\caption[Ratio of the yields of baryons compared to anti-baryons measured at SPS and~RHIC\@.]{\label{fig:BtoAntiB}Ratio of the yields of baryons compared to anti-baryons measured at SPS and RHIC\@. The full symbols represent full maximum RHIC energy $\snn = \SI{200}{GeV}$ and the empty symbols represent experiments at SPS at CERN with the collision energy per nucleon $\snn = \SI{17}{GeV}$. Taken from~\cite{BtoAntiB}.}

\end{figure}

\section{Charm production in p+p collisions}

\begin{figure}[!htb]
\centering
\includegraphics[width=0.5\textwidth]{img/inclusiveC}
\caption[Charm quark pair production cross-section at mid-rapidity vs.\ \pt.]{Charm quark pair production cross-section~\cite{pp500GeVcharm} at mid-rapidity vs.\ \pt~\cite{FONLLcharm,GuannanQM15,ALICEcharm,CDFcharm,inclusiveC}.}
\label{fig:inclusiveC}
\end{figure}

Figure~\ref{fig:inclusiveC} shows the inclusive cross-section of c--$\overline{\mathrm{c}}$ pairs in p+p collisions at mid-rapidity at RHIC, Tevatron, and the LHC~\cite{GuannanQM15,ALICEcharm,CDFcharm,inclusiveC,pp500GeVcharm}\@. At STAR, the total cross-section has been assessed via measurement of the cross-section of D$^0$ and D* while assuming that the fragmentation ratios of these mesons is known in p+p collisions at RHIC energies. We can see that throughout the experiments and collision energy $\sqrt{s}$, the data are consistent with the theory calculation using FONLL~\cite{FONLLcharm}, albeit the data are consistently on the upper limit of FONLL\@. This may be caused by a non-perturbative contribution to the total cross-section. If the charm quark production is not significantly altered by the nuclear Parton-Distribution Functions (nPDF\nomenclature{nPDF}{Nuclear Parton-Distribution Function}\nomenclature{PDF}{Nuclear Parton-Distribution Function}), we can assume that the number of charm quarks scales with the number of binary collisions ($N_\mathrm{coll}$\nomenclature{$N_\mathrm{coll}$}{Number of binary collisions}), because charm quarks are created in the early stages in heavy-ion collisions. Therefore, by measuring the charm hadrons, we have calibrated probes into the interaction of quark-gluon plasma.

% \section{Charm production in Au+Au chere ollisions}
% 
% Figure~\ref{fig:nCollScaling} shows the inclusive production of the c-quark measured at STAR~\cite{inclusiveC}.
% It exhibits scaling with the
% number of binary nucleon--nucleon collisions ($N_\mathrm{coll}$) in Au+Au
% collisions. The D$^0$ meson, being the lightest hadron
% that contains a charm quark, provides, therefore, an excellent calibrated probe
% to the behavior of the medium. Recent measurements of the D$^0$ meson 
% at Relativistic Heavy-Ion Collider (RHIC)
% and the Large Hadron Collider (LHC)~\cite{AuAuD0, DRaaALICE, Dv2ALICE} show suppression of yields at
% high transverse momenta ($p_\mathrm{T}$) and suggest a non-zero $v_2$
% at intermediate to high $p_\mathrm{T}$\@. Measurements of better precision are, however, needed
% to provide more stringent constraints on model calculations.
% 
% \begin{figure}[htb]
% \centering
% \includegraphics[width=0.7\textwidth]{img/inclusiveCinAuAu}
% \caption{Charm cross-section at mid-rapidity from p+p to central Au+Au collisions~\cite{GuannanQM15,dAuD0,AuAuD0,inclusiveC,totalCharmCrossSection}.}
% \label{fig:nCollScaling}
% \end{figure}
% 
% The D$_\mathrm{s}$ meson, which contains a strange quark,
% provides an additional handle on the ha\-dro\-ni\-za\-tion process of charm quarks. Recent 
% calculations~\cite{DsTAMU}
% suggest an enhancement of the D$_\mathrm{s}$ meson yield compared to the D$^0$ meson because 
% of the process of quark coalescence.


\section{D$^0$ and D$^\pm$ measurements}
In p+p collisions, non-strange D-mesons are the most abundant hadrons including charm quarks. This makes them excellent tools for measuring the properties of the charm quarks. 

\subsection{\dzero\ \Raa\ and \Rcp\ nuclear modification factors}
\begin{figure}[!htb]
\begin{center}
 \includegraphics[width=0.5\textwidth]{img/D0_RAA}\\
\end{center}
\caption[D$^0$ $R_\mathrm{AA}$ measured in 0--10$\,\%$, 10--20$\,\%$, and 20--40$\,\%$ central Au+Au collisions.]{\label{dzeroRAA}D$^0$ $R_\mathrm{AA}$~\cite{D0paper} measured in 0--10$\,\%$~(a), 10--20$\,\%$~(b), and 20--40$\,\%$~(c) central Au+Au collisions at $\sqrt{s_\mathrm{NN}} = \SI{200}{\giga\electronvolt}$ as a function of $p_\mathrm{T}$\@. The
light and dark green vertical bands around unity are uncertainties related
to the $N_\mathrm{coll}$ in Au+Au collisions and the global normalization in the p+p collisions,
respectively~\cite{AuAuD0}.}
\end{figure}
Even if the inclusive charm quark production scales with $N_\mathrm{coll}$,
the D$^0$ spectrum shape can be significantly modified in Au+Au collisions. Fig.~\ref{dzeroRAA}
shows the D$^0$ nuclear modification factor
$R_\mathrm{AA}$
as a function of $p_\mathrm{T}$ in the several centrality regions of the Au+Au collisions.
The new results
(black circles)~\cite{D0paper} obtained with the HFT (see Section~\ref{HFTsection}) are consistent with the published
$R_\mathrm{AA}$ from 1.1$\,$B minimum-bias events taken in the years 2010 and 2011 
without the HFT (blue empty diamonds)~\cite{AuAuD0}\@.
For the new results, a much better precision is achieved despite the less statistics used. 
Compared to p+p collisions, the D$^0$ production is significantly suppressed at low and high $p_\mathrm{T}$ in central collisions which indicates
strong interactions between charm quarks and the medium.

\begin{figure}[!htb]
\begin{center}
 \includegraphics[width=0.5\textwidth]{img/D0_Rcp22}\\
\end{center}
\caption[D$^0$ \Rcp.]{\label{dzeroRcp}D$^0$ \Rcp~\cite{D0paper}, with the 40--60$\,\%$ centrality spectrum as a reference, measured in 0--10$\,\%$~(a), 10--20$\,\%$~(b), and 20--40$\,\%$~(c) central Au+Au collisions at $\sqrt{s_\mathrm{NN}} = \SI{200}{\giga\electronvolt}$ as a function of \pt\@. The
light and dark green vertical bands around unity are uncertainties related
to the $N_\mathrm{coll}$ in Au+Au collisions and the global normalization in the p+p collisions,
respectively. The gray bands around unity depict the systematic uncertainty due to corrections on the uncertainty of the position of the primary vertex. The measurement is compared to model calculations~\cite{Duke, Duke2015, LBT, LBTprivate}.}
\end{figure}


STAR has recently published~\cite{D0paper} a high-precision measurement of the D$^0$  nuclear-modification factor \Rcp\ (40--60$\,\%$)\@. This measurement greatly benefits from the precision of the HFT, since it can be used in both central and peripheral collisions.
The D$^0$ \Rcp\ measurement is shown in Figure~\ref{dzeroRcp}. In the high-\pt\ region ($\pt \gtrsim 3\,$GeV), the \dzero\ production is increasingly suppressed with centrality. The \Rcp\ is compared to two models, including charm-quark diffusion. The Duke model~\cite{Duke, Duke2015} uses Langevin calculation of the charm-quark diffusion. Both radiative and collisional energy loss processes are included in this model. The hadronization of the charm quark is calculated in a hybrid approach that combines both fragmentation and coalescence. In the LBT (Linearized Boltzmann Transport) model~\cite{LBT, LBTprivate} a jet-transport framework is extended to include heavy quarks. The hadronization is handled in the same hybrid approach as in the Duke model. These models also agree with the \dzero\ $v_2$ measurement shown in the next section.


\begin{figure}[!htb]
% novejsi
\begin{center}
 \includegraphics[width=0.6\textwidth]{img/DpmLatest}\\
 \includegraphics[width=0.6\textwidth]{img/DpmLatest10-40}\\
\end{center}
\caption[D$^\pm$ meson \Raa\ in 0--10$\,\%$ and 10--40$\,\%$ central Au+Au collisions.]{\label{dpm} D$^\pm$ meson \Raa\ in 0--10$\,\%$ (top) and 10--40$\,\%$ (bottom) central Au+Au collisions. The gray bands depict the D$^0$ uncertainty in p+p collisions~\cite{flashTalkICHEPVanek,D0paper}. The dark blue (yellow) band refers to the global uncertainty in Au+Au (p+p) collisions. Taken from~\cite{flashTalkICHEPVanek}\@.}
\end{figure}


As an important crosscheck, STAR has measured the D$^\pm$ meson production as well. Figure~\ref{dpm} shows the nuclear modification factor \Raa\ in Au+Au collisions, recorded in 2014 and 2016~\cite{flashTalkICHEPVanek}\@. The D$^\pm$ production is consistent with the \dzero~\cite{D0paper} in the whole \pt\ range in both, 0--10$\,\%$ and 10--40$\,\%$ centrality brackets.

\subsection{\dzero\ azimuthal anisotropy, $v_2$, $v_3$, and the charm diffusion coefficient}

Another way of probing the interactions of the charm quark with the sQGP medium is measuring the distribution of charmed hadrons in the azimuthal angle by measuring the Fourier terms $v_n$~\cite{StrangeAndChargedv2paper}\@. 
For light hadrons. The change in pressure normally drives the $v_n$ coefficients as they move from areas with large pressure into areas with lower. Charm quarks are, however, created in hard collisions during the early stages of the evolution of the medium and because their masses are much higher than the QCD scale $m_\mathrm{c} \gg \Lambda_\text{QCD}$ of the medium properties, their masses are not affected by the QCD medium. This makes the c-quarks compelling probes to the behavior of the medium as they are carried away in a similar fashion to Brownian particles in liquids or gases. 

\begin{figure}[!htb]
\begin{center}
 \includegraphics[width=0.6\textwidth]{img/D0v2_080_withModels-eps-converted-to}\\
\end{center}
\caption[D$^0$ $v_2$ as a function of $p_\mathrm{T}$.]{\label{dzerov2}D$^0$ $v_2$ as a function of $p_\mathrm{T}$ for Au+Au collisions
at $\sqrt{s_\mathrm{NN}} = \SI{200}{\giga\electronvolt}$~\cite{D0v2paper}, compared to theoretical calculations~\cite{PHSD2014,LBT,LBTprivate,TAMU,PHSD2015,SUBATECHvn,SUBATECHquenching,Duke,Duke2015,Hydro2012,Hydro2015}.}
\end{figure}


The HFT enables the measurement of the D$^0$ $v_2$ for the first 
time at RHIC, as shown in Figure~\ref{dzerov2NCQ}\@. The $v_2$ values were calculated from 0--80$\,\%$ central Au+Au collisions at $\snn = 200\,$GeV recorded in 2014 and 2016. The vertical bars (brackets) indicate 
the statistical (systematic)
uncertainties while the gray bands represent the estimated non-flow contribution inferred
from D meson--hadron correlations in p+p collisions. The data show that the $v_2$
is significantly larger than 0 above $\SI{1.5}{\giga\electronvolt}/c$\@.

Several models~\cite{Hydro2012,Hydro2015,PHSD2014,LBT,LBTprivate,TAMU,PHSD2015,SUBATECHvn,SUBATECHquenching,Duke,Duke2015} are compared to the
measurements of $R_\mathrm{AA}$ in Figure~\ref{dzeroRAA} and $v_2$ in 0--80$\,\%$ centrality 
in Figure~\ref{dzerov2}\@. The 3D viscous hydrodynamic calculation with AMPT\nomenclature{AMPT}{A Multi-Phase Transport} 
initial conditions, tuned to the light-hadrons $v_2$~\cite{Hydro2012,Hydro2015}, describes the D$^0$ $v_2$ data well 
which points to the thermal equilibrium between the c-quarks and the medium.

\begin{figure}[!htb]
\begin{center}
  \includegraphics[width=0.6\textwidth]{img/D0v2_1040_Data_Light-eps-converted-to}\\
\end{center}
\caption[D$^0$ $v_2/n_\mathrm{q}$ as a function of $(m_\mathrm{T} - m_0)/n_\mathrm{q}$ for Au+Au collisions.]{\label{dzerov2NCQ}D$^0$ $v_2/n_\mathrm{q}$, where $n_\mathrm{q}$ is the number of valence quarks, as a function of $(m_\mathrm{T} - m_0)/n_\mathrm{q}$ for Au+Au collisions
at $\sqrt{s_\mathrm{NN}} = \SI{200}{\giga\electronvolt}$ with centrality of 10--40$\,\%$ compared to strange hadrons $v_2/n_\mathrm{q}$~\cite{D0v2paper,StrangeAndChargedv2paper}.}
\end{figure}


Another way of parametrizing the $v_2$ is using the number of valence quarks $n_\mathrm{q}$ and the transverse mass $m_\mathrm{T}= \sqrt{\pt^2 - m_0^2}$, where $m_0$ is the rest mass of the particle. Figure~\ref{dzerov2NCQ} shows $v_2 / n_\mathrm{q}$ of the \dzero~\cite{D0v2paper} and several lighter hadrons~\cite{StrangeAndChargedv2paper} plotted versus $(m_\mathrm{T} - m_0)/n_\mathrm{q}$ in 10--40$\,\%$ central Au+Au collisions\@. The $v_2/n_\mathrm{q}$ of the light-flavor hadrons follows the same pattern with relatively high precision. This phenomenon is called the Number-Of-Constituent-Quarks (NCQ) scaling. Remarkably, as shown in this Figure, the D$^0$s follow the NCQ scaling of the light-flavor hadrons at the RHIC top energy. This suggests that the charm quarks from D$^0$ were close to thermal equilibrium with the sQGP medium.

% \begin{figure}[!htb]
% \begin{center}
%   \includegraphics[height=5.5cm]{img/D0_triangular_NCQ}
%   \includegraphics[height=5.6cm]{img/D0_triangular_model}
% \end{center}
% \caption{\label{dzerov3NCQ}D$^0$ $v_3/n_\mathrm{q}$, .}
% \end{figure}
% 
% The triangular flow $v_3$ can be measured at STAR thanks to the HFT as well. Figure~\ref{dzerov3NCQ}


Models that incorporate diffusion of the charm quark with various values of the diffusion coefficient multiplied by 
the temperature $2\pi T D_s$~\cite{PHSD2014,LBT, LBTprivate, TAMU,PHSD2015,SUBATECHvn,SUBATECHquenching,Duke,Duke2015} were compared to the \Raa,  $v_2$, and $v_3$ data. The PHSD\nomenclature{PHSD}{Parton-Hadron-String Dynamics} 
(Parton-Hadron-String Dynamics --- pink dash-double-dot line
in Figure~\ref{dzerov2}) model~\cite{PHSD2014,PHSD2015} uses the dynamical quasiparticle model to calculate 
an effective potential of the charm quark to the medium. It describes the data with $2\pi T D_s \sim$5--12.
It is consistent with the data with the predicted transport coefficient of $2\pi T D_s \sim$3--6.
The group from TAMU\nomenclature{TAMU}{Texas A\&M University}~\cite{TAMU} (blue in~\ref{dzeroRAA} and full green lines in~\ref{dzerov2}) employs
a non-perturbative T-matrix approach with the assumption
that two-body interactions can be described by a potential, which is a function of the transferred
4-momentum. Two scenarios of this model are plotted in Figure~\ref{dzerov2}: one with no c-quark diffusion (lighter green) 
and one where the c-quark diffuses (darker green). The data clearly prefer the latter scenario in which
% The comparison between STAR measurements and this model strongly favors a scenario where the
c-quarks flow.
This model predicts the charm quark diffusion coefficient multiplied by temperature as
$3 \leq 2\pi T D_s \lesssim 11$\@.
The SUBATECH group~\cite{SUBATECHvn}\nocite{SUBATECHquenching} (green in~\ref{dzeroRAA} and dashed red in~\ref{dzerov2}) uses a pQCD approach 
with the Hard Thermal Loop (HTL\nomenclature{HTL}{Hard Thermal Loop approximation}) approximation for soft collisions.
In this approach, the diffusion coefficient is within $2 \leq 2\pi T D_s \leq 4$\@.
The model by the Duke university group~\cite{Duke,Duke2015} uses
$2\pi T D_s$ as a free parameter. The full red curve shown in Figure~\ref{dzeroRAA} and the cyan dot-dash curve from~\ref{dzerov2}b use
the value $2\pi T D_s = 7$ which is fixed to match the D$^0$ $R_\mathrm{AA}$ measured at
the LHC\@. The Duke model can describe the shape of $R_\mathrm{AA}$ well, however it systematically
underestimates the $v_2$\@. The other two models are consistent with 
both $R_\mathrm{AA}$ and $v_2$ data. The inferred estimates of the diffusion coefficient of $2 \lesssim 2 \pi D_s < 12$ are 
consistent with the lattice QCD calculations~\cite{DingLattice,BanerjeeLattice}\nocite{continuumEstimate,Tmatrix,SvetinskyDiffusion} 
in the sensitive temperature range, corresponding to $\sqrt{s_\mathrm{NN}} = \SI{200}{\giga\electronvolt}$ (between $\sim$1--2~$T_\mathrm{c}$, 
where $T_\mathrm{c}$ is the temperature of the critical point). 

\section{Measurement of the $\overline{\text{D}^0}/$D$^0$ ratio}

\begin{figure}[!htb]
\begin{center}
 \includegraphics[width=0.5\textwidth]{img/D0_spectra_ratioposneg_fit}\\
\end{center}
\caption[$\overline{\dzero}/\dzero$ invariant yield ratio at mid-rapidity ($|y| < 1$).]{\label{dzeroRatio}$\overline{\dzero}/\dzero$ invariant yield ratio at mid-rapidity ($|y| < 1$). The dashed lines illustrate constant function fits to the $\overline{\dzero}/\dzero$ ratios~\cite{D0paper}.}
\end{figure}

So far, we have discussed combined measurements of both D$^0$ and its antiparticle $\overline{\text{D}^0}$\@. The charm quarks are produced in pairs, therefore, we would expect the same number of D$^0$ and $\overline{\text{D}^0}$\@. The Statistical-Hadronization Model~\cite{SHM_LcRatio}, however, suggests that the $\overline{\Lambdac^-}/\Lambdac^+$ is lower than unity and the c-quarks can get, therefore, depleted, compared to $\overline{\text{c}}$-quarks. Figure~\ref{dzeroRatio} shows the $\overline{\text{D}^0}/$D$^0$ invariant-yields ratio plotted versus \pt, together with constant function fits of the data~\cite{D0paper}\@.  Although the ratio is at 1 when divided into \pt\ bins, the overall ratio is significantly larger than unity in the 0--60$\,\%$ most central collisions. E.g.\ in the 0--10$\,\%$ most central collisions, the fit is $4.9\,\sigma$ higher than unity.



\section{\Ds\ measurements}

Thanks to the HFT, the \Ds\ meson, consisting of a charm quark and a strange quark,
is measured for the first time at RHIC~\cite{DsPaper}\@. Such measurements can
shed more light on the mechanism of the charm quark coalescence.

\begin{figure}[!htb]
\begin{center}
  \includegraphics[width=0.6\textwidth]{img/DsD0_ALICE.pdf}\\
\end{center} 
\caption[The ratio between \Ds\ and D$^0$ yield in
Au+Au collisions, compared to p+p and Pb+Pb collisions.]{\label{Ds}(a) \Ds/D$^0$ yield ratio, measured in 0--10$\,\%$ (full circles), 10--40$\,\%$ (empty circles), and 40--80$\,\%$ (triangles) central Au+Au
collisions at $\sqrt{s_\mathrm{NN}} = \SI{200}{\giga\electronvolt}$\@. (b) \Ds/D$^0$ ratio in 0--10$\,\%$ most central collisions, measured by STAR, compared to 0--10$\,\%$ central Pb+Pb collisions (empty circles)~\cite{AliceDs2018} and p+p collisions (brown triangles)~\cite{AliceDsPP} measured by ALICE\@. The magenta and green curves denote PYTHIA simulations~\cite{PYTHIA} tuned to p+p collisions at $\sqrt{s} = 7\,$TeV and 200$\,$GeV, respectively\@. Taken from~\cite{DsPaper}\@.}
% \vspace{-6px}
\end{figure}

In Figure~\ref{Ds}, the yield ratio of
\Ds/D$^0$ measured by STAR is shown as a function of \pt~\cite{DsPaper}\@. The D$^0$ spectrum is obtained from the published STAR 
data~\cite{D0paper}\nocite{DavidThesis}\@. No significant change of the ratio with centrality is observed. Moreover, the measured \Ds/D$^0$
ratio is consistent within uncertainties with a similar measurement from minimum-bias Pb+Pb collisions  measured by ALICE~\cite{AliceDs2018}\@. 
To compare our measurement to the \Ds/D$^0$ in p+p collisions, 
PYTHIA 6.4~\cite{PYTHIA} is used, tuned to \snnFull\ (green curve) and $\snn = 7\,$TeV (magenta curve)\@. The PYTHIA simulations are consistent with the ALICE measurement~\cite{AliceDsPP} of the \Ds/D$^0$ ratio in p+p collisions (small brown circles)\@. The STAR measurement is significantly
enhanced compared to the p+p ratio in all the centrality brackets. 

\begin{figure}[!htb]
\begin{center}
  \includegraphics[width=0.6\textwidth]{img/DsD0_Model_new.pdf}\\
\end{center} 
\caption[The \Ds/D$^0$ yield ratio in
Au+Au collisions, compared to theoretical calculations.]{\label{DsModels}Top: The \Ds/D$^0$ yield ratio in
0--10$\,\%$ (full circles) and 10--20$\,\%$ (empty circles) central Au+Au collisions at $\sqrt{s_\mathrm{NN}} = \SI{200}{\giga\electronvolt}$, compared to theoretical predictions, calculated in central collisions (a), more peripheral collisions (b), and PYTHIA simmulation~\cite{PYTHIA} in p+p collisions at $\snn = \SI{200}{\giga\electronvolt}$ (green curve)\@. Taken from~\cite{DsPaper}\@.}
% \vspace{-6px}
\end{figure}

In Figure~\ref{DsModels}, we compare the \Ds/D$^0$ yield ratio, as a function of \pt, to several theoretical predictions that incorporate quark coalescence in their calculations. The Tsinghua model~\cite{Tsinghua} stands out as the only one that includes sequential coalescence in which the \Ds\ hadronize earlier than D$^0$, denoted as Tsinghua (seq.\ coal.)\@. This model does not count with any hadronization through fragmentation, only coalescence. The Catania model~\cite{PlumariGreco} can compare scenarios with only coalescence hadronization -- Catania (coal.) -- and with both, fragmentation and coalescence -- Catania (coal.+frag.)\@. The He, Rapp~\cite{RappLc} calculation incorporates in itself resonance recombination model that conserves the number of heavy quarks, energy, and momentum.  This model includes both, fragmentation and coalescence hadronization scenarios. The Cao,Ko model~\cite{CaoKoDs} also contains fragmentation and coalescence, and energy conservation. 

In the $\pt > 4\,$GeV$/c$ region, the measured \Ds/D$^0$ yield ratio in 0--10$\,\%$ most central collisions features the same general enhancement, compared to the PYTHIA p+p simulation, as models that include quark coalescence Tsinghua, Catania (coal.), He,Rapp, and Cao,Ko. There is, however, some tension in the lower-\pt\ region. On the other hand, the Catania (coal.+frag.) model describes the data in the low-\pt\ region, but fails to predict the high-\pt\ data points. The Tsinghua calculation for 20--40$\,\%$ centrality is close to the measured data as shown in Figure~\ref{DsModels}(b)\@. Overall, this comparison shows that charm-quark coalescence in the surrounding QGP plays a significant role in the hadronization process.
