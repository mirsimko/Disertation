Open-heavy-flavor hadrons are hadrons with non-zero charm, meaning that the number of valence-anti-charm quarks is different from the number of valence-charm quarks. In this chapter, we deal with the production of open charm in heavy-ion collisions at RHIC and, in the end, we summarize the current knowledge about the production of the charmed baryon \Lambdac\ baryon in e+e, e+p, p+p, p+A, and A+A collisions.

\section{Charm production in p+p collisions}

\begin{figure}[!htb]
\centering
\includegraphics[width=0.5\textwidth]{img/inclusiveC}
\caption{Charm quark pair production cross-section at mid-rapidity vs.\ \pt~\cite{FONLLcharm,GuannanQM15,ALICEcharm,CDFcharm,pp500GeVcharm,pp200GeVcharm}.}
\label{fig:inclusiveC}
\end{figure}

Figure~\ref{fig:inclusiveC} shows the inclusive cross-section of c--$\overline{\mathrm{c}}$ pairs in p+p collisions at mid-rapidity at RHIC, Tevatron, and the LHC~\cite{GuannanQM15,ALICEcharm,CDFcharm,pp500GeVcharm,pp200GeVcharm}\@. At STAR, the total cross-section have been assessed via measurement of the cross-section of D$^0$ and D* while assuming that the fragmentation ratios of these mesons is known in p+p collisions at RHIC energies. We can see that throughout the experiments and collision energy $\sqrt{s}$, the data are consistent with the theory calculation using FONLL~\cite{FONLLcharm}, albeit the data are consistently on the upper limit of FONLL\@. This may be caused by a non-perturbative contribution to the total cross-section. If the charm quark production is not significantly altered by the nuclear Parton-Distribution Functions (nPDF\nomenclature{nPDF}{Nuclear Parton-Distribution Function}\nomenclature{PDF}{Nuclear Parton-Distribution Function}), we can assume that the number of charm quarks scales with the number of binary collisions ($N_\mathrm{coll}$\nomenclature{$N_\mathrm{coll}$}{Number of binary collisions}), because charm quarks are created in the early stages in heavy-ion collisions. Therefore, by measuring the charm hadrons, we have calibrated probes into the interaction of quark-gluon plasma.

% \section{Charm production in Au+Au chere ollisions}
% 
% Figure~\ref{fig:nCollScaling} shows the inclusive production of the c-quark measured at STAR~\cite{pp200GeVcharm}.
% It exhibits scaling with the
% number of binary nucleon--nucleon collisions ($N_\mathrm{coll}$) in Au+Au
% collisions. The D$^0$ meson, being the lightest hadron
% that contains a charm quark, provides, therefore, an excellent calibrated probe
% to the behavior of the medium. Recent measurements of the D$^0$ meson 
% at Relativistic Heavy-Ion Collider (RHIC)
% and the Large Hadron Collider (LHC)~\cite{publishedDzero, DRaaALICE, Dv2ALICE} show suppression of yields at
% high transverse momenta ($p_\mathrm{T}$) and suggest a non-zero $v_2$
% at intermediate to high $p_\mathrm{T}$\@. Measurements of better precision are, however, needed
% to provide more stringent constraints on model calculations.
% 
% \begin{figure}[htb]
% \centering
% \includegraphics[width=0.7\textwidth]{img/inclusiveCinAuAu}
% \caption{Charm cross-section at mid-rapidity from p+p to central Au+Au collisions~\cite{GuannanQM15,dAuD0,publishedDzero,pp200GeVcharm,totalCharmCrossSection}.}
% \label{fig:nCollScaling}
% \end{figure}
% 
% The D$_\mathrm{s}$ meson, which contains a strange quark,
% provides an additional handle on the ha\-dro\-ni\-za\-tion process of charm quarks. Recent 
% calculations~\cite{DsTAMU}
% suggest an enhancement of the D$_\mathrm{s}$ meson yield compared to the D$^0$ meson because 
% of the process of quark coalescence.


\section{D$^0$ and D$^\pm$ measurements}
In p+p collisions, non-strange D-mesons are the most abundant hadrons including charm quarks. This makes them excellent tools for measuring the properties of the charm quarks. 
\subsection{\dzero\ \Raa\ and \Rcp\ nuclear modification factors}
\begin{figure}[!htb]
\begin{center}
 \includegraphics[width=0.5\textwidth]{img/D0_RAA}\\
\end{center}
\caption{\label{dzeroRAA}D$^0$ $R_\mathrm{AA}$ measured in 0--10$\,\%$~(a), 10--20$\,\%$~(b), and 20--40$\,\%$~(c) central Au+Au collisions at $\sqrt{s_\mathrm{NN}} = \SI{200}{\giga\electronvolt}$ as a function of $p_\mathrm{T}$\@. The
light and dark green vertical bands around unity are uncertainties related
to the $N_\mathrm{coll}$ in Au+Au collisions and the global normalization in the p+p collisions,
respectively~\cite{D0paper, publishedDzero}.}
\end{figure}
Even if the inclusive charm quark production scales with $N_\mathrm{coll}$,
the D$^0$ spectrum shape can be significantly modified in Au+Au collisions. Fig.~\ref{dzeroRAA}
shows the D$^0$ nuclear modification factor
$R_\mathrm{AA}$
as a function of $p_\mathrm{T}$ in the several centrality regions of the Au+Au collisions, where $R_\mathrm{AA}$ is the ratio between the yield in Au+Au collisions
and that in p+p collisions scaled by $N_\mathrm{coll}$\@.
\begin{equation}
R_\mathrm{AA} = \frac{\dd N_\mathrm{AA}/\dd p_\mathrm{T}}
{\langle N_\mathrm{coll} \rangle\, \dd N_\mathrm{pp}/\dd p_\mathrm{T}}\,.
\end{equation}
The new results
(black circles)~\cite{D0paper} obtained with the HFT (see Section~\ref{HFTsection}) are consistent with the published
$R_\mathrm{AA}$ from 1.1$\,$B minimum-bias events taken in the years 2010 and 2011 
without the HFT (blue empty diamonds)~\cite{publishedDzero}\@.
For the new results, a much better precision is achieved despite the less statistics used. 
Compared to p+p collisions, the D$^0$ production is significantly suppressed at low and high $p_\mathrm{T}$ in central collisions which indicates
strong interactions between charm quarks and the medium.

\begin{figure}[!htb]
\begin{center}
 \includegraphics[width=0.5\textwidth]{img/D0_Rcp22}\\
\end{center}
\caption{\label{dzeroRcp}D$^0$ $R_\mathrm{cp}$~\cite{D0paper}, with the 40--60$\,\%$ centrality spectrum as a reference, measured in 0--10$\,\%$~(a), 10--20$\,\%$~(b), and 20--40$\,\%$~(c) central Au+Au collisions at $\sqrt{s_\mathrm{NN}} = \SI{200}{\giga\electronvolt}$ as a function of \pt\@. The
light and dark green vertical bands around unity are uncertainties related
to the $N_\mathrm{coll}$ in Au+Au collisions and the global normalization in the p+p collisions,
respectively. The gray bands around unity depict the systematic uncertainty due to corrections on the uncertainty of the position of the primary vertex. The measurement is compared to model calculations~\cite{Duke, Duke2015, LBT, LBTprivate}.}
\end{figure}


STAR has recently published a high-precision measurement of the D$^0$  nuclear-modification factor $R_\mathrm{cp}$ (40--60$\,\%$)\@. The $R_\mathrm{cp}$, defined as
\begin{equation}
R_\mathrm{cp}\text{ (centrality2)} = \frac{\langle N_\mathrm{coll}\text{ (centrality2)}\rangle }
{\langle N_\mathrm{coll}\text{ (centrality1)} \rangle}
\cdot
\frac{\dd N_\text{centrality1}/\dd p_\mathrm{T}}
 {\dd N_\text{centrality2}/\dd p_\mathrm{T}}\,,
\end{equation}
compares the central Au+Au collisions to the peripheral ones. This measurement greatly benefits from the precision of the HFT, since it can be used in both central and peripheral collisions. They can be claculated, using the formula





The D$^0$ \Rcp\ measurement is shown in Figure~\ref{dzeroRcp}. In the high-\pt\ region ($\pt \gtrsim 3\,$GeV), the \dzero\ are increasingly suppressed with centrality. The \Rcp\ is compared to two models, including charm-quark diffusion. The Duke model~\cite{Duke, Duke2015} uses Langevin calculation of the charm-quark diffusion. Both radiative and collisional energy loss processes are included in this model. The hadronization of the charm quark is calculated in via a hybrid approach that combines both fragmentation and coalescence. In the LBT (Linearized Boltzmann Transport) model~\cite{LBT, LBTprivate} a jet-transport framework is extended to include heavy quarks. The hadronization is handled in the same hybrid approach as in the Duke model. These models also agree with the \dzero\ $v_2$ measurement shown in the next section.

\begin{figure}[!htb]
\begin{center}
 \includegraphics[width=0.7\textwidth]{img/Dpm_RaaRun16}\\
\end{center}
\caption{\label{dpm} D$^\pm$ meson \Raa\ in 0--10$\,\%$ most central Au+Au collisions. The gray bands depict the D$^0$ uncertainty in p+p collisions~\cite{posterQmVanek,publishedDzero}. The dark blue (yellow) band refers to the global uncertainty in Au+Au (p+p) collisions.}
\end{figure}

As an important crosscheck, STAR has measured the D$^\pm$ meson production as well. Figure~\ref{dpm} shows the nuclear modification factor \Raa\ in Au+Au collisions, recorded in 2016~\cite{posterQmVanek}\@. The D$^\pm$ production is consistent with the \dzero~\cite{publishedDzero} in the whole \pt\ range.

\subsection{\dzero\ azimuthal anisotropy, $v_2$, $v_3$, and the charm diffusion coefficient}

Another way of probing the interactions of the charm quark with the sQGP medium is measuring the distribution of charmed hadrons in the azimuthal angle $\dd N/ \dd \phi$~\cite{StrangeAndChargedv2paper}\@. To evaluate the properties of this distribution, it is usually recalculated in terms of the Fourier expansion
\begin{equation}
\frac{\dd N}{ \dd \phi} = N \left(1 + 2\sum_{i=1}^\infty v_i \cos[i(\phi - \psi_i)] \right)
\end{equation}
where $\psi_i$ is called the $i$-th event plane angle. The first three $v_i$ coefficients are called the direct flow, elliptic flow, and triangular flow coefficients for $v_1$, $v_2$, and $v_3$, respectively. They can be measured using the formula
\begin{equation}
v_i = \langle \cos[i(\phi - \psi_i)] \rangle\,,
\end{equation}
where the angular brackets denote the average over all particles, summed over all events.

Normally, the system relaxes from the anisotropies of pressure and energy density in the initial state. The change in pressure normally drives the $v_i$ of the light hadrons as they move from areas with large pressure into areas with lower. Charm quarks are, however, created in hard collisions during the early stages of the evolution of the medium and because their high masses are much higher than the QCD scale $m_\mathrm{c} \gg \Lambda_\text{QCD}$, their masses are not affected by the QCD medium. This makes the c-quarks compelling probes to the behavior of the medium as they are carried away in a similar fashion to Brownian particles in liquids or gases. 

\begin{figure}[!htb]
\begin{center}
 \includegraphics[width=0.6\textwidth]{img/D0v2_080_withModels-eps-converted-to}\\
\end{center}
\caption{\label{dzerov2}D$^0$ $v_2$ as a function of $p_\mathrm{T}$ for Au+Au collisions
at $\sqrt{s_\mathrm{NN}} = \SI{200}{\giga\electronvolt}$~\cite{D0v2paper}, compared to theoretical calculations~\cite{PHSD2014,LBT,LBTprivate,TAMU,PHSD2015,SUBATECHvn,SUBATECHquenching,Duke,Duke2015,Hydro2012,Hydro2015}.}
\end{figure}


The HFT enables the measurement of the D$^0$ $v_2$ for the first 
time at RHIC, as shown in Figure~\ref{dzerov2NCQ}\@. This result was calculated from 0--80$\,\%$ central Au+Au collisions at $\snn = 200\,$GeV recorded in 2014 and 2016. The vertical bars (brackets) indicate 
the statistical (systematic)
uncertainties while the gray bands represent the estimated non-flow contribution inferred
from D meson--hadron correlations in p+p collisions. The data show that the $v_2$
is significantly larger than 0 above $\SI{1.5}{\giga\electronvolt}/c$\@.

Several models \cite{Hydro2012,Hydro2015,PHSD2014,LBT,LBTprivate,TAMU,PHSD2015,SUBATECHvn,SUBATECHquenching,Duke,Duke2015} are compared to the
measurements of $R_\mathrm{AA}$ in Figure~\ref{dzeroRAA} and $v_2$ in 0--80$\,\%$ centrality 
in Figure~\ref{dzerov2}\@. The 3D viscous hydrodynamic calculation with AMPT\nomenclature{AMPT}{A Multi-Phase Transport} 
initial conditions, tuned to describe the light-hadrons $v_2$~\cite{Hydro2012,Hydro2015}, describes the D$^0$ $v_2$ data well 
which is another hint that points to the thermal equilibrium between the c-quarks and the medium.

\begin{figure}[!htb]
\begin{center}
  \includegraphics[width=0.6\textwidth]{img/D0v2_1040_Data_Light-eps-converted-to}\\
\end{center}
\caption{\label{dzerov2NCQ}D$^0$ $v_2/n_\mathrm{q}$, where $n_\mathrm{q}$ is the number of valence quarks, as a function of $(m_\mathrm{T} - m_0)/n_\mathrm{q}$ for Au+Au collisions
at $\sqrt{s_\mathrm{NN}} = \SI{200}{\giga\electronvolt}$ with centrality of 10--40$\,\%$ compared to strange hadrons $v_2/n_\mathrm{q}$~\cite{D0v2paper,StrangeAndChargedv2paper}.}
\end{figure}


Another way of parametrizing the $v_2$ is using the number of valence quarks $n_\mathrm{q}$ and the, so called, transverse mass $m_\mathrm{T}= \sqrt{\pt^2 - m_0^2}$, where $m_0$ is the rest mass of the particle. Figure~\ref{dzerov2NCQ} shows $v_2 / n_\mathrm{q}$ of the \dzero~\cite{D0v2paper} and several lighter hadrons~\cite{StrangeAndChargedv2paper} plotted versus $(m_\mathrm{T} - m_0)/n_\mathrm{q}$ in 10--40$\,\%$ central Au+Au collisions\@. The $v_2/n_\mathrm{q}$ of the light-flavor hadrons follows the same pattern with relatively high precision. This phenomenon is called the Number-Of-Constituent-Quarks (NCQ\nomenclature{NCQ scaling}{Number-Of-Constituent-Quarks scaling}) scaling. Remarkably, as shown in this Figure, the D$^0$ follow the NCQ scaling of the light-flavor hadrons at the RHIC top energy. This suggests that the charm quarks from D$^0$ were close to thermal equilibrium with the sQGP medium.

% \begin{figure}[!htb]
% \begin{center}
%   \includegraphics[height=5.5cm]{img/D0_triangular_NCQ}
%   \includegraphics[height=5.6cm]{img/D0_triangular_model}
% \end{center}
% \caption{\label{dzerov3NCQ}D$^0$ $v_3/n_\mathrm{q}$, .}
% \end{figure}
% 
% The triangular flow $v_3$ can be measured at STAR thanks to the HFT as well. Figure~\ref{dzerov3NCQ}


Models that incorporate diffusion of the charm quark with various values of the diffusion coefficient multiplied by 
the temperature $2\pi T D_s$~\cite{PHSD2014,LBT, LBTprivate, TAMU,PHSD2015,SUBATECHvn,SUBATECHquenching,Duke,Duke2015} were compared to the \Raa,  $v_2$, and $v_3$ data. The PHSD\nomenclature{PHSD}{Parton-Hadron-String Dynamics} 
(Parton-Hadron-String Dynamics --- pink dash-double-dot line
in Figure~\ref{dzerov2}) model~\cite{PHSD2014,PHSD2015} uses the dynamical quasiparticle model to calculate 
an effective potential of the charm quark to the medium. It describes the data with $2\pi T D_s \sim$5--12.
It is consistent with the data with the predicted transport coefficient of $2\pi T D_s \sim$3--6.
The group from TAMU\nomenclature{TAMU}{Texas A\&M University}~\cite{TAMU} (blue in~\ref{dzeroRAA} and full green lines in~\ref{dzerov2}) employs
a non-perturbative T-matrix approach with the assumption
that two-body interactions can be described by a potential, which is a function of the transferred
4-momentum. Two scenarios of this model are plotted in Figure~\ref{dzerov2}: one with no c-quark diffusion (lighter green) 
and one where the c-quark diffuses (darker green). The data clearly prefer the latter scenario in which
% The comparison between STAR measurements and this model strongly favors a scenario where the
c-quarks flow.
This model predicts the charm quark diffusion coefficient multiplied by temperature as
$3 \leq 2\pi T D_s \lesssim 11$\@.
The SUBATECH group~\cite{SUBATECHvn}\nocite{SUBATECHquenching} (green in~\ref{dzeroRAA} and dashed red in~\ref{dzerov2}) uses a pQCD approach 
with the Hard Thermal Loop approximation for soft collisions.
In this approach, the diffusion coefficient is within $2 \leq 2\pi T D_s \leq 4$\@.
The model by the Duke university group~\cite{Duke,Duke2015} uses
$2\pi T D_s$ as a free parameter. The full red curve shown in Figure~\ref{dzeroRAA} and the cyan dot-dash curve from~\ref{dzerov2}b use
the value $2\pi T D_s = 7$ which is fixed to match the D$^0$ $R_\mathrm{AA}$ measured at
the LHC\@. The Duke model can describe the shape of $R_\mathrm{AA}$ well, however it systematically
underestimates the $v_2$\@. The other two models are consistent with 
both $R_\mathrm{AA}$ and $v_2$ data. The inferred estimates of the diffusion coefficient of $2 \lesssim 2 \pi D_s < 12$ are 
consistent with the lattice QCD calculations~\cite{DingLattice,BanerjeeLattice}\nocite{continuumEstimate,Tmatrix,SvetinskyDiffusion} 
in the sensitive temperature range, corresponding to $\sqrt{s_\mathrm{NN}} = \SI{200}{\giga\electronvolt}$ (between $\sim$1--2~$T_\mathrm{c}$, 
where $T_\mathrm{c}$ is the temperature of the critical point). 

\section{Measurement of the $\overline{\text{D}^0}/$D$^0$ ratio}

\begin{figure}[!htb]
\begin{center}
 \includegraphics[width=0.5\textwidth]{img/D0_spectra_ratioposneg_fit}\\
\end{center}
\caption{\label{dzeroRatio}$\overline{\dzero}/\dzero$ invariant yield ratio at mid-rapidity ($|y| < 1$). The dashed lines illustrate constant function fits to the $\overline{\dzero}/\dzero$ ratios~\cite{D0paper}.}
\end{figure}

So far, we have discussed combined measurements of both D$^0$ and its antiparticle $\overline{\text{D}^0}$\@. The charm quarks are produced in pairs, therefore, we would expect the same number of D$^0$ and $\overline{\text{D}^0}$\@. The Statistical-Hadronization Model~\cite{SHM_LcRatio}, however, suggest that the $\overline{\Lambdac^-}/\Lambdac^+$ is lower than unity and the c-quarks can get, therefore, depleted, compared to $\overline{\text{c}}$-quarks. Figure~\ref{dzeroRatio} shows the $\overline{\text{D}^0}/$D$^0$ invariant-yields ratio plotted versus \pt, together with constant function fits of the data~\cite{D0paper}\@.  Although the ratio is at 1 when divided into \pt\ bins, the overall ratio is significantly larger than unity in the 0--60$\,\%$ most central collisions. E.g. in the 0--10$\,\%$ most central collisions, the fit is $4.9\,\sigma$ higher than unity.



\section{D$_\text{s}$ measurements}

Thanks to the HFT, the $\mathrm{D_s}$ meson, consisting of a charm quark and a strange quark,
is measured for the first time at RHIC\@. Such measurements are expected to
shed more light on the mechanism of the charm quark coalescence.

\begin{figure}[!htb]
\begin{center}
  \includegraphics[width=0.6\textwidth]{img/Ds_D0_ratio_a}\\
\end{center} 
\caption{\label{Ds}The ratio between D$_\mathrm{s}$ and D$^0$ yield in
Au+Au collisions at $\sqrt{s_\mathrm{NN}} = \SI{200}{\giga\electronvolt}$
in 0--10$\,\%$ (blue) and 10--40$\,\%$ (red) central in Au+Au
collisions at $\sqrt{s_\mathrm{NN}} = \SI{200}{\giga\electronvolt}$.}
% \vspace{-6px}
\end{figure}

In Figure~\ref{Ds}a, the yield ratio of produced 
D$_\mathrm{s}$ to D$^0$ is shown as a function of $p_\mathrm{T}$ in 10--40$\,\%$ central
Au+Au collisions at $\sqrt{s_\mathrm{NN}} = \SI{200}{\giga\electronvolt}$
(red circles)\@. The D$^0$ spectrum is obtained from the published STAR 
data~\cite{publishedDzero}\nocite{DavidThesis}\@. The measured D$_\mathrm{s}$/D$^0$
ratios are compared to similar measurements at the LHC for minimum-bias Pb+Pb collisions~\cite{ALICEds}
(red squares)\@. Both of the measurements are consistent with each other within uncertainties. 
To compare our measurement to the D$_\mathrm{s}$/D$^0$ in p+p collisions, 
PYTHIA 6.4~\cite{PYTHIA} (purple curve) is used\@. The STAR measurement is slightly
enhanced compared to the p+p ratio; however, the enhancement is statistically insignificant.

In Figure~\ref{Ds}b, the nuclear modification factor for D$_\mathrm{s}$ is shown as blue points. 
The p+p baseline is obtained from
the measured total charm cross-section by STAR~\cite{inclusiveC} multiplied by the 
$\mathrm{c}\rightarrow\mathrm{D_s}$
fragmentation factor obtained from the measurements at HERA~\cite{H1,ZEUS}\@.
The uncertainty of the baseline is indicated by the
green hashed band and the uncertainty on $N_\mathrm{coll}$ is plotted
as the black rectangle.
The D$_\mathrm{s}$ $R_\mathrm{AA}$ is compared to the calculation done by the TAMU
group~\cite{DsTAMU} and the published D$^0$ $R_\mathrm{AA}$ by
STAR~\cite{publishedDzero}\@. Again, we observe a hint of D$_\mathrm{s}$ enhancement compared 
to D$^0$, which can be described
by the TAMU model within uncertainties; however, more data is needed to draw firmer conclusions.

Figure~\ref{Ds}c shows the first result of D$_\mathrm{s}$ $v_2$ at RHIC as the red filled square.
The data slightly prefer a non-zero $v_2$, albeit not significantly. The D$_\mathrm{s}$ $v_2$ is
compared to the $v_2$ of D$^0$ and $\upphi$
meson~\cite{phi} which, like the D$_\mathrm{s}$, contains strange quarks. 
The D$_\mathrm{s}$ $v_2$ is consistent with both measurements within uncertainties.


